% Kategorientheorie-Makros

% Konzepte
\DeclareMathOperator{\Ob}{Ob} % Objekte (einer Kategorie)
\DeclareMathOperator{\Mor}{Mor} % Morphismenmenge / -klasse
\DeclareMathOperator{\End}{End} % Endomorphismenmenge / -klasse
\DeclareMathOperator{\Hom}{Hom} % Homomorphisms
\DeclareMathOperator{\Nat}{Nat} % Natürliche Transformationen
\DeclareMathOperator{\dom}{dom} % Domain
\DeclareMathOperator{\codom}{codom} % Codomain
\newcommand{\op}{\mathrm{op}} % opposite category
\DeclareMathOperator{\Aut}{Aut} % Automorphismengruppe
\newcommand{\ladj}{\dashv} % Links-adjungiert (left-adjoint)
\newcommand{\Lim}{\lim} % Limes
\DeclareMathOperator{\colim}{colim} % Kolimes
\newcommand{\Colim}{\colim} % Kolimes
%\newcommand{\myint}[2]{{\textstyle \int\limits_{#1}^{#2}}}
\newcommand{\EndC}[2]{\myint{#1}{} #2} % Ende
\newcommand{\CoEndC}[2]{\myint{}{#1} #2} % Koende
\DeclareMathOperator{\Ran}{Ran} % Rechts-Kan-Erweiterung
\DeclareMathOperator{\Lan}{Lan} % Links-Kan-Erweiterung
\newcommand{\IHom}{\underline{\Hom}} % Inner Homomorphisms

% Platzhalter
\newcommand{\DiaTodo}{\fcolorbox{red}{white}{TODO: Diagramm einfügen!}}

% Populäre Kategorien
\newcommand{\SetC}{\mathbf{Set}} % Kategorie der Mengen
\newcommand{\FinSetC}{\mathbf{FinSet}} % Kategorie der endlichen Mengen
\newcommand{\sSet}{\mathbf{sSet}} % Kategorie der simplizialen Mengen
\newcommand{\Top}{\mathbf{Top}} % Kategorie der topologischen Räume
\newcommand{\AbGrp}{\mathbf{Ab}} % Kategorie der abelschen Gruppen
\newcommand{\Grp}{\mathbf{Grp}} % Kategorie der Gruppen
\newcommand{\Ouv}{\mathbf{Ouv}} % Kategorie der offenene Mengen eines topol. Raumes
\newcommand{\KHaus}{\mathbf{KHaus}} % Kategorie der kompakten Hausdorffräume
\newcommand{\CatC}{\mathbf{Cat}} % Kategorie der kleinen Kategorien
\newcommand{\Vect}{\mathbf{Vect}} % Kategorie der Vektorräume über einem Körper
\newcommand{\Field}{\mathbf{Field}} % Kategorie der Körper
\newcommand{\Ring}{\mathbf{Ring}} % Kategorie der Ringe
\newcommand{\Alg}{\mathbf{Alg}} % Kategorie der Algebren über einem Körper/Ring
\newcommand{\VectFin}{\mathbf{Vect}_{\mathrm{fin}}} % Kategorie der endlichen Vektorräume über einem Körper
\newcommand{\kVect}{\text{$k$-$\Vect$}} % Kategorie der k-Vektorräume über einem Körper k
\newcommand{\kVectFin}{\text{$k$-$\VectFin$}} % Kategorie der endlichen k-Vektorräume über einem Körper k
\newcommand{\Mod}{\mathbf{Mod}} % Kategorie der Moduln über einem Ring
\newcommand{\LMod}[1]{{#1}\text{-}\Mod} % Kategorie der (#1)-Linksmoduln
\newcommand{\RMod}[1]{\Mod\text{-}{#1}} % Kategorie der (#1)-Rechtsmoduln
\newcommand{\BMod}[2]{{#1}\text{-}\Mod\text{-}{#2}} % Kategorie der (#1)-(#2)-Bimoduln
\newcommand{\Kom}{\mathbf{Kom}} % Kategorie der Komplexe in einer abelschen Kategorie
\newcommand{\Der}{\mathcal{D}} % abgeleitete Kategorie einer abelschen Kategorie
\newcommand{\kAlg}{k\text{-}\Alg} % Kategorie der k-Algebren

% Bezeichnungen für Variablen, die für Kategorien stehen
\newcommand{\Aat}{\mathcal{A}} % Category-A
\newcommand{\Bat}{\mathcal{B}} % Category-B
\newcommand{\Cat}{\mathcal{C}} % Category-C
\newcommand{\Dat}{\mathcal{D}} % Category-D
\newcommand{\Eat}{\mathcal{E}} % Category-E
\newcommand{\Fat}{\mathcal{F}} % Category-F
\newcommand{\Gat}{\mathcal{G}} % Category-G
\newcommand{\HatC}{\mathcal{H}} % Category-H
\newcommand{\Iat}{\mathcal{I}} % Category-I (Indexkategorie)
\newcommand{\Jat}{\mathcal{J}} % Category-J (Indexkategorie)
\newcommand{\Lat}{\mathcal{L}} % Category-L
\newcommand{\MatC}{\mathcal{M}} % Category-M
\newcommand{\NatC}{\mathcal{N}} % Category-N
\newcommand{\Sit}{\mathcal{S}} % Situs-S
\newcommand{\Sat}{\mathcal{S}} % Category-S
\newcommand{\Wat}{\mathcal{W}} % Category-W (whaaaatt?)
\newcommand{\Xat}{\mathcal{X}} % Category-X
\newcommand{\Yat}{\mathcal{Y}} % Category-Y

% Makros für Adjunktionen. Geklaut von
% http://sma.epfl.ch/~werndli/latex/adjunction.pdf
% http://sma.epfl.ch/~werndli/latex/adjunction.tex
\usepackage[all]{xy}
\newcommand{\adj}[1][]{\def\ArgI{#1}\adjRelayI}
\newcommand{\adjRelayI}[1][]{\def\ArgII{#1}\adjRelayII}
\newcommand{\adjRelayII}[3][2.2em]{
  \ensuremath{\SelectTips{lu}{10}\xymatrix@C=#1@1{
    {#2\,}
    \ar@<1ex>[r]^-{\ArgI}^-{}="1" &
    {\,#3}
    \ar@<1ex>[l]^-{\ArgII}^-{}="2"
    \ar@{}"1";"2"|(.3){\hbox{}}="7"
    \ar@{}"1";"2"|(.7){\hbox{}}="8"
    \ar@{|-} "8" ;"7"
  }}
}
\newcommand{\radj}[1][]{\def\ArgI{#1}\radjRelayI}
\newcommand{\radjRelayI}[1][]{\def\ArgII{#1}\radjRelayII}
\newcommand{\radjRelayII}[3][2.2em]{
  \ensuremath{\SelectTips{lu}{10}\xymatrix@C=#1@1{
  {#2\,}
  \ar@<-1ex>[r]_-{\ArgI}^-{}="1" &
  {\,#3}
  \ar@<-1ex>[l]_-{\ArgII}^-{}="2"
  \ar@{}"1";"2"|(.3){\hbox{}}="7"
  \ar@{}"1";"2"|(.7){\hbox{}}="8"
  \ar@{|-} "7" ;"8"
  }}
}
