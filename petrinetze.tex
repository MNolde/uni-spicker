\documentclass{cheat-sheet}

\pdfinfo{
  /Title (Zusammenfassung Petrinetze)
  /Author (Tim Baumann)
}

\usepackage{algorithmicx}
\usepackage[noend]{algpseudocode}

\newcommand{\transition}{\square} % Transition
\newcommand{\place}{\bigcirc} % Stelle, Platz
\newcommand{\preset}[1]{{}^\bullet{#1}} % Vorbereich
\newcommand{\postset}[1]{{#1}^\bullet} % Nachbereich
\newcommand{\activeTransition}[1]{[{#1}\rangle} % aktive Transition
\DeclareMathOperator{\FS}{FS} % Menge der Schaltfolgen (firing sequences)
\newcommand{\ReachabilityGraph}{\mathfrak{R}} % Erreichbarkeitsgraph
\newcommand{\StepReachabilityGraph}{\mathfrak{SR}} % Schritt-Erreichbarkeitsgraph
\DeclareMathOperator{\Parikh}{Parikh} % Parikh-Bild
\newcommand{\inferrule}[2]{\frac{{#1}}{{#2}}} % logische Inferenzregel
\newcommand{\Markings}{\mathfrak{M}} % Menge von Markierungen
\newcommand{\ExtMarkings}{\mathfrak{M}^\omega} % Menge von erweiterten Markierungen
\DeclareMathOperator{\Fin}{Fin} % Menge von Endzuständen
\newcommand{\Lang}{\mathfrak{L}} % Sprachen von beschrifteten Netzen mit Endmarkierung
\newcommand{\fin}{\mathrm{fin}} % endlich
\newcommand{\Multisets}{\mathfrak{M}} % Multimengen
\DeclareMathOperator{\StepSequences}{SS} % Schrittfolgen
\newcommand{\SInv}{S\text{-}\mathrm{Inv}} % S-Invarianten
\newcommand{\TInv}{T\text{-}\mathrm{Inv}} % T-Invarianten
\newcommand{\reducesTo}{\mapsto} % reduziert zu (entscheidungsprobleme)
\newcommand{\reducesManyOneToLin}{\xmapsto{\mathrm{lin}}_M} % reduziert zu ... in linearer Zeit (entscheidungsprobleme)
\newcommand{\reducesManyOneToPoly}{\xmapsto{\mathrm{poly}}_M} % reduziert zu ... in polynomieller Zeit (entscheidungsprobleme)
\DeclareMathOperator{\Cov}{Cov} % Überdeckungsgraph
\newcommand{\nil}{\mathbf{nil}} % NULL

% Kleinere Klammern
\delimiterfactor=701

\setlength{\tabcolsep}{2pt}

\begin{document}

\raggedcolumns % stretche Inhalt nicht über die gesamte Spaltenhöhe

\maketitle{Zusammenfassung Petrinetze}

% §2. Grundbegriffe

% 2.1
\begin{defn}
  Ein \emph{Netzgraph} ist ein Tripel $(S, T, W)$, wobei~$S$ und~$T$ disjunkte, endliche Mengen sind und $W : S \times T \cup T \times S \to \N$.
  Dadurch ist ein gerichteter, gewichteter, bipartiter Graph mit Kantenmenge $F = \Set{(x, y)}{W(x, y) \neq 0}$ gegeben.
\end{defn}

\begin{center}
  \begin{tabular}{r l l}
    Notation & Bezeichnung & Symbol \\ \hline
    $t \in T$ & Transition & $\transition$ \\
    $s \in S$ & Stelle, Platz & $\place$ \\
    $(x, y) \in F$ & Kante & $\xrightarrow{\enspace}$ falls $W(x, y) = 1$ \\
    && $\xrightarrow{w}$ falls $w \coloneqq W(x, y) > 1$
  \end{tabular}
\end{center}

\begin{defn}
  Sei $x \in S \cup T$.
  \begin{itemize}
    \item $\preset{x} \coloneqq \Set{y}{(y, x) \in F}$ heißt \emph{Vorbereich} von~$x$ und
    \item $\postset{x} \coloneqq \Set{y}{(x, y) \in F}$ heißt \emph{Nachbereich} von~$x$.
    \item $x$ heißt \emph{isoliert}, falls $\preset{x} \cup \postset{x} = \emptyset$.
    \item $x$ heißt \emph{vorwärts-verzweigt}, falls $\card{\postset{x}} \geq 2$ 
    \item $x$ heißt \emph{rückwärts-verzweigt}, falls $\card{\preset{x}} \geq 2$ 
  \end{itemize}
\end{defn}

\begin{defn}
  $(x, y) \in S \times T \cup T \times S$ bilden eine \emph{Schlinge} falls $(x, y) \in F$ und $(y, x) \in F$.
\end{defn}

\begin{defn}
  Eine \emph{Markierung} ist eine Abbildung $M : S \to \N$. \\
  Eine Teilmenge $S' \subseteq S$ heißt \emph{markiert} unter~$M$, falls $\ex{s \in S'} M(s') > 0$, andernfalls \textit{unmarkiert}. \\
  Ein Element $s \in S$ heißt \textit{(un-)markiert}, falls $\{ s \} \subseteq S$ es ist.
\end{defn}

\begin{nota}
  $\Markings(S) \coloneqq \{ M : S \to \N \}$
\end{nota}

\begin{defn}
  Ein \emph{Petrinetz} $N = (S, T, W, M_N)$ besteht aus
  \begin{itemize}
    \item einem Netzgraphen $(S, T, W)$ und
    \item einer \textit{Anfangsmarkierung}~$M_N : S \to \N$.
  \end{itemize}
\end{defn}

\begin{nota}
  Für eine feste Transition $t \in T$ ist
  \[
    t^{-} : S \to \N, \enspace s \mapsto W(s, t), \qquad
    t^{+} : S \to \N, \enspace s \mapsto W(t, s)
  \]
\end{nota}

% 2.2
\begin{defn}
  Eine Transition $t \in T$ heißt \emph{aktiviert} unter einer Markierung~$M$, notiert $M \activeTransition{t}$, falls
  \[
    \fa{s \in S} W(s, t) \leq M(s) \iff
    t^{-} \leq M.
  \]
  Ist~$t$ aktiviert, so kann~$t$ \textit{schalten} und es entsteht die \textit{Folgemarkierung} $M' \coloneqq M + \Delta t$, wobei
  \[
    \Delta t : S \to \Z, \enspace s \mapsto W(t, s) - W(s, t).
  \]
\end{defn}

\begin{nota}
  $M \activeTransition{t} M'$
\end{nota}

\begin{defn}
  Für $w = t_1 \cdots t_n \in T^{*}$ und Markierungen~$M$ und~$M'$ gilt
  \[ M \activeTransition{w} M' \coloniff M \activeTransition{t_1} M_1 \activeTransition{t_2} \cdots \activeTransition{t_{n-1}} M_{n-1} \activeTransition{t_n} M' \]
  für (eindeutig bestimmte) Markierungen $M_1, \ldots, M_{n-1}$. \\
  Ein Wort $w \in T^{*}$ heißt \emph{Schaltfolge} (\textit{firing sequence}) von~$N$, notiert $M_N \activeTransition{w}$, falls $\ex{M'} M_N \activeTransition{w} M'$.
\end{defn}

\begin{nota}
  \begin{minipage}[t]{0.8 \linewidth}
    $\activeTransition{M} \coloneqq \Set{M'}{\ex{w \in T^{*}} M \activeTransition{w} M'}$ \\
    $\FS(N) \coloneqq \Set{w \in T^{*}}{M_N \activeTransition{w}}$ \enspace
    für ein Petrinetz~$N$
  \end{minipage}
\end{nota}

\begin{defn}
  $M'$ heißt \emph{erreichbar} von~$M$, falls $M' \in \activeTransition{M}$.
\end{defn}

\begin{defn}
  $w \in T^\omega$ heißt \emph{unendliche Schaltfolge} von~$N$, falls alle endlichen Präfixe von~$w$ Schaltfolgen von~$N$ sind.
\end{defn}

\begin{defn}
  Der \emph{Erreichbarkeitsgraph} $\ReachabilityGraph(N)$ zu~$N$ besitzt die Knoten~$\activeTransition{M_N}$ und die Kanten $\Set{(M, M')}{\ex{t} M \activeTransition{t} M'}$.
\end{defn}

% 2.5
\begin{defn}
  Für $w = a_1 \cdots a_n \in A^{*}$ ist $\Parikh(w) : A \to \N, \enspace a \mapsto \card{i}{a_i = a}$.
\end{defn}

% 2.6
\begin{lem}
  In $M \activeTransition{w} M'$ hängt~$M'$ nur von~$M$ und $\Parikh(w)$ ab, genauer
  \[
    M' = M + {\sum}_{t \in T} \Parikh(w)(t) \cdot \Delta t.
  \]
\end{lem}

% 2.7
\begin{lem}
  $M_1 \activeTransition{w} M_2 \implies M + M_1 \activeTransition{w} M + M_2$
\end{lem}

\TODO{Satz 2.8}

% 2.10
\begin{lem}
  Sei $N$ ein Petri-Netz.
  Dann gilt:
  \begin{itemize}
    \item $\FS(N)$ ist \textit{präfix-abg.}, \dh{} $w = v u \in \FS(N) \implies v \in \FS(N)$.
    \item Ist $\activeTransition{M_N}$ endlich, so ist $\FS(N)$ regulär.
  \end{itemize}
\end{lem}

% 2.11
\begin{defn}
  Ein \emph{beschriftetes Petrinetz} $N = (S, T, W, M_N, \ell)$ best. aus
  \begin{itemize}
    \item einem Petrinetz $(S, T, W, M_N)$ und
    \item einer Transitionsbeschriftung (\textit{labelling}) $\ell : T \to \Sigma \cup \{ \lambda \}$, wobei~$\Sigma$ eine Menge von \textit{Aktionen} ist.
  \end{itemize}
\end{defn}

\begin{sprechweise}
  $t \in T$ mit $\ell(t) = \lambda$ heißt \textit{intern} oder \textit{unsichtbar}.
\end{sprechweise}

\begin{nota}
  \begin{minipage}[t]{0.8 \linewidth}
    Für $t \in T^{*}$ ist $\ell(w) \coloneqq \ell(t_1) \cdots \ell(t_n) \in \Sigma^{*}$. \\
    Dabei wird $\lambda$ als das leere Wort in $\Sigma^{*}$ aufgefasst.
  \end{minipage}
\end{nota}

\begin{defn}
  Mit $t \in T$, $w \in T^{*}$ und Markierungen $M$, $M'$ ist definiert:
  \[
    \inferrule
      {M \activeTransition{t} M'}
      {M \activeTransition{\ell(t)} M'} \quad
    \inferrule
      {M \activeTransition{t}}
      {M \activeTransition{\ell(t)}} \quad
    \inferrule
      {M \activeTransition{w} M'}
      {M \activeTransition{\ell(w)} M'} \quad
    \inferrule
      {M \activeTransition{w}}
      {M \activeTransition{\ell(w)}}
  \]
\end{defn}

\begin{defn}
  Die \emph{Sprache} eines beschrifteten Netzes~$N$ ist
  \[
    L(N) \coloneqq \Set{v \in \Sigma^{*}}{M_n \activeTransition{v}}.
  \]
\end{defn}

% 2.12
\begin{defn}
  Ein \emph{beschriftetes Netz mit Endmarkierung} ist ein Tupel $N = (S, T, W, M_N, \ell, \Fin)$ wobei
  \begin{itemize}
    \item $(S, T, W, M_N, \ell)$ ein beschriftetes Netz und
    \item $\Fin \subseteq \Markings(S)$ eine endliche Menge ist.
  \end{itemize}
  Die entspr. Sprache ist $L_\fin(N) \coloneqq \Set{v \in \Sigma^{*}}{\ex{M \in \Fin} M_N \activeTransition{v} M}$.
\end{defn}

\begin{nota}
  $\Lang^\lambda \coloneqq \Set{L_\fin(N)}{N \text{ beschr. Netz mit Endmarkierung}}$ \\
  $\Lang \coloneqq \Set{L_\fin(N)}{N \text{ beschr. Netz mit Endmark. ohne interne Trans.}}$
\end{nota}

% 2.13
\begin{satz}
  $\{ \text{ reguläre Sprachen } \} \subseteq \Lang$
\end{satz}

\subsection{Nebenläufigkeit I}

% 2.14
\begin{defn}
  Eine Multimenge über~$X$ ist eine Funktion $\mu : X \to \N$.
\end{defn}

\begin{nota}
  \begin{minipage}[t]{0.8 \linewidth}
    $\Multisets(X) \coloneqq \{ \mu : X \to \N \}$ \\
    $\mu_Y \in \Multisets(X), x \mapsto \card{\Set{\star}{x \in Y}}$ für $Y \subset X$, \\
    $\emptyset \coloneqq \mu_\emptyset \in \Multisets(X)$, \enspace
    $\mu_x \coloneqq \mu_{\{ x \}} \in \Multisets(X)$ für $x \in X$
  \end{minipage}
\end{nota}

% 2.15
\begin{defn}
  Ein \emph{Schritt}~$\mu$ ist eine Multimenge $\mu \neq \emptyset \in \Multisets(T)$. \\
  Der Schritt~$\mu$ ist \emph{aktiviert} unter~$M$, notiert $M \activeTransition{\mu}$, falls
  \[
    \fa{s \in S} \mu^{-}(s) \coloneqq {\sum}_{t \in T} \mu(t) W(s, t) \leq M(s).
  \]
  Durch \textit{Schalten} von~$\mu$ entsteht die Folgemarkierung $M' \in \Markings(S)$ mit
  \[
    M'(s) = M(s) + {\sum}_{t \in T} \mu(t) \cdot (W(t, s) - W(s, t)).
  \]
\end{defn}

\begin{bem}
  Analog wird verallgemeinert:
  $M \activeTransition{\mu} M'$, $M \activeTransition{w}$, $M \activeTransition{w} M'$
  für $\mu \in \Markings(T) \setminus \{ \emptyset \}$ bzw. $w \in (\Markings(T) \setminus \{ \emptyset \})^{*}$.
\end{bem}

\begin{defn}
  \begin{minipage}[t]{0.8 \linewidth}
    $\StepSequences(N) \coloneqq \Set{w \in (\Markings(T) \setminus \{ \emptyset \})^{*}}{M_N \activeTransition{w}}$ \\
    heißen \emph{Schrittfolgen} (\textit{step sequences}).
  \end{minipage}
\end{defn}

\begin{defn}
  Zwei Transitionen $t, t' \in T$ sind
  \begin{itemize}
    \item \emph{nebenläufig} \textit{unter~$M$}, falls $M \activeTransition{t + t'}$,
    \item \emph{in Konflikt} \textit{unter~$M$}, falls $\neg M \activeTransition{t + t'}$.
  \end{itemize}
\end{defn}

\begin{nota}
  Für $\mu \in \Multisets(T)$ ist $\ell(\mu)$ die Multimenge mit
  \[
    \ell(\mu) : \Sigma \to \N, x \mapsto {\sum}_{t \in T, \ell(t) = x} \mu(t)
  \]
  (falls die rechte Zahl endlich ist für alle $x \in \Sigma$). \\
  Für $w = \mu_1 \cdots \mu_n \in \Multisets(T)^{*}$ ist 
  $\ell(w) \coloneqq \ell(\mu_1) \cdots \ell(\mu_n)$.
\end{nota}

\begin{defn}
  Mit $\mu \in \Markings(T) \setminus \{ 0 \}$, $w \in (\Markings(T) \setminus \{ 0 \})^{*}$ und $M$, $M'$ ist defin.:
  \[
    \inferrule
      {M \activeTransition{\mu} M'}
      {M \activeTransition{\ell(\mu)} M'} \quad
    \inferrule
      {M \activeTransition{\mu}}
      {M \activeTransition{\ell(\mu)}} \quad
    \inferrule
      {M \activeTransition{w} M'}
      {M \activeTransition{\ell(w)} M'} \quad
    \inferrule
      {M \activeTransition{w}}
      {M \activeTransition{\ell(w)}}
  \]
\end{defn}

% 2.16
\begin{lem}
  $M \activeTransition{t_1}, \ldots, M \activeTransition{t_n} \wedge \fa{i \neq j} \preset{t_i} \cap \preset{t_j} = \emptyset \implies M \activeTransition{t_1 + \ldots t_n}$
\end{lem}

% 2.17
\begin{lem}
  $M \activeTransition{\mu} M' \wedge \Parikh(w) = \mu \implies M \activeTransition{w} M'$
\end{lem}

\begin{bem}
  Über Schrittfolgen werden somit dieselben Markierungen erreicht wie über Schaltfolgen.
\end{bem}

\begin{defn}
  Der \emph{schrittweise Erreichbarkeitsgraph}~$\StepReachabilityGraph(N)$ besitzt die Knoten $\activeTransition{M}$ und die Kanten $\Set{(M, M')}{\ex{\mu \in \Markings(T) \!\setminus\! \{ \emptyset \}} M \activeTransition{\mu} M'}$.
\end{defn}

% 2.18
\begin{lem}
  Sei~$N$ schlingenfrei. Dann gilt:
  \[
    (\fa{w \in T^{*}, \Parikh(w) = \mu} M \activeTransition{w}) \iff M \activeTransition{\mu}
  \]
\end{lem}

% Erreichbarkeit

% (Problemdefinitionen nach hinten verschoben)

% 2.19
\begin{defn}
  Eine Stelle $s \in S$ heißt \emph{$n$-beschränkt} / \emph{beschränkt}, falls
  \[
    \sup \Set{M(s)}{M \in \activeTransition{M_N}} \leq n
    \quad / \quad
    \sup \Set{M(s)}{M \in \activeTransition{M_N}} < \infty.
  \]
  Ein Netz heißt ($n$-) \textit{beschränkt}, wenn alle Stellen $s \in S$ ($n$-) beschränkt sind.
  Ein Netz heißt \emph{sicher}, wenn es 1-beschränkt ist. \\
  % Def 3.6:
  Ein Netz heißt \emph{strukturell beschränkt}, wenn es bei beliebig geänderter Anfangsmarkierung beschränkt ist.
\end{defn}

% 2.20
\begin{prop}
  $\activeTransition{M_N}$ endlich $\iff$ $N$ beschränkt
\end{prop}

% Verklemmung

% 2.21
\begin{defn}
  Eine Transition $t \in T$ heißt \emph{tot} \textit{unter~$M$}, falls $\fa{M' \in \activeTransition{M}} \neg M' \activeTransition{t}$.
  \begin{itemize}
    \item $M$ heißt \textit{tot}, falls alle Transitionen unter~$M$ tot sind.
    \item $N$ heißt \textit{tot}, falls $M_N$ tot ist.
    \item $N$ heißt \emph{verklemmungsfrei}, falls $\fa{M \in \activeTransition{M_N}} \neg (M \text{ tot})$
    \item $t$ heißt \emph{lebendig} \textit{unter $M$}, falls $\fa{M' \in \activeTransition{M}} \neg (t \text{ ist tot unter~$M$})$
    \item $t$ heißt \textit{lebendig}, falls $t$ lebendig unter~$M_N$ ist.
    \item $M$ heißt \textit{lebendig}, wenn alle $t \in T$ unter~$M$ lebendig sind.
    \item $N$ heißt \textit{lebendig}, wenn $M_N$ lebendig ist.
  \end{itemize}
\end{defn}

% §3. S- und T-Invarianten
\section{$S$- und $T$-Invarianten}

% Ausgelassen: Beispiel mit 3 lesenden, zwei schreibenden Prozessen

% 3.1
\begin{defn}
  Die \emph{Inzidenzmatrix} \textit{eines Netzes~$N$} ist die Matrix $C(N) \in \Z^{\card{T} \times \card{S}}$ mit $C(N)_{s t} = \Delta t (s)$ für $s \in S$ und $t \in T$.
\end{defn}

\begin{bem}
  Folglich ist $\Delta t = C(N) \cdot t$ (wenn man $t$ als One-Hot-Vektor auffasst) und für $M \activeTransition{w} M'$ ist $M' = M + C(N) \cdot \Parikh(w)$.
\end{bem}

% 3.2
\begin{defn}
  Eine \emph{$S$-Invariante} $y : S \to \Z$ ist eine Lsg von $C(N)^T \cdot y = 0$. \\
  Der \emph{Träger} $\supp(y)$ \textit{einer $S$-Invarianten~$y$} ist $\Set{s \in S}{y(s) \neq 0}$. \\
\end{defn}

\begin{nota}
  $\SInv(N) \coloneqq \{ \text{ $S$-Invarianten von~$N$ } \} = \ker(C(N)^T)$
\end{nota}

\begin{lemdefn}
  Das Netz~$N$ heißt \emph{von $S$-Invarianten überdeckt}, falls folgende äquivalente Bedingungen gelten:
  \begin{itemize}
    \item $N$ besitzt eine positive (\dh{} $\fa{s \in S} y(s) > 0$) $S$-Invariante.
    \item Für alle $s \in S$ gibt es eine nichtnegative (\dh{} $\fa{s \in S} y(s) \geq 0$) $S$-Invariante mit $s \in \supp(y)$.
  \end{itemize}
\end{lemdefn}

% 3.3
\begin{lem}
  $y \in \SInv(N) \implies \fa{M \in \activeTransition{M_N}} y^T \cdot M = y^T \cdot M_N$
\end{lem}

\begin{bem}
  Das Lemma kann verwendet werden um zu zeigen, dass ein~$M$ nicht erreichbar ist.
\end{bem}

% 3.4
\begin{lem}
  Sei keine Transition in~$N$ tot.
  Dann gilt für $y \in \Z^S$:
  \[
    \fa{M \in \activeTransition{M_N}} y^T \cdot M = y^T \cdot M_N \implies
    y \in \SInv(N)
  \]
\end{lem}

% 3.5
\begin{lem}
  Sei $s \in S$ und $y \in \SInv(N)$ nichtnegativ mit $y(s) > 0$. \\
  Dann ist $s$ beschränkt, genauer $(y^T \cdot M_N / y(s))$-beschränkt.
\end{lem}

% 3.6
\begin{lem}
  Ist~$N$ von $S$-Invarianten überdeckt, so ist~$N$ strukturell beschränkt.
\end{lem}

\TODO{Umkehrung, siehe Buch von Starke}

\begin{defn}
  Ein \emph{home state} ist eine Markierung~$M$ mit
  \[
    \fa{M' \in \activeTransition{M}} M \in \activeTransition{M'}.
  \]
  Ein Netz~$N$ heißt \emph{reversibel}, wenn $M_N$ ein home state ist.
\end{defn}

\begin{lem}
  Angenommen, $N$ ist reversibel und keine Transitionen sind tot unter~$M_N$.
  Dann ist $N$ lebendig.
\end{lem}

\begin{bem}
  Es gibt lebendige, sichere Netze, die \textit{nicht} von $S$-Invarianten überdeckt sind.
\end{bem}

% 3.8
\begin{defn}
  Eine \emph{$T$-Invariante} $x : T \to \Z$ ist eine Lsg von $C(N) \cdot x = 0$.
  Das Netz~$N$ heißt \emph{von $T$-Invarianten überdeckt}, wenn es eine positive $T$-Invariante gibt.
\end{defn}

\begin{nota}
  $\TInv(N) \coloneqq \{ \text{ $T$-Invarianten von~$N$ } \} = \ker(C(N))$
\end{nota}

% 3.9
\begin{lem}
  Sei $w \in T^*$ mit $M \activeTransition{w} M'$.
  Dann gilt:
  \[
    \Parikh(w) \in \TInv(N) \iff M = M'
  \]
\end{lem}

\begin{satz}
  Ist~$N$ lebendig und beschränkt, so ist $N$ von $T$-Invarianten überdeckt.
\end{satz}

% §4. Einige Entscheidbarkeitsprobleme

\section{Einige Entscheidbarkeitsprobleme}

\begin{problem}[E -- \emph{Erreichbarkeit}]
  Gegeben seien ein Netz~$N$ und eine Markierung~$M$.
  Frage: Ist~$M$ erreichbar in~$N$?
\end{problem}

\begin{problem}[0-E -- \emph{$0$-Erreichbarkeit}]
  Gegeben seien ein Netz~$N$.
  Frage: Ist die Nullmarkierung erreichbar?
\end{problem}

\begin{bem}
  Diese Probleme sind lösbar, falls der Erreichbarkeitsgraph endlich ist.
\end{bem}

\begin{problem}[TE -- \emph{Teilerreichbarkeit}]
  Gegeben ein Netz~$N$, eine Teilmenge $S' \subseteq S$ und $M : S' \to \N$.
  Frage: Gibt es eine erreichbare Markierung~$M \in \Markings(S)$ mit $M|_{S'} = M'$?
\end{problem}

% ausgelassen: Definition 4.1 von Entscheidungsproblem und Entscheidungsverfahren

\begin{defn}
  \begin{itemize}
    \item Ein Entscheidungsproblem~$A$ ist auf ein Entscheidungs- problem~$B$ \emph{reduzierbar} (notiert $A \reducesTo B$), falls ein Lösungsalgo- rithmus für~$A$ existiert, welcher einen (vllt. gar nicht existenten!) Lösungsalgorithmus für~$B$ verwenden darf.
    \item $A$ ist \emph{linear / polynomiell many-one-reduzierbar} \textit{auf~$B$}, falls aus einer Instanz~$I$ von~$A$ in linearer / polynomieller Zeit eine Instanz~$I'$ von~$B$ berechnet werden kann, sodass die Antwort auf~$I$ gleich der Antwort auf~$I'$ ist.
    Notation: $A \reducesManyOneToLin B$ / $A \reducesManyOneToPoly B$
  \end{itemize}
\end{defn}

% 4.2
\begin{satz}
  $\text{(0-E)} \reducesManyOneToLin \text{(E)} \reducesManyOneToLin \text{(TE)}  \reducesManyOneToLin \text{(0-E)}$
\end{satz}

\begin{beweis}[$\text{(TE)}  \reducesManyOneToLin \text{(0-E)}$]
  Konstruiere $\overline{N} = (\overline{S}, \overline{T}, \overline{W}, M_{\overline{N}})$ mit
  \begin{align*}
    \overline{S} & \coloneqq S \amalg \Set{\overline{s'}}{s' \in S'} \\
    \overline{T} & \coloneqq T \amalg \Set{t_{s'}}{s' \in S'} \amalg \Set{t_s}{s \in S \setminus S'} \\
    \overline{W} & \coloneqq W \cup \Set{s \to t_s}{s \in S \setminus S'} \cup \Set{s' \to t_{s'} \leftarrow \overline{s'}}{s' \in S'} \\
    M_{\overline{N}} & \coloneqq (s \in S \mapsto M_N(s), \enspace \overline{s'} \mapsto M'(s'))
  \end{align*}
  Dann: $M'$ teilerreichbar in~$N$ $\iff$ Nullmark. erreichbar in~$\overline{N}$
\end{beweis}

% 4.3
\begin{satz}
  (E) ist entscheidbar.
\end{satz}

\TODO{Beweis lesen, zusammenfassen}

\begin{problem}[L -- \emph{Lebendigkeit}]
  Gegeben $N$. Frage: Ist $N$ lebendig?
\end{problem}

\begin{problem}[EL -- \emph{Einzellebendigkeit}]
  Gegeben seien $N$ und $t \in T$.
  Frage: Ist $t$ lebendig?
\end{problem}

% 4.4 a)
\begin{satz}
  (L) ist reduzierbar auf (EL)
\end{satz}

\begin{beweis}
  Konstruiere $\overline{N} = (\overline{S}, \overline{T}, \overline{W}, M_{\overline{N}})$ mit
  \begin{align*}
    \overline{S} & \coloneqq S \amalg \Set{s_t}{t \in T} \\
    \overline{T} & \coloneqq T \amalg \{ t_\mathrm{afterall} \} \\
    \overline{W} & \coloneqq W \cup \Set{t \to s_t}{t \in T} \cup \Set{s_t \to t_\mathrm{afterall}}{t \in T} \\
    M_{\overline{N}} & \coloneqq (s \in S \mapsto M_N(s), \enspace s_t \mapsto 0)
  \end{align*}
  Dann: $N$ lebendig $\iff$ $t_\mathrm{afterall}$ lebendig in $\overline{N}$.
\end{beweis}

% 4.4 b)
\begin{satz}
  (EL) ist reduzierbar auf (TE)
\end{satz}

\TODO{Beweis verstehen, zusammenfassen}

% 4.4 c)
\begin{satz}
  (0-E) $\reducesManyOneToLin$ Co-(L), das ist (L) mit umgekehrter Antwort
\end{satz}

\begin{beweis}
  Konstruiere $\overline{N} = (\overline{S}, \overline{T}, \overline{W}, M_{\overline{N}})$ mit
  \begin{align*}
    \overline{S} & \coloneqq S \amalg \{ s_\mathrm{distr}, s_\mathrm{control} \} \\
    \overline{T} & \coloneqq T \amalg \Set{t_s}{s \in S} \amalg \{ t_\mathrm{distr}, t_\mathrm{blackhole} \} \\
    \overline{W} & \coloneqq W \cup \Set{t \rightleftarrows s_\mathrm{control}}{t \in T} \cup \Set{t_\mathrm{distr} \to s \to t_s \to s_\mathrm{distr}}{s \in S} \\
    & \cup \{ s_\mathrm{distr} \rightleftarrows t_\mathrm{distr} \} \cup \{ s_\mathrm{control} \to t_\mathrm{blackhole} \} \\
    M_{\overline{N}} & \coloneqq (s \in S \mapsto M_N(s), \enspace s_\mathrm{distr} \mapsto 0, \enspace s_\mathrm{control} \mapsto 1)
  \end{align*}
  (Bemerke: Jede Markierung $\hat{M}$ mit $\hat{M}(s_\mathrm{distr}) > 0$ ist lebendig.) \\
  Dann: Nullmarkierung in $N$ erreichbar $\iff$ $\overline{N}$ nicht lebendig
\end{beweis}

% 4.5
\begin{satz}
  $\text{(L)} \reducesManyOneToLin \text{(EL)} \reducesManyOneToLin \text{(L)}$
\end{satz}

\TODO{Beweise nachvollziehen, aufschreiben}

\begin{fazit}
  (L) und (EL) sind entscheidbar, aber mindestens so schwer wie (E), (0-E) und (TE).
\end{fazit}

% §5. Beschränktheit und Überdeckbarkeit
\section{Beschränktheit und Überdeckbarkeit}

% 5.3
\begin{lem}[\emph{Dickson}]
  $\leq$ ist eine Wohlquasiordnung auf~$\N^n$, \dh{} für alle unendlichen Folgen $(M_i)_{i \in \N}$ in~$\N^n$ gibt es eine Teilfolge $(M_{i_j})_{j \in \N}$ mit $M_{i_j} \leq M_{i_{j+1}}$ für alle $j \in \N$.
\end{lem}

\begin{defn}
  Ein \emph{Weg} in einem Graphen~$(V, E)$ ist eine Folge $v_1 \ldots v_n$ in~$V$ mit $\fa{i \neq j} v_i \neq v_j$ und $(v_i, v_{i+1}) \in E$.
\end{defn}

\begin{defn}
  Ein Graph~$(V, E)$ heißt \emph{lokal endlich}, falls für alle $v \in V$ die Menge $\Set{w \in V}{(v, w) \in E}$ endlich ist.
\end{defn}

\begin{lem}[\emph{König}]
  Sei $(V, E)$ ein lokal endlicher gerichteter Graph und $v_0 \in V$ ein Knoten, sodass für alle $v \in V$ ein Weg von~$v_0$ nach~$v$ existiert. 
  Dann gibt es einen unendlichen Weg ausgehend von~$v_0$.
\end{lem}

% 5.2
\begin{satz}
  $
    N \text{ ist unbeschränkt} \enspace \iff
    \begin{array}[t]{l}
      \ex{M, M' \in \activeTransition{M_N}} \ex{w \in T^*} \\
      M \activeTransition{w} M' \wedge M \leq M' \wedge M \neq M'
    \end{array}
  $
\end{satz}

% 5.1
\begin{defn}
  Eine \emph{erweiterte Markierung} \textit{von~$N$} ist eine Abbildung
  \[
    M : S \to \N \cup \{ \omega \}.
  \]
\end{defn}

\begin{nota}
  $\ExtMarkings(S) \coloneqq \{ \text{ erw. Mark. von~$N$ } \} \coloneqq (\N \cup \{ \omega \})^S$
\end{nota}

% 5.7
\begin{defn}
  Sei $N$ ein Netz und $M_1$, $M_2$ erweiterte Markierungen.
  \begin{itemize}
    \item $M_2$ \emph{überdeckt} $M_1$ $\coloniff$ $M_1 \leq M_2$
    \item $M_1$ ist \emph{überdeckbar} $\coloniff$ $\ex{M \in \activeTransition{M_N}} M_1 \leq M$
  \end{itemize}
\end{defn}

\begin{defn}
  Eine Menge $S' \subseteq S$ heißt \emph{simultan unbeschränkt}, falls
  \[
    \fa{n \in \N} \ex{M \in \activeTransition{M_N}} \fa{s \in S} M(s) \geq n.
  \]
\end{defn}

\begin{defn}
  Sei $N = (S, T, W, M_N)$ ein Netz.
  Ein \emph{Überdeckungsgraph} von~$N$ ist ein kantenbeschrifteter, gericht. Graph $\Cov(N) = (V, E)$, der von folgendem (nichtdet.) Algorithmus berechnet wird:
  \begin{algorithmic}[1]
    \State{
      $V \coloneqq \emptyset \subset \ExtMarkings(S)$, \enspace
      $A \coloneqq \{ M_N \} \subset \ExtMarkings(S)$,
    }
    \State{
      $E \coloneqq \emptyset \subset \ExtMarkings(S) \times T \times \ExtMarkings(S)$,
    }
    \State{
      $\Call{pred}{} \coloneqq \mathrm{const} \, \nil \in (\ExtMarkings(S) \cup \{\nil\})^{\ExtMarkings(S)}$
    }
    \While{$A \neq \emptyset$}
      \State wähle $M \in A$
      \State $A \coloneqq A \setminus \{ M \}$, \enspace $V \coloneqq V \cup \{ M \}$
      \For{$t \in T$ mit $M \activeTransition{t}$}
        \State $M' \coloneqq M + \Delta t$, \enspace $M^* \coloneqq M$
        \While{$M^* \neq \nil \wedge M^* \not\leq M'$}
          $M^* \coloneqq \Call{pred}{M^*}$
        \EndWhile
        \If{$M^* \neq \nil$}
          $M' \coloneqq M' + \omega \cdot (M' - M^*)$ %= (s \mapsto \max (\{ M'(s) \} \cup \Set{\omega}{M'(s) > M^*(s)}) )
        \EndIf
        \State $E \coloneqq \{ (M, t, M') \}$
        \If{$M' \not\in V \cup A$}
          $A \coloneqq A \cup \{ M' \}$, \enspace
          $\Call{pred}{M'} \coloneqq M$
        \EndIf
      \EndFor
    \EndWhile
  \end{algorithmic}
\end{defn}

% 5.8
\begin{satz}
  $\Cov(N)$ ist endlich ($\!\iff\!$ der Algorithmus terminiert)
\end{satz}

\begin{kor}
  Es ist entscheidbar, ob~$N$ beschränkt ist.
\end{kor}

\begin{beweis}
  Konstruiere $\Cov(N) = (V, E)$ wobei $V \subset \ExtMarkings(S)$ endl. ist.
  Überprüfe, ob sogar $V \subset \Markings(S)$ gilt.
  Falls ja, so ist $\ReachabilityGraph(N) = \Cov(N)$ endlich.
  Falls nein, so gibt es $M$, $M'$ wie im letzten Satz und~$N$ ist somit unbeschränkt.
\end{beweis}

\begin{bem}
  Jedes $\Cov(N)$ ist (nach Einführen eines Fehlerzustandes und Kanten dorthin) ein determ. endl. Automat mit Startzustand~$M_N$.
\end{bem}

\begin{defn}
  $L(\Cov(N)) \subseteq T^*$ ist die Sprache der von einem $\Cov(N)$ akzeptierten Wörter. 
\end{defn}

\begin{nota}
  $M_w \coloneqq $ durch $w \in L(\Cov(N))$ erreichter Zust. in $\Cov(N)$
\end{nota}

% 5.10 a)
\begin{lem}
  $
    M_N \activeTransition{w} M \enspace \implies
    \begin{array}[t]{l}
      w \in L(\Cov(N)) \wedge \\
      \fa{s \in S} M_w(s) \in \{ M(s), \omega \}
    \end{array}
  $
  % ausgelassen: triviales Korollar
\end{lem}

% 5.11
\begin{lem}
  Für alle $M$ in $\Cov(N)$ u. alle $n \in \N$ gibt es ein $M' \!\in\! \activeTransition{M_N}$ mit
  \[
    \begin{cases}
      M'(s) = M(s) & \text{falls } M(s) \neq \omega, \\
      M'(s) > n & \text{falls } M(s) = \omega.
    \end{cases}
  \]
\end{lem}

% 5.12, 5.13, 5.14
\begin{kor}
  \begin{itemize}
    \item $S'$ ist simultan unbeschränkt $\iff$ $(\mathrm{const} \, \omega) \in \Cov(N)$
    \item Sei $\tilde{M}$ eine Markierung von~$N$. Dann gilt: $\tilde{M}$ ist überdeckbar in~$N$ $\iff$ $\tilde{M}$ wird von einem $M$ in~$\Cov(N)$ überdeckt
    \item $t$ ist nicht tot in~$N$ $\iff$ $t$ ist Kantenbeschriftung in $\Cov(N)$
  \end{itemize}
\end{kor}

\end{document}
