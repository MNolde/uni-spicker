\documentclass{cheat-sheet}

\pdfinfo{
  /Title (Zusammenfassung Term Rewriting and All That)
  /Author (Tim Baumann)
}

%\usepackage{nicefrac}
%\usepackage{tikz}
%\usetikzlibrary{calc}
%\usepackage[normalem]{ulem} % \sout
%\usepackage{multirow}

\newcommand{\from}{\leftarrow}
\newcommand{\reducesTo}{\xrightarrow{*}}
\newcommand{\strictlyReducesTo}{\xrightarrow{+}}
\newcommand{\reducesFrom}{\xleftarrow{*}}
\newcommand{\toOrEq}{\xrightarrow{=}}
\newcommand{\fromOrEq}{\xleftarrow{=}}
\newcommand{\joinable}{\downarrow}
\newcommand{\zzEq}{\xleftrightarrow{*}} % Zigzag-Equivalent
\newcommand{\NF}[1]{{#1}\!\downarrow\,\,} % Normalform
\newcommand{\Multisets}{\mathcal{M}} % Multimengen über einer Menge

% Kleinere Klammern
\delimiterfactor=701

\setlength{\tabcolsep}{2pt}

\begin{document}

\raggedcolumns % stretche Inhalt nicht über die gesamte Spaltenhöhe

\maketitle{Zusammenfassung Term Rewriting aAT}

Dies ist eine übersetzte Zusammenfassung des Buches Term Rewriting and All That von Franz Baader und Tobias Nipkow.

% §2. Abstrakte Reduktionssysteme
\section{Abstrakte Reduktionssysteme}

\begin{defn}
  Ein \emph{abstraktes Reduktionssystem} ist ein Tupel $(A, {\to})$, wobei ${\to} \in A \times A$ eine Relation auf $A$ ist.
\end{defn}

% §2.1. Äquivalenz und Reduktion

\begin{defn}
  \,
  \begin{minipage}[t]{0.88 \linewidth}
    \begin{tabular}[t]{ r l l }
      $\xrightarrow{0}$ & $\coloneqq \Set{(a, a)}{a \in A}$ & Identität \\
      $\xrightarrow{i+1}$ & $\coloneqq {\xrightarrow{i}} \circ {\to}$ & $(i+1)$-fache Komposition, $i \geq 0$ \\
      $\from$ & $\coloneqq \Set{(t, s)}{(s, t) \in {\to}}$ & Inverse Relation \\
      $\toOrEq$ & $\coloneqq (\to) \cup (\xrightarrow{0})$ & refl. Hülle \\
      $\reducesTo$ & $\coloneqq \cup_{i \geq 0} (\xrightarrow{i})$ & refl. trans. Hülle \\
      $\strictlyReducesTo$ & $\coloneqq \cup_{i \geq 1} (\xrightarrow{i})$ & refl. trans. Hülle \\
      $\xleftrightarrow{}$ & $\coloneqq {\to} \cup {\from}$ & symm. Hülle \\
      $\zzEq$ & $\coloneqq (\xleftrightarrow{})^{*}$ & refl. trans. symm. Hülle
    \end{tabular}
  \end{minipage}
\end{defn}

\begin{defn}
  Sei $x \in A$ ein Term.
  \begin{itemize}
    \item Der Term $x$ heißt \emph{reduzibel}, falls ein $y \in A$ mit $x \to y$ existiert,
    \item \emph{irreduzibel} (oder in \emph{Normalform}) falls $x$ nicht reduzibel ist.
    \item Ein Term $y \in A$ heißt \emph{Normalform} von~$x$, falls $x \reducesTo y$ und~$y$ irreduzibel ist.
    \item Eine Term~$y$ heißt \emph{direkter Nachfolger} von~$x$, falls $x \to y$.
    \item Eine Term~$y$ heißt \emph{Nachfolger} von~$x$, falls $x \strictlyReducesTo y$.
    \item $x$ und~$y$ heißen \textit{joinable}, notiert \emph{$x \joinable y$}, falls $\ex{z} x \reducesTo z \reducesFrom y$.
  \end{itemize}
\end{defn}

\begin{defn}
  Eine Reduktion ${\to}$ heißt
  \begin{tabular}{r l l}
    \emph{Church-Rosser} & $\coloniff$ & $x \zzEq y \implies x \joinable y$ \\
    \emph{konfluent} & $\coloniff$ & $y_1 \reducesFrom y \reducesTo y_2 \implies y_1 \joinable y_2$ \\
    \emph{semi-konfluent} & $\coloniff$ & $y_1 \from y \reducesTo y_2 \implies y_1 \joinable y_2$ \\
    \emph{terminierend} & $\coloniff$ &
      \begin{minipage}[t]{0.8 \linewidth}
        es gibt keine unendlich absteigende Kette \\
        $x_0 \to x_1 \to x_2 \to \ldots$
        \quad (auch: \textit{noethersch})
      \end{minipage} \\
    \emph{normalisierend} & $\coloniff$ & jeder Term besitzt eine Normalform \\
    \emph{konvergent} & $\coloniff$ & konfluent $\wedge$ normalisierend \\
  \end{tabular}
\end{defn}

\begin{lem}
  Für eine Reduktion ${\to}$ sind äquivalent:
  \begin{itemize}
    \item ${\to}$ ist Church-Rosser
    \item ${\to}$ ist konfluent
    \item ${\to}$ ist semi-konfluent
  \end{itemize}
\end{lem}

\begin{lem}
  Ist die Reduktion ${\to}$ konfluent/terminierend/konvergent, so besitzt jeder Term höchstens/mindestens/genau eine Normalform.
\end{lem}

\begin{nota}
  Falls~$x$ eine NF~$y$ besitzt, so schreibe $x \NF \coloneqq y$.
\end{nota}

\begin{thm}
  Ist ${\to}$ konvergent, so gilt $x \zzEq y \iff \NF{x} = \NF{y}$.
\end{thm}

\begin{bem}
  Dies liefert einen einfachen Algorithmus, um $x \zzEq y$ zu entscheiden: Reduziere die Terme~$x$ und~$y$ zu Normalformen und vergleiche diese.
\end{bem}

% §2.2. Wohlfundierte Induktion
\subsection{Terminierungsbeweise}

\begin{lem}
  ${\to}$ ist terminierend $\iff$ ${\to}$ ist eine Wohlordnung
\end{lem}

\begin{defn}
  Eine Relation ${\to}$ heißt
  \begin{itemize}
    \item \emph{endlich verzweigend}, falls jeder Term nur endlich viele direkte Nachfolger besitzt,
    \item \emph{global endlich}, falls jeder Term nur endl. viele Nachfolger hat,
    \item \emph{azyklisch}, falls kein Term~$a$ mit $a \strictlyReducesTo a$ existiert.
  \end{itemize}
\end{defn}

\begin{lem}
  \begin{itemize}
    \item Eine endlich verzweigende Relation ist global endlich, falls sie terminierend ist.
    \item Eine azykl. Relation ist terminierend, falls sie global endlich ist.
  \end{itemize}
\end{lem}

% §2.3. Terminierungsbeweise

\begin{lem}
  Sei $(A, {\to})$ ein Reduktionssystem und $(B, {>})$ eine wohlgeordnete Menge.
  Gibt es eine streng monotone Abbildung $\varphi : A \to B$, so ist~$A$ terminierend.
\end{lem}

\begin{lem}
  Ein endlich verzweigendes Reduktionssystem~$(A, {\to})$ ist genau dann terminierend, falls es eine streng monotone Abbildung $\varphi : (A, {\to}) \to (\N, >)$ gibt.
\end{lem}

% §2.4. Lexikographische Ordnungen

\begin{defn}
  Seien $(A_i, >_i)_{i = 1, \ldots, n}$ geordnete Mengen.
  Die \emph{lexikalische Ordnung}~$>_{\text{lex}}$ auf $A_1 \times \ldots \times A_n$ ist definiert durch
  \[
    (x_1, \nldots, x_n) >_{\text{lex}} (y_1, \nldots, y_n) \!\!\coloniff\! \ex{k {\leq} n\!}\! (\fa{i {<} k\!}\! x_i \!=\! y_i) \,\wedge\, x_k <_k y_k.
  \]
\end{defn}

\begin{lem}
  Ist $>$ eine strikte (Wohl-) Ordnung, so auch $>_{\textit{lex}}$.
\end{lem}

% ausgelassen: $>_{lex}^*$, $>_{Lex}$

% §2.5. Multimengen-Ordnungen

\begin{defn}
  Eine \textit{Multimenge}~$M$ über einer Menge~$A$ ist eine Abbildung $M : A \to \N$.
  Sie ist endlich, falls ${\sum}_{a \in A} M(a) < \infty$.
\end{defn}

\begin{nota}
  \begin{minipage}[t]{0.8 \linewidth}
    $\Multisets(A) \coloneqq \{ \text{ Multimengen über $A$ } \}$ \\
    $a \in M \coloniff M(a) \geq 1$
  \end{minipage}
\end{nota}

\begin{defn}
  Die \textit{Differenz} von Multimengen~$M, N \in \Multisets(A)$ ist $M - N \in \Multisets(A)$ mit
  $(M - N)(a) \coloneqq \max \{ 0, M(a) - N(a) \}$.
\end{defn}

\begin{defn}
  Sei ${>}$ eine strikte Ordung auf~$A$.
  Die \emph{Multimengenordnung} $>_{\text{mul}}$ auf $\Multisets(A)$ ist dann definiert durch
  \[
    M >_{\text{mul}} N \coloniff M \neq N \wedge \fa{n \in N - M} \ex{m \in M - N} m > n.
  \]
\end{defn}

\begin{lem}
  Ist $>$ eine strikte (Wohl-) Ordnung, so auch $>_{\textit{mul}}$.
\end{lem}

% §2.6. Ordnungen in ML (ausgelassen)

% §2.7. Konfluenzbeweise
\subsection{Konfluenzbeweise}

% TODO: ist die Diamant-Eigenschaft relevant?
\begin{defn}
  Eine Relation ${\to}$
  \begin{itemize}
    \item heißt \emph{lokal konfluent}, falls $y_1 \!\from\! y \to y_2 \implies y_1 \joinable y_2$.
    \item heißt \emph{stark konfluent}, falls $y_1 \!\from\! y \!\to\! y_2 \implies \ex{z\!}\! y_1 \reducesTo z \fromOrEq y_2$.
    \item besitzt die \emph{Diamant-Eigenschaft}, falls
    \[
      y_1 \!\from\! y \!\to\! y_2 \implies \ex{z} y_1 \to z \from y_2.
    \]
  \end{itemize}
\end{defn}

\begin{lem}
  Falls ${\to_1} \leq {\to_2} \leq {\xrightarrow{*}_1}$, so gilt ${\xrightarrow{*}_1} = {\xrightarrow{*}_2}$.
  Ist zusätzlich~$\to_2$ (stark) konfluent, so auch ${\to_1}$.
\end{lem}

\begin{lem}
  \begin{itemize}
    \item Stark konfluente Relationen sind konfluent.
    \item Eine terminierende Rel. ist konfluent, falls sie lokal konfluent ist.
  \end{itemize}
\end{lem}

% §2.7.1. Kommutation

% TODO: sind die beiden letzen Definitionen relevant?
\begin{defn}
  Zwei Relationen~${\to_1}$ und~${\to_2}$ auf~$A$
  \begin{itemize}
    \item \emph{kommutieren}, falls $y_1 \reducesFrom_1 x \reducesTo_2 y_2 \implies \ex{z} y_1 \reducesTo_2 z \reducesFrom_1 y_2$.
    \item \emph{kommutieren stark}, falls
    \[
      y_1 \from_1 x \to_2 y_2 \implies \ex{z} y_1 \toOrEq_2 z \reducesFrom_1 y_2.
    \]
    \item besitzen die \emph{Kommutierender-Diamant-Eigenschaft}, falls
    \[
      y_1 \from_1 x \to_2 y_2 \implies \ex{z} y_1 \to_2 z \from_1 y_2.
    \]
  \end{itemize}
\end{defn}

\begin{lem}
  Angenommen, $\to_1$ und~$\to_2$ sind konfluent und kommutieren. \\
  Dann ist auch ${\to_1} \cup {\to_2}$ konfluent.
\end{lem}

\end{document}
