\documentclass{cheat-sheet}

\pdfinfo{
  /Title (Zusammenfassung Modellkategorien)
  /Author (Tim Baumann)
}

\usepackage{tikz}
\usetikzlibrary{arrows,cd}
\usepackage{cancel} % für durchgestrichenen Text

\usepackage{mathabx} % für \boxslash
\usepackage{stmaryrd} % für \leftarrowtriangle
\usepackage{bbm} % Für 1 mit Doppelstrich (Einheitsobjekt in monoidaler Kategorie)
\usepackage{dsfont} % \mathds

% http://tex.stackexchange.com/questions/117732/tikz-and-babel-error
% Es ist schierer Wahnsinn, welche Hacks LaTeX benötigt!
\tikzset{
  every picture/.prefix style={
    execute at begin picture=\shorthandoff{"}
  }
}

% Kategorientheorie-Makros

% Konzepte
\DeclareMathOperator{\Ob}{Ob} % Objekte (einer Kategorie)
\DeclareMathOperator{\Mor}{Mor} % Morphismenmenge / -klasse
\DeclareMathOperator{\End}{End} % Endomorphismenmenge / -klasse
\DeclareMathOperator{\Hom}{Hom} % Homomorphisms
\DeclareMathOperator{\Nat}{Nat} % Natürliche Transformationen
\DeclareMathOperator{\dom}{dom} % Domain
\DeclareMathOperator{\codom}{codom} % Codomain
\newcommand{\op}{\mathrm{op}} % opposite category
\DeclareMathOperator{\Aut}{Aut} % Automorphismengruppe
\newcommand{\ladj}{\dashv} % Links-adjungiert (left-adjoint)
\newcommand{\Lim}{\lim} % Limes
\DeclareMathOperator{\colim}{colim} % Kolimes
\newcommand{\Colim}{\colim} % Kolimes
%\newcommand{\myint}[2]{{\textstyle \int\limits_{#1}^{#2}}}
\newcommand{\EndC}[2]{\myint{#1}{} #2} % Ende
\newcommand{\CoEndC}[2]{\myint{}{#1} #2} % Koende
\DeclareMathOperator{\Ran}{Ran} % Rechts-Kan-Erweiterung
\DeclareMathOperator{\Lan}{Lan} % Links-Kan-Erweiterung
\newcommand{\IHom}{\underline{\Hom}} % Inner Homomorphisms

% Platzhalter
\newcommand{\DiaTodo}{\fcolorbox{red}{white}{TODO: Diagramm einfügen!}}

% Populäre Kategorien
\newcommand{\SetC}{\mathbf{Set}} % Kategorie der Mengen
\newcommand{\FinSetC}{\mathbf{FinSet}} % Kategorie der endlichen Mengen
\newcommand{\sSet}{\mathbf{sSet}} % Kategorie der simplizialen Mengen
\newcommand{\Top}{\mathbf{Top}} % Kategorie der topologischen Räume
\newcommand{\AbGrp}{\mathbf{Ab}} % Kategorie der abelschen Gruppen
\newcommand{\Grp}{\mathbf{Grp}} % Kategorie der Gruppen
\newcommand{\Ouv}{\mathbf{Ouv}} % Kategorie der offenene Mengen eines topol. Raumes
\newcommand{\KHaus}{\mathbf{KHaus}} % Kategorie der kompakten Hausdorffräume
\newcommand{\CatC}{\mathbf{Cat}} % Kategorie der kleinen Kategorien
\newcommand{\Vect}{\mathbf{Vect}} % Kategorie der Vektorräume über einem Körper
\newcommand{\Field}{\mathbf{Field}} % Kategorie der Körper
\newcommand{\Ring}{\mathbf{Ring}} % Kategorie der Ringe
\newcommand{\Alg}{\mathbf{Alg}} % Kategorie der Algebren über einem Körper/Ring
\newcommand{\VectFin}{\mathbf{Vect}_{\mathrm{fin}}} % Kategorie der endlichen Vektorräume über einem Körper
\newcommand{\kVect}{\text{$k$-$\Vect$}} % Kategorie der k-Vektorräume über einem Körper k
\newcommand{\kVectFin}{\text{$k$-$\VectFin$}} % Kategorie der endlichen k-Vektorräume über einem Körper k
\newcommand{\Mod}{\mathbf{Mod}} % Kategorie der Moduln über einem Ring
\newcommand{\LMod}[1]{{#1}\text{-}\Mod} % Kategorie der (#1)-Linksmoduln
\newcommand{\RMod}[1]{\Mod\text{-}{#1}} % Kategorie der (#1)-Rechtsmoduln
\newcommand{\BMod}[2]{{#1}\text{-}\Mod\text{-}{#2}} % Kategorie der (#1)-(#2)-Bimoduln
\newcommand{\Kom}{\mathbf{Kom}} % Kategorie der Komplexe in einer abelschen Kategorie
\newcommand{\Der}{\mathcal{D}} % abgeleitete Kategorie einer abelschen Kategorie
\newcommand{\kAlg}{k\text{-}\Alg} % Kategorie der k-Algebren

% Bezeichnungen für Variablen, die für Kategorien stehen
\newcommand{\Aat}{\mathcal{A}} % Category-A
\newcommand{\Bat}{\mathcal{B}} % Category-B
\newcommand{\Cat}{\mathcal{C}} % Category-C
\newcommand{\Dat}{\mathcal{D}} % Category-D
\newcommand{\Eat}{\mathcal{E}} % Category-E
\newcommand{\Fat}{\mathcal{F}} % Category-F
\newcommand{\Gat}{\mathcal{G}} % Category-G
\newcommand{\HatC}{\mathcal{H}} % Category-H
\newcommand{\Iat}{\mathcal{I}} % Category-I (Indexkategorie)
\newcommand{\Jat}{\mathcal{J}} % Category-J (Indexkategorie)
\newcommand{\Lat}{\mathcal{L}} % Category-L
\newcommand{\MatC}{\mathcal{M}} % Category-M
\newcommand{\NatC}{\mathcal{N}} % Category-N
\newcommand{\Sit}{\mathcal{S}} % Situs-S
\newcommand{\Sat}{\mathcal{S}} % Category-S
\newcommand{\Wat}{\mathcal{W}} % Category-W (whaaaatt?)
\newcommand{\Xat}{\mathcal{X}} % Category-X
\newcommand{\Yat}{\mathcal{Y}} % Category-Y

% Makros für Adjunktionen. Geklaut von
% http://sma.epfl.ch/~werndli/latex/adjunction.pdf
% http://sma.epfl.ch/~werndli/latex/adjunction.tex
\usepackage[all]{xy}
\newcommand{\adj}[1][]{\def\ArgI{#1}\adjRelayI}
\newcommand{\adjRelayI}[1][]{\def\ArgII{#1}\adjRelayII}
\newcommand{\adjRelayII}[3][2.2em]{
  \ensuremath{\SelectTips{lu}{10}\xymatrix@C=#1@1{
    {#2\,}
    \ar@<1ex>[r]^-{\ArgI}^-{}="1" &
    {\,#3}
    \ar@<1ex>[l]^-{\ArgII}^-{}="2"
    \ar@{}"1";"2"|(.3){\hbox{}}="7"
    \ar@{}"1";"2"|(.7){\hbox{}}="8"
    \ar@{|-} "8" ;"7"
  }}
}
\newcommand{\radj}[1][]{\def\ArgI{#1}\radjRelayI}
\newcommand{\radjRelayI}[1][]{\def\ArgII{#1}\radjRelayII}
\newcommand{\radjRelayII}[3][2.2em]{
  \ensuremath{\SelectTips{lu}{10}\xymatrix@C=#1@1{
  {#2\,}
  \ar@<-1ex>[r]_-{\ArgI}^-{}="1" &
  {\,#3}
  \ar@<-1ex>[l]_-{\ArgII}^-{}="2"
  \ar@{}"1";"2"|(.3){\hbox{}}="7"
  \ar@{}"1";"2"|(.7){\hbox{}}="8"
  \ar@{|-} "7" ;"8"
  }}
}
 % Kategorientheorie-Makros
\input{include/sheaves} % Garbentheorie-Makros (für Beispiele)

\newcommand{\nspace}[1]{\foreach \i in {1,...,#1}{ \! }} % Negativer Abstand
\newcommand{\Ord}{\mathcal{O}_n} % Menge der Ordinalzahlen
\newcommand{\Prop}{\mathbf{Prop}} % Typ der Propositionen
\newcommand{\xtwoheadrightarrow}[1]{\xrightarrow{#1}\nspace{8}\to\,\,} % Pfeil mit zwei Spitzen (HACK!!!)
\DeclareMathOperator{\Quot}{Quot} % Quotientenkörper
\newcommand{\gut}{\text{gut}}
\newcommand{\sg}{\text{sehr gut}}
\newcommand{\Cyl}[1]{#1 \!\times\! I} % Zylinderobjekt
\newcommand{\PO}[1]{{#1}^I} % Pfadobjekt
\newcommand{\gutCyl}[1]{#1 \!\times_\gut\! I} % gutes Zylinderobjekt
\newcommand{\lhhe}{\boxslash} % Linkshochhebungseigenschaftsdiagramm-im-Miniaturformat-Symbol
\newcommand{\Weak}{\mathcal{W}} % weak equivalences
\newcommand{\Cof}{\mathcal{C}} % cofibrations
\newcommand{\Fib}{\mathcal{F}} % fibrations
\newcommand{\ModC}{\mathcal{M}} % model category
\newcommand{\NodC}{\mathcal{N}} % another model category
\DeclareMathOperator{\Ho}{Ho} % Homotopiekategorie
\DeclareMathOperator{\Cell}{Cell} % Zellkomplexe
\DeclareMathOperator{\Coff}{Cof} % Kofaserungen
\newcommand{\ModStr}{$(\Weak, \Cof, \Fib)$} % Daten einer Modellstruktur
\newcommand{\LL}{\mathbb{L}} % linksabgeleiteter
\newcommand{\RR}{\mathbb{R}} % rechtsabgeleiteter
\newcommand{\LD}[1]{\mathbb{L} #1} % linksabgeleiteter Funktor
\newcommand{\RD}[1]{\mathbb{R} #1} % linksabgeleiteter Funktor
\newcommand{\LonelyHeart}{\{ \, \heartsuit \, \}} % Menge mit genau einem Element
\newcommand{\UnitOb}{\mathbbm{1}} % Einheitsobjekt in einer monoidalen Kategorie
\DeclareMathOperator{\Ch}{Ch} % Kategorie der Kettenkomplexe
\newcommand{\Simpl}{\mathcal{S}} % Homotopiekategorie von sSet
\DeclareMathOperator{\Map}{Map} % Objekt von Morphismen
\DeclareMathOperator{\fib}{fib} % Faser
\DeclareMathOperator{\cofib}{cofib} % Kofaser
\newcommand{\Sph}{\mathds{S}} % Sphäre
\DeclareMathOperator{\sk}{sk} % Skelett

\newenvironment{centertikzcd}
  {\begin{center}\begin{tikzcd}}
  {\end{tikzcd}\end{center}}

% Abkürzungen
\newcommand{\kD}{k.\,D.} % kommutatives Diagramm

% Kleinere Klammern
\delimiterfactor=701

\begin{document}

\maketitle{Zusammenfassung Modellkategorien}

\section{Kategorientheorie}

\begin{bem}
  Die \spickerref{topo}{Topologie-Zusammenfassung} bietet eine Übersicht über Grundbegriffe der Kategorientheorie. Weiterführende Begriffe werden in der \spickerref{homoalg}{Homologische-Algebra-Zusammenfassung} behandelt.
\end{bem}

% Vorlesung vom 22.4.2015

\begin{defn}
  Eine \emph{(schwache) 2-Kategorie} $\C$ besteht aus
  \begin{itemize}
    \item einer Ansammlung $\Ob(\C)$ von Objekten,
    \item für jedes Paar $(\Cat, \Dat)$ von Objekten einer Kategorie
    \[
      \Hom_{\C}(\Cat, \Dat) = \left\{\,
      \begin{tikzcd}
        \Cat \arrow[r, bend left=50, "F"{above}, ""{name=U, below}]
        \arrow[r, bend right=50, "G"{below}, ""{name=D, above}]
        & \Dat
        \arrow[Rightarrow, from=U, to=D]
      \end{tikzcd}
      \,\right\},
    \]
    \item für jedes Tripel $(\Cat, \Dat, \Eat)$ von Objekten einem Funktor
    \[
      \Hom_\C(\Cat, \Dat) \times \Hom_\C(\Dat, \Eat) \to \Hom_\C(\Cat, \Eat), \enspace
      (F, G) \mapsto G \circ F,
    \]
    \item für jedes Objekt $\Cat \in \Ob(\C)$ einem Objekt $\Id_\Cat \in \Hom_\C(\Cat, \Cat)$,
    \item für alle $\Cat, \Dat, \Eat, \Fat \in \Ob(\C)$ einem natürlichen Isomorphismus
    \[ \alpha_{\Cat, \Dat, \Eat, \Fat} : \enspace \blank \circ (\blank \circ \blank) \Longrightarrow (\blank \circ \blank) \circ \blank, \]
    wobei beide Seiten Funktoren sind vom Typ
    \[ \Hom(\Eat, \Fat) \times \Hom(\Dat, \Eat) \times \Hom(\Cat, \Dat) \to \Hom(\Cat, \Fat), \]
    \iffalse
    \begin{centertikzcd}[column sep=1cm, row sep=0.5cm]
      & \Hom(\Cat, \Eat) \times \Hom(\Eat, \Fat) \arrow[rd] \\
      \Hom(\Cat, \Dat) \times \Hom(\Dat, \Eat) \times \Hom(\Eat, \Fat) \arrow[ru] \arrow[rd] & \Hom(\Cat, \Fat) \\
      & \Hom(\Cat, \Dat) \times \Hom(\Dat, \Fat)
    \end{centertikzcd}
    \fi
    \item und für alle $\Cat, \Dat \in \Ob(\C)$ natürlichen Isomorphismen
    \[
      \lambda_{\Cat, \Dat} : (\Id_\Dat \circ \, \blank) \Rightarrow \Id_{\Hom(\Cat, \Dat)}, \enspace
      \rho_{\Cat, \Dat} : (\blank \circ \Id_\Cat) \Rightarrow \Id_{\Hom(\Cat, \Dat)},
    \]
  \end{itemize}
  sodass folgende \emph{Kohärenzbedingungen} erfüllt sind:
  \begin{itemize}
    \item Für alle $(\Cat \xrightarrow{F} \Dat \xrightarrow{G} \Eat \xrightarrow{H} \Fat \xrightarrow{K} \Gat) \in \Cat$ kommutiert
    \begin{centertikzcd}[column sep=1cm, row sep=0.5cm]
      K (H (G F)) \arrow[r, "\alpha_{\Cat, \Eat, \Fat, \Gat}"] \arrow[d, "K \alpha_{\Cat, \Dat, \Eat, \Fat}"] &
      (K H) (G F) \arrow[r, "\alpha_{\Cat, \Dat, \Eat, \Gat}"] &
      ((K H) G) H \\
      K ((H G) F) \arrow[rr, "\alpha_{\Cat, \Dat, \Fat, \Gat}"] &&
      (K (H G)) F \arrow[u, "\alpha_{\Dat,\Eat,\Fat,\Gat} F"]
    \end{centertikzcd}
    \item Für alle $(\Cat \xrightarrow{F} \Dat \xrightarrow{G} \Eat) \in \C$ kommutiert
    \begin{centertikzcd}[column sep=1cm, row sep=0.5cm]
      G \circ (\Id_\Dat \circ F) \arrow[rr, "\alpha_{\Cat,\Dat,\Dat,\Eat}"] \arrow[rd, "G \lambda_{\Cat, \Dat}"] &&
      (G \circ \Id_\Dat) \circ F \arrow[ld, "\rho_{\Dat, \Eat} F"] \\
      & G \circ F
    \end{centertikzcd}
  \end{itemize}
\end{defn}

% TODO: Bemerkung: Es gibt drei assoziierte "duale" Kategorien

\begin{bspe}
  \begin{itemize}
    \item Die Kategorie $\CatC$ der Kategorien ist eine 2-Kategorie.
    \item Jede Kategorie $\Cat$ ist natürlich eine 2-Kategorie.
    \item Die Kategorie der Ringe $\R$ mit $\Ob(\R) \coloneqq \{ \, \text{Ringe mit Eins} \, \}$ und $\Hom_\R(A, B) \coloneqq \text{Kat. der $B$-$A$-Bimoduln}$ mit $N \circ M \coloneqq N \otimes_B M$ für $M \in \Hom(A, B)$ und $N \in \Hom(B, C)$. Dabei ist $\Id_A \coloneqq A$. %als $A$-$A$-Bimodul.
  \end{itemize}
\end{bspe}

\begin{defn}
  Eine \emph{monoidale Kategorie} ist eine 2-Kategorie mit genau einem Objekt.
  In der Regel wird dann $\otimes$ anstelle von $\circ$ geschrieben.
\end{defn}

% §2. Universelle Eigenschaften

% Motto: Interessante Objekte einer mathematischen Theorie werden durch universelle Eigenschaften definiert

% Ausgelassen: Definition "darstellen"

% Ausgelassen: Limiten, terminales/initiales Objekt

% Ausgelassen: Produkt, Koprodukt, Differenzkern, Kodifferenzkern

\begin{defn}
  Sei $S : \Cat^\op \times \Cat \to \Aat$ ein Funktor. Ein \emph{Ende} $E \in \Ob(\Aat)$ von $S$ ist eine Familie $\alpha_c : E \to S(c, c)$, $c \in \Ob(\Cat)$ von Morphismen in $\Aat$, sodass für alle $(f : c \to c') \in \Cat$ das Diagramm
  \begin{centertikzcd}[column sep=1.4cm, row sep=0.05cm]
    & S(c, c) \arrow[rd, "{S(\id_c, f)}"] \\
    E \arrow[ru, "\alpha_c"] \arrow[rd, "\alpha_{c'}"{below}] && S(c, c') \\
    & S(c', c') \arrow[ru, "{S(f, \id_{c'})}"{below}]
  \end{centertikzcd}
  kommutiert, und $E$ universell (terminal) mit dieser Eigenschaft ist. \\
  Sprechweise: Ein Ende ist ein terminaler $S$-\emph{Keil}.
\end{defn}

\begin{nota}
  $E = \EndC{c}{S(c,c)}$.
\end{nota}

\begin{bem}
  Enden sind spezielle Limiten, und umgekehrt sind Limiten spezielle Enden: $\lim F = \EndC{c}{F(c)}$; der Integrand ist $\Cat^\op \times \Cat \to \Cat \stackrel{F}{\to} \Aat$.
\end{bem}

\begin{bem}
  Das duale Konzept ist das eines \cancel{Anfangs} Koendes $\CoEndC{c} S(c, c)$.
\end{bem}

\begin{bsp}
  Seien $F, G : \Cat \to \Aat$ zwei Funktoren. Dann ist
  %$\Hom_\Aat(F(\blank), G(\blank)) : \Cat^\op \times \Cat \to \SetC$ ein Funktor mit Ende
  \[ \EndC{c}{\Hom_\Aat(F(c), G(c))} \enspace\cong\enspace \Nat(F, G). \]
\end{bsp}

\begin{satz}[Fubini]
  Sei $S : \Dat^\op \!\times\! \Dat \!\times\! \Cat^\op \!\times\! \Cat \to \Aat$ ein Funktor. Dann gilt
  \[ \EndC{(d,c)}{S(d,d,c,c)} \enspace\cong\enspace \EndC{d}{\EndC{c}{S(d,d,c,c)}}, \]
  falls die rechte Seite und $\EndC{c}{S(d,d',c,c)}$ für alle $d, d' \in \Dat$ existieren.
\end{satz}

\begin{bsp}
  Sei $R$ ein Ring, aufgefasst als präadditive Kategorie mit einem Objekt $*$.
  Ein additiver Funktor $R^{(\op)} \to \AbGrp$ ist nichts anderes als ein $R$-Linksmodul (bzw. $R$-Rechtsmodul). Dann ist
  \[ A \otimes_R B \cong \CoEndC{* \in R}{A \otimes_{\Z} B}. \]
\end{bsp}

% Übungsblatt 2, Aufgabe 4
\begin{lem}[Ninja-Yoneda-Lemma]
  Für jede Prägarbe $F : \Cat^\op \to \SetC$ gilt
  \[ F \cong \CoEndC{c}{F(c) \times \Hom_\Cat(\blank, c)}. \]
\end{lem}

\begin{defn}
  Sei $\C$ eine 2-Kategorie. Seien $\Cat, \Dat \in \C$. Eine \emph{Adjunktion} von $F \in \Hom_\C(\Cat, \Dat)$ und $G \in \Hom_\C(\Dat, \Cat)$ ist geg. durch Morphismen $\eta : \Id_\Cat \Rightarrow G \circ F$ (genannt \emph{Eins}) und $\epsilon : F \circ G \Rightarrow \Id_\Dat$ (\emph{Koeins}) mit $G \epsilon \circ \eta G = \Id_G$ und $\epsilon F \circ F \eta = \Id_F$.
  Man notiert $F \ladj G$.
\end{defn}

\begin{lem}
  R/L-Adjungierte sind eindeutig bis auf eindeutige Isomorphie.
\end{lem}

\begin{bem}
  Seien $F : \Cat \to \Dat$ und $G : \Dat \to \Cat$ Funktoren. Dann gilt $F \ladj G$ genau dann, wenn es einen nat. Iso zwischen den Hom-Mengen gibt:
  \[ \Hom(F \circ \blank, \blank) \cong \Hom(\blank, G \circ \blank) \]
\end{bem}

\begin{bsp}
  $\exists \,_f \ladj f^* \ladj \forall \,_f$
\end{bsp}

\begin{bsp}
  Betrachte die 2-Kat. der Ringe. Dann gilt: Ein $B$-$A$-Modul $M$ ist genau dann ein Linksadjungierter, wenn $M$ als Rechts-$A$-Modul endlich erzeugt und projektiv ist.
\end{bsp}

\begin{bem}
  Sind $\eta$ und $\epsilon$ in $F \ladj G$ sogar Isomorphismen, so heißt $F \ladj G$ auch \emph{adjungierte Äquivalenz}. Jede beliebige Äquivalenz lässt sich stets (unter Beibehaltung von $F$ und $G$ sowie einem der Morphismen $\epsilon$, $\eta$) zu einer adj. Äquivalenz verfeinern.
\end{bem}

\vfill
\columnbreak

\subsection{Kan-Erweiterungen}

\begin{defn}
  Sei $\Aat \xleftarrow{T} \ModC \xrightarrow{K} \Cat$ ein Ausschnitt einer 2-Kategorie. Eine \emph{Rechts-Kan-Erw.} (RKE) $(R, \epsilon)$ von $T$ längs $K$ besteht aus
  \begin{itemize}
    \miniitem{0.45 \linewidth}{einem Morph. $R : \Cat \to \Aat$}
    \miniitem{0.48 \linewidth}{einem 2-Morph. $\epsilon : R \circ K \Rightarrow T$,}
  \end{itemize}
  sodass gilt:
  % $(R, \epsilon)$ unter diesen Daten terminal ist:
  Für alle Möchtegern-RKE $(S : \Cat \to \Aat, \eta : S \circ K \Rightarrow T)$ gibt es genau ein $\sigma : S \Rightarrow R$ mit $\epsilon \circ \sigma K = \eta$.
  Notation: $R = \Ran_K(T)$
\end{defn}

\begin{bem}
  Es sind äquivalent: \enspace
  \inlineitem{$(R, \epsilon)$ ist RKE von $T$ längs $K$} \\
  \inlineitem{$\eta \mapsto \epsilon \circ \eta K : \Nat(S, R) \to \Nat(S \circ K, T)$ ist bij. $\forall \, S : \Cat \to \Aat$}
\end{bem}

% Bemerkung vom Guide zum dritten Übungsblatt
\begin{bem}
  Es gilt $R = \Ran_K(T)$ genau dann, wenn es in $S \in [\Cat, \Aat]$ natürliche Isomorphismen
  $\Nat(S, R) \cong \Nat(S \circ K, T)$ gibt.
\end{bem}

\begin{prop}
  RKEs sind eindeutig bis auf eindeutige Isomorphie.
\end{prop}

\begin{bspe}
  \begin{itemize}
    \item Die RKE eines bel. Morphismus $T : \ModC \to \Aat$ längs $\Id_\ModC$ existiert stets und ist gegeben durch $(T, T \circ \Id_M \Rightarrow T)$.
    \item In der 2-Kategorie der Ringe existieren alle RKE:
    \[ \Ran_K(T) = (\Hom_M(K, T), \enspace ev : \Hom_\ModC(K, T) \otimes_C K \Rightarrow T). \]
  \end{itemize}
\end{bspe}

\begin{bsp}
  Sei $K : \ModC \to \UnitOb, \enspace * \mapsto 1$ und $T : \ModC \to \Aat$ irgendein Funktor. Dann ist eine RKE von $T$ längs $K$ dasselbe wie ein Limes von $T$.
\end{bsp}

\begin{thm}
  Seien $K : \ModC \to \Cat$ und $T : \ModC \to \Aat$ Funktoren.
  Existiere für alle $c \in \Ob(\Cat)$ der Limes
  $R(c) \coloneqq \Lim ((f : c \to Km) \mapsto Tm)$. \\
  Dabei ist die Indexkategorie des Limes die Kommakat. $\Delta(c) \downarrow K$. \\
  Dann lässt sich diese Setzung zu einem Funktor $\Cat \to \Aat$ ausdehnen und zwar zu einer RKE von $T$ längs $K$.
\end{thm}

\begin{bem}
  Ist $\ModC$ klein und $\Cat$ lokal klein und ist $\Aat$ vollständig, so sind die Voraussetzungen des Theorems für jeden Funktor $K : \ModC \to \Cat$, $T : \ModC \to \Aat$ erfüllt. Insbesondere ist dann jede solche RKE von der Form im Theorem. Solche RKE heißen auch \emph{punktweise RKE}.
\end{bem}

\begin{lem}
  Eine RKE ist genau dann punktweise, wenn sie für alle $a \in \Ob(\Aat)$ unter dem Funktor $\Hom_\Aat(a, \blank)$ erhalten bleibt.
\end{lem}

\begin{thm}
  Sei $K \!:\! M \to C$ ein Funktor. Betrachte $K^* \!:\! [\Cat, \Aat] \to [\ModC, \Aat]$.
  \begin{itemize}
    \item Wenn ein Funktor $\Ran_K : [\ModC,\Aat] \to [\Cat, \Aat]$ mit $K^* \ladj \Ran_K$ ex., so ist $\Ran_K(T)$ für alle $T : \ModC \!\to\! \Aat$ eine RKE von $T$ längs $K$.
    \item Existiere für alle $T : \ModC \to \Aat$ eine RKE $\Ran_K(T)$. Dann kann man die Zuordnung $T \mapsto \Ran_K(T)$ zu einem Rechtsadjungierten von $K^*$ ausdehnen.
  \end{itemize}
\end{thm}

\begin{thm}
  Sei $G : \Aat \to \Xat$ in einer 2-Kategorie. Dann sind äquivalent:
  \begin{itemize}
    \item $G$ besitzt einen Linksadjungierten.
    \item $\Ran_G(\Id_\Aat)$ existiert und $G \circ \Ran_G(\Id_\Aat) = \Ran_G(G \circ \Id_\Aat)$.
  \end{itemize}
  In diesem Fall gilt $\Ran_G(\Id_\Aat) \ladj G$ und $\Ran_G(\Id_\Aat)$ wird sogar von allen Morphismen $H : \Aat \to \Yat$ bewahrt.
\end{thm}

\begin{thm}
  Rechtsadjungierte bewahren RKE.
\end{thm}

\begin{kor}
  Rechtsadjungierte bewahren Limiten (RAPL)
\end{kor}

\iffalse
\begin{bsp}
  Sei $G : \Field \to \Ring$ der Vergissfunktor. Dann ist
  \[ (\Ran_G(G))(R) = \prod_{\mathfrak{p} \subset R} \Quot(R/{\mathfrak{p}}). \]
\end{bsp}
\fi

% §. Algebraische Strukturen in Kategorien
\subsection{Algebraische Strukturen in Kategorien}

\begin{defn}
  Ein \emph{Retrakt} ist ein Morphismus $r : Y \to X$, sodass ein Morphismus $i : X \to Y$ mit $r \circ i = \id_X$ existiert. \\
  Sprechweise: $X$ ist ein Retrakt von $Y$ (vermöge $i$).
\end{defn}

% Ausgelassen: $S^1 \hookrightarrow \R^2$ besitzt kein Linksinverses, $S^1 \hookrightarrow \R^2 \setminus \{ 0 \}$ schon.

\begin{bsp}
  Ein Modul $U$ ist genau dann Retrakt von einem Modul $M$, wenn $U$ ein direkter Summand von $M$ ist.
\end{bsp}

\begin{prop}
  "`$\blank$ ist Retrakt von $\blank$\,"' ist eine reflexive und trans. Relation.
\end{prop}

\begin{defn}
  Ein \emph{Retrakt eines Morphismus} $(X \xrightarrow{g} Y) \!\in\! \Cat$ ist ein Mor. $f : A \to B$, sodass es ein komm. Diagramm folgender Form gibt:
  \vspace{-8pt}
  \begin{centertikzcd}
    A \arrow[r, "i"] \arrow[d, "f"] \arrow[rr, bend left, swap, "id_A"] &
    X \arrow[r, "r"] \arrow[d, "g"] &
    A \arrow[d, "f"] \\
    B \arrow[r, "j"] \arrow[rr, bend right, "id_B"] &
    Y \arrow[r, "s"] &
    B
  \end{centertikzcd}
  \vspace{-8pt}
\end{defn}

\begin{bem}
  Ein Retrakt von $g \in \Mor(\Cat)$ ist ein Retrakt von $g \in \Ob(\Cat^{\to})$.
\end{bem}

\begin{prop}
  \begin{itemize}
    \item Retrakte von Isomorphismen sind Isomorphismen.
    \item Sei $f \circ g = \id$. Dann ist $f$ ein Retrakt von $g \circ f$.
  \end{itemize}
\end{prop}

% Ausgelassen: Retrakte sind abgeschlossen unter Komposition mit Isos

\begin{prop}
  Sei $F : \Cat \to \Dat$ ein Funktor. Dann ist die Klasse $\Set{f \in \Mor(\Cat)}{F(f) \text{ ist ein Iso}}$ abgeschlossen unter Retrakten.
\end{prop}

\begin{defn}
  Sei $i : A \to X$ und $p : E \to B$. Dann werden als äq. definiert:
  \begin{itemize}
    \miniitem{0.3 \linewidth}{$p$ ist \emph{$i$-injektiv}}
    \miniitem{0.35 \linewidth}{$i$ ist \emph{$p$-projektiv}}
    \miniitem{0.3 \linewidth}{$i \lhhe p$}
    \item $i$ hat die \textit{Linkshochhebungseigenschaft} (LHHE) bzgl. $p$
    \item $p$ hat die \textit{Rechtshochhebungseigenschaft} (RHHE) bzgl. $i$
    \item Für alle $f$, $g$ wie unten, sodass das Quadrat kommutiert, gibt es ein diagonales $\lambda$, sodass die Dreiecke kommutieren:
    \begin{centertikzcd}
      A \arrow[r, "g"] \arrow[d, "i"] &
      E \arrow[d, "p"] \\
      X \arrow[r, "f"] \arrow[ur, "\exists\, \lambda", dashed] &
      B
    \end{centertikzcd}
  \end{itemize}
\end{defn}

\begin{bsp}
  Wegeliftung aus der Topologie: $i : \{ 0 \} \to \cinterval{0}{1}$ erfüllt die LHHE bezüglich allen Überlagerungen $\pi : E \to B$.
\end{bsp}

\begin{samepage}
\begin{bsp}
  Ein Objekt $P$ einer ab. Kat. $\Aat$ ist genau dann \emph{projektiv}, wenn $(0 \to P)$ die LHHE bzgl. aller Epis in $\Aat$ hat.
  Dual ist $I \!\in\! \Ob(\Aat)$ injektiv g.d.w. alle Monos in $\Aat$ die LHHE bzgl. $(I \to 0)$ besitzen.
\end{bsp}

\begin{bsp}
  In $\SetC$ gilt: Alle Inj. haben die LHHE bzgl. aller Surjektionen.
\end{bsp}

\begin{lem}[\emph{Retrakt-Argument}]\mbox{}
  Sei $f \!=\! q \circ j$.
  \begin{itemize}
    \item Ist $f$ $q$-projektiv ($f \lhhe q$), so ist $f$ ein Retrakt von $j$.
    \item Ist $f$ $j$-injektiv ($j \lhhe f$), so ist $f$ ein Retrakt von $q$.
  \end{itemize}
\end{lem}

\subsection{Zellenkomplexe}

\end{samepage}

\begin{defn}
  Sei $\lambda$ eine Ordinalzahl. Eine \emph{$\lambda$-Sequenz} in einer Kategorie $\Cat$ ist ein kolimesbewahrender Funktor $X : \lambda \to \Cat$ (wobei man $\lambda$ als Präordnungskategorie aller $\beta < \lambda$ auffasst).
  Ihre \emph{transfinite Komposition} ist der induzierte Morphismus $X_0 \to \colim_{\beta < \lambda} X_\beta$.
\end{defn}

\begin{bem}
  Kolimesbewahrung bedeutet: $\colim_{\alpha < \beta} X_\alpha \!=\! X_\beta$ für alle $\beta \!<\! \lambda$.
\end{bem}

% Ausgelassen: Bsp $n$-Sequenz ($n \in \N$)

% Ausgelassen: Verklebung von zwei Halbsphären entlang $S^1$ zu einer $S^2$
% Verklebung von $[0, 1]$ an den Endpunkten zur $S^1$

\begin{defn}
  Sei $\Cat$ eine kovollständige Kategorie, $I \subset \Mor(\Cat)$ eine Menge.
  %Sei $I$ eine Menge von Morphismen in einer kovollständigen Kategorie $\Cat$.
  \begin{itemize}
    \item Ein \emph{relativer $I$-Zellenkomplex} ist eine transf. Komp. einer $\lambda$-Sequenz $Z$, sodass $\forall \, \alpha \!\in\! \Ord$ mit $\alpha \!+\! 1 \!<\! \lambda$ ein Pushoutdiagramm
    \begin{centertikzcd}[row sep=0.3cm]
      C \arrow[r] \arrow[d, "f"] \arrow[rd, phantom, "\ulcorner", very near end] &
      Z_\alpha \arrow[d] &
      \text{$\leftarrowtriangle$ \emph{Anklebeabbildung}} \\
      B \arrow[r] &
      Z_{\alpha + 1} &
      \mathrlap{\text{$\leftarrowtriangle$ \emph{Zelle}}}
      \phantom{\text{$\leftarrowtriangle$ \emph{Anklebeabbildung}}}
    \end{centertikzcd}
    mit $f \in I$ existiert. Sprechweise: \\
    "`$Z_{\alpha+1}$ entsteht aus $Z_\alpha$, indem wir $B$ längs $C$ ankleben."'
    \item Ein Objekt $A \in \Ob(\Cat)$ heißt \emph{$I$-Zellenkomplex}, wenn der Morph. $0 \to A$ aus dem initialen Obj. ein relativer $I$-Zellenkomplex ist.
  \end{itemize}
\end{defn}

\begin{bsp}
  CW-Komplexe aus der algebraischen Topologie sind $I$-Zellenkomplexe mit
  $I \coloneqq \Set{S^{n-1} \hookrightarrow B^n}{n \geq 0}$
  (und $\Cat = \Top$).
\end{bsp}

\begin{bspe}
  \begin{itemize}
    \item Identitäten $A \to A$ sind relative $I$-Zellenkomplexe.
    \item Das initiale Objekt ist ein absoluter $I$-Zellenkomplex.
  \end{itemize}
\end{bspe}

\begin{samepage}
\begin{lem}
  Sei $Z : \lambda \to \Cat$ eine $\lambda$-Sequenz.
  Sei jeder Mor. $Z_\beta \to Z_{\beta + 1}$ ($\beta + 1 < \lambda$) ein Pushout eines Morphismus aus $I$ oder ein Iso. \\
  Dann ist die transfinite Komposition von $Z$ ein $I$-Zellenkomplex.
\end{lem}

\begin{thm}
  Die Klasse der relativen $I$-Zellenkomplex ist abgeschl. unter: \\
  \inlineitem{transfiniten Kompositionen} \enspace
  \inlineitem{Isomorphismen} \enspace
  \inlineitem{Koprodukten}
\end{thm}

% Vorlesung vom 6.5.2015
\subsection{Faktorisierungssysteme}
\end{samepage}

\begin{defn}
  Eine Unterkat. $\Lat \subseteq \Cat$ heißt \emph{links-saturiert}, falls $\Lat$ abgeschl. ist unter Pushouts, transfiniten Kompositionen und Retrakten.
\end{defn}

\begin{lem}
  Sei $\Lat \subseteq \Cat$ links-saturiert. Dann ist $\Lat$ unter Koprodukten abgeschlossen und enthält alle Isomorphismen.
\end{lem}

\begin{bsp}
  Sei $R \subset \Mor(\Cat)$. Dann ist die Unterkategorie $\Lat \subseteq \Cat$ mit
  $\Mor(\Lat) \coloneqq \prescript{\lhhe}{} R \coloneqq \Set{i \!\in\! \Mor(\Cat)}{\fa{r \!\in\! R\!}\! i \lhhe r}$
  links-saturiert.
\end{bsp}

% Schwache Faktorisierungssysteme

% Ausgelassen, da trivial
\iffalse
\begin{prop}
  Für $L_1 \subseteq L_2 \subseteq \Mor(\Cat)$ gilt $L_1^\lhhe \leq L_2^\lhhe$.
\end{prop}
\fi

\begin{defn}
  \begin{itemize}
    \item $L \subseteq \Mor(\Cat)$ heißt \emph{proj. abgeschlossen}, falls $L \supseteq \prescript{\lhhe}{} (L^\lhhe)$.
    \item $R \subseteq \Mor(\Cat)$ heißt \emph{injektiv abgeschlossen}, falls $R \supseteq (\prescript{\lhhe}{} L)^\lhhe$.
  \end{itemize}
\end{defn}

\begin{prop}
  \begin{itemize}
    \item $\prescript{\lhhe}{}(L^\lhhe)$ ist die projektive Hülle von $L$, \dh{} die kleinste Klasse von Morphismen, die projektiv abgeschl. ist und $L$ umfasst.
    \item Die projektive Hülle von $L$ ist links-saturiert.
    Ist $L$ schon projektiv abgeschlossen, so ist $L$ insbesondere links-saturiert.
  \end{itemize}
\end{prop}

\begin{defn}
  \begin{itemize}
    \item Ein Paar $(L, R)$ von Klassen von Morphismen von $\Cat$ \emph{faktorisiert} $\Cat$, falls
    $\fa{f \in \Mor(\Cat)} \ex{i \in L, p \in R} f = p \circ i$.
    \item Ein faktorisierendes Paar $(L, R)$ heißt \emph{schwaches Faktorisierungssystem} (SFS), falls $L = \prescript{\lhhe}{} R$ und $R = L^\lhhe$.
    \item Ein SFS $(L, R)$ heißt \emph{orth. Faktorisierungssystem}, falls jedes $i \!\in\! L$ die eindeutige LHHE bzgl. allen $p \in R$ erfüllt.
  \end{itemize}
\end{defn}

\begin{prop}
  Sei $(L, R)$ faktorisierend. Dann ist $(L, R)$ genau dann ein SFS, wenn $L \lhhe R$ und $L$ und $R$ unter Retrakten abgeschlossen sind.
  % L \lhhe R \coloniff \fa{i \in L} \fa{p \in R} i \lhhe p
\end{prop}

\begin{bsp}
  $(\{ \, \text{Surjektionen} \, \}, \{ \, \text{Injektionen} \, \})$ ist ein (S)FS in $\SetC$
\end{bsp}

\vfill
\columnbreak

% §4. Modellkategorien
\section{Modellkategorien}

\begin{motto}
  Modellkat. sind ein Werkzeug, math. Theorien zu studieren.
\end{motto}

\begin{defn}
  Eine Klasse $W \subseteq \Mor(\Cat)$ von Morphismen erfüllt die \emph{2-aus-3-Eigenschaft}, falls für jede Komposition $h = g \circ f$ in $\Cat$ gilt: Liegen zwei der drei Morphismen $f$, $g$, $h$ in $W$, so auch der dritte.
\end{defn}

\begin{defn}
  $\Weak \subseteq \Cat$ wie eben heißt \emph{Unterkat. schwacher Äquivalenzen}, falls $\Weak$ die 2-aus-3-Eig. erfüllt und abgeschl. unter Retrakten ist.
\end{defn}

\begin{bsp}
  Sei $F : \Cat \to \Dat$ ein Funktor. Dann ist $\Weak \coloneqq F^{-1}(\{ \, \text{Isos in $\Dat$} \, \})$ eine Unterkategorie schwacher Äquivalenzen.
\end{bsp}

\begin{defn}
  Ein Tripel $(\Weak, \Cof, \Fib)$ von Unterkategorien einer Kategorie $\ModC$ heißt \emph{Modellstruktur} auf $\ModC$, falls sowohl $(\Cof, \Fib \cap \Weak)$ als auch $(\Cof \cap \Weak, \Fib)$ schwache Faktorisierungssysteme sind und $\Weak$ die 2-aus-3-Eigenschaft erfüllt.
\end{defn}

\begin{defn}
  Eine bivollständige Kategorie $\ModC$ zusammen mit einer Modellstruktur \ModStr{} heißt eine \emph{Modellkategorie}.
\end{defn}

\begin{sprech}
  Man verwendet folgende Bezeichnungen und Pfeile:
  \begin{center}
    \begin{tabular}{r l l}
      $\Weak$ & $\xrightarrow{{\sim}}$ & \emph{schwache Äquivalenz} \\
      $\Cof$ & $\xhookrightarrow{\phantom{{\sim}}}$ & \emph{Kofaserung} \\
      $\Cof \cap \Weak$ & $\xhookrightarrow{{\sim}}$ & \emph{azyklische Kofaserung} \\
      $\Fib$ & $\xtwoheadrightarrow{\phantom{{\sim}}}$ & \emph{Faserung} \\
      $\Fib \cap \Weak$ & $\xtwoheadrightarrow{{\sim}}$ & \emph{azyklische Faserung}
    \end{tabular}
  \end{center}
\end{sprech}

\begin{bem}
  Ist $(\Weak, \Cof, \Fib)$ eine Modellstruktur auf $\ModC$, so ist $(\Weak^\op, \Fib^\op, \Cof^\op)$ eine Modellstruktur auf $\ModC^\op$.
\end{bem}

\begin{bem}
  Wegen $\Cof = \prescript{\lhhe}{} (\Fib \cap \Weak)$ bzw. $\Fib = (\Cof \cap \Weak)^\lhhe$ ist das Datum $(\Weak, \Cof, \Fib)$ überbestimmt.
\end{bem}

\begin{bsp}
  Sei $\ModC$ bivollständig. Sei $\Weak \coloneqq \Cof \coloneqq \{ \, \text{Isos in $\ModC$} \, \}$. \\
  Dann wird $\ModC$ mit $\Fib \coloneqq \ModC$ eine Modellkategorie.
\end{bsp}

\begin{prop}
  In einer Modellkategorie sind $\Cof$ und $\Cof \cap \Weak$ links-saturiert.
\end{prop}

\begin{lem}
  $\Weak$ enthält alle Isomorphismen und ist unter Retrakten abgeschlossen, bildet also eine Unterkat. schwacher Äquivalenzen.
\end{lem}

\begin{nota}
  Das initiale Objekt von $\ModC$ wird mit $\emptyset$, das terminale Objekt mit $*$ bezeichnet.
  %Mit $\emptyset$ wird das initiale Objekt und mit $*$ das terminale Objekt von $\ModC$ bezeichnet.
\end{nota}

\begin{defn}
  \begin{itemize}
    \item Ein Objekt $X \in \Ob(\ModC)$ heißt \emph{kofasernd}, falls $\emptyset \to X$ eine Kofaserung ist.
    Eine azyklische Faserung $q : QX \xtwoheadrightarrow{{\sim}} X$ mit $QX$ kofasernd heißt \emph{kofasernder Ersatz} (oder Approx.) von $X$. \\
    \item Dual heißt $X \in \Ob(\ModC)$ \emph{fasernd}, falls $X$ in $\ModC^\op$ kofasernd ist und $X \xhookrightarrow{{\sim}} RX$ mit $RX$ fasernd heißt \emph{fasernder Ersatz} von $X$.
  \end{itemize}
\end{defn}

\begin{bsp}
  Sei $X \in \Ob(\ModC)$ beliebig. Dann faktorisiere $\emptyset \to X$ wie folgt:
  \begin{centertikzcd}[row sep=0.2cm]
    \emptyset \arrow[rr] \arrow[rd, hook] && X \\
    & QX \arrow[ru, twoheadrightarrow, swap, "\sim"]
  \end{centertikzcd}
  Man erhält also immer einen kofasernden Ersatz $QX$ für $X$. \\
  Dual gibt es immer einen fasernden Ersatz $RX$ für $X$.
\end{bsp}

\begin{prop}
  Seien $q : QX \xtwoheadrightarrow{{\sim}} X$ und $q' : Q' X \xtwoheadrightarrow{{\sim}} X$ zwei kofasernde Approximationen von $X$. Dann existiert eine schwache Äquivalenz $\xi : QX \xrightarrow{{\sim}} Q' X$ mit $q' \circ \xi = q$.
\end{prop}

\begin{defn}
  Ein Obj. $X$ heißt \emph{bifasernd}, falls es fasernd und kofasernd ist.
\end{defn}

\begin{prop}
  Für alle $X \in \Ob(\ModC)$ sind $RQX$ und $QRX$ schwach äquivalent und beide bifasernd.
\end{prop}

\begin{lem}[Ken Brown]
  Sei $F : \ModC \to \NatC$ ein Funktor, $\ModC$ eine Modell- kategorie, $\NatC$ besitze eine Unterkat. $\Weak'$ schwacher Äquivalenzen. \\
  Wenn $F$ azyklische Kofaserungen zwischen kofasernden Objekten nach $\Weak'$ abbildet, so bildet $F$ alle schwachen Äquivalenzen zwischen kofasernden Objekten nach $\Weak'$ ab.
\end{lem}

% Homotopien in Modellkategorien

\begin{defn}
  Sei $\ModC$ eine Modellkategorie.
  Ein \emph{Zylinderobjekt} $\Cyl{X}$ zu einem $X \!\in\! \Ob(\ModC)$ ist ein Obj. zusammen mit Morphismen wie folgt:
  \begin{centertikzcd}[row sep=0.2cm]
    X \arrow{rd}{i_0}[swap]{{\sim}} \arrow[rrd, bend left, "\id"] \\
    & \Cyl{X} \arrow{r}{p}[swap]{{\sim}} & X \\
    X \arrow{ru}{{\sim}}[swap]{i_1} \arrow[rru, bend right, swap, "\id"]
  \end{centertikzcd}
  Der Zylinder $\Cyl{X}$ heißt \emph{gut}, falls $X \amalg X \to \Cyl{X}$ eine Kofaserung ist.
  Ein guter Zylinder heißt \emph{sehr gut}, falls $p : \Cyl{X} \to X$ eine azyklische Faserung ist.
\end{defn}

\begin{bem}
  Sei die Kodiagonale $\nabla : X \amalg X \to X$ wie folgt faktorisiert:
  \begin{centertikzcd}[row sep=0.2cm]
    X \arrow[rd, hook] \\
    & X \amalg X \arrow[r, hook] \arrow[rr, bend left, "\nabla"] & \Cyl{X} \arrow[r, twoheadrightarrow, "\sim"] & X \\
    X \arrow[ru, hook]
  \end{centertikzcd}
  Dann erhalten wir ein (sehr gutes) Zylinderobjekt $\Cyl{X}$ für $X$. \\
\end{bem}

\begin{defn}
  Zwei Morphismen $f, g : X \to Y$ in $\ModC$ heißen \emph{links-homotop} (notiert $f \sim^l g$), falls ein Zylinder $\Cyl{X}$ und ein Diagramm der Form
  \begin{centertikzcd}[row sep=0.2cm]
    X \arrow{rd}[swap]{\sim} \arrow[rrd, bend left, "f"] \\
    & \Cyl{X} \arrow[r, dashed, "h"] & Y \\
    X \arrow{ru}{\sim} \arrow[rru, bend right, swap, "g"]
  \end{centertikzcd}
  existiert. Wir definieren $\pi^l(X, Y) \coloneqq \Hom_\ModC(X, Y) / \langle {\sim^l} \rangle$, wobei $\langle {\sim^l} \rangle$ die von der symmetrischen, refl. Relation ${\sim^l}$ erzeugte Äq'relation ist. \\
  Die Homotopie heißt (sehr) gut, wenn der Zylinder $\Cyl{X}$ es ist.
\end{defn}

\begin{beob}
  Sei $X \amalg X \xrightarrow{i} C \xrightarrow{p \, \sim} X$ irgendein Zylinderobjekt.
  Faktorisiere $i = q \circ i'$ in Kofaserung und azyklische Faserung. Dann ist auch
  \[ X \amalg X \xhookrightarrow{i'} X' \xrightarrow{pq \, \sim} X \]
  ein Zylinderobjekt, sogar ein gutes.
  Ebenso kann man $p$ faktorisieren und ein anderes Zylinderobjekt erhalten.
\end{beob}

\begin{lem}
  Sei $X$ kofasernd, $X \amalg X \to \Cyl{X} \to X$ ein gutes Zylinderobj. \\
  Dann sind $i_{0,1} : X \to X \amalg X \to \Cyl{X}$ azyklische Kofaserungen.
\end{lem}

\begin{lem}
  Sei $h : f \simeq^l g$. Dann: $f \in \Weak \iff g \in \Weak$.
\end{lem}

\begin{defn}
  Ein \emph{Pfadobjekt} $\PO{X}$ ist eine Faktorisierung
  \[ X \xrightarrow[\sim]{i} \PO{X} \xrightarrow{p} X \times X \]
  des Diagonalmorph. $\Delta : X \to X \times X$.
  Das Pfadobjekt $\PO{X}$ heißt gut, wenn $p$ eine Faserung und sehr gut, wenn zus. $i$ eine Kofaserung ist.
\end{defn}

\begin{defn}
  Eine \emph{Rechtshomotopie} $h : f \simeq^r g$ ist ein \kD{} der Form
  \begin{centertikzcd}[row sep=0.2cm]
    && Y \\
    X \arrow[r, "h"] \arrow[rru, bend left, "f"] \arrow[rrd, bend right, "g"] & Y^I \arrow{ru}{p_0}[swap]{{\sim}} \arrow{rd}{{\sim}}[swap]{p_1} \\
    && Y
  \end{centertikzcd}
\end{defn}

\begin{bem}
  Ein Pfadobj. in $\ModC$ ist dasselbe wie ein Zylinderobj. in $\ModC^\op$.
\end{bem}

% Ausgelassen: Dualisierte Versionen der techn. Lemmate über Zylinderobjekte

\begin{lem}
  Seien $f, g : X \to Y$ und $e : W \to X$, $d : Y \to Z$.
  \begin{itemize}
    \item $\exists \, h : f \simeq^l g \iff \exists \, h' : f \simeq^{l,\gut} g$.
    \item Sei $Y$ fasernd. Dann: $\exists \, h : f \simeq^{l,\gut} g \iff \exists \, h' : f \simeq^{l,\sg} g$
    \item $\exists \, h : f \simeq^l g \implies \exists \, h' : d \circ f \simeq^l d \circ g$
    \item $\exists \, h : f \simeq^{l,\sg} g \implies \exists \, h' : f \circ e \simeq^{l,\sg} g \circ e$
    \item Sei $X$ kofasernd. Dann ist $\simeq^l$ eine Äq'relation auf $\Hom_\ModC(X, Y)$.
  \end{itemize}
\end{lem}

\begin{kor}
  Sei $Y$ fasernd. Dann induziert Komposition eine Abbildung
  \[
    \pi^l(X, Y) \times \pi^l(W, X) \to \pi^l(W, Y), \quad
    ([g], [f]) \mapsto [g \circ f].
  \]
\end{kor}

\begin{prop}
  Seien $f, g : X \to Y$.
  \begin{itemize}
    \item Sei $X$ kofasernd. Dann: $f \simeq^l g \implies f \simeq^r g$
    \item Sei $Y$ fasernd.\phantom{ko} Dann: $f \simeq^l g \impliedby f \simeq^r g$
  \end{itemize}
\end{prop}

\begin{nota}
  Wenn $X$ kofasernd und $Y$ fasernd ist, schreibt man
  \[ \pi(X, Y) \coloneqq \pi^l(X, Y) = \pi^r(X, Y). \]
\end{nota}

\begin{thm}
  Sei $X$ kofasernd. Sei $p : Z \xtwoheadrightarrow{\sim} Y$ eine azyklische Faserung. \\
  Dann ist $p_* : \pi^l(X, Z) \to \pi^l(X, Y), \, [f] \mapsto [p \circ f]$ eine Bijektion.
\end{thm}

\begin{thm}[\emph{Whitehead}]\mbox{}\\
  Für einen Morphismus $f : X \to Y$ zw. bifasernden Objekten gilt
  \begin{align*}
    f \in \Weak \iff & \text{$f$ ist eine Homotopieäquivalenz} \\
    \coloniff & \ex{g : Y \to X} g \circ f \simeq \id_X \wedge f \circ g \simeq \id_Y.
  \end{align*}
\end{thm}

\begin{lem}
  Sei $f : X \to Y$. Seien $RX$ und $RY$ fixierte fasernde Approx. an $X$ bzw. $Y$.
  Dann hängt $Rf : RX \to RY$ bis auf Rechts- und auch Linkshomotopie nur von der Rechtshomotopieklasse von $r \circ f$ ab.
\end{lem}

\begin{acht}
  I.\,A. ist $f \mapsto R(f)$ nicht funktoriell.
\end{acht}

\subsection{Die Homotopiekategorie einer Modellkategorie}

% Ausgelassen: Erinnerung an die Lokalisierung von Ringen

% Aufgabenstellung: Sei $\ModC$ eine Modellkategorie. Gibt es eine Kategorie $\Ho \ModC$ zusammen mit einem Funktor $\gamma : \ModC \to \Ho \ModC$, sodass gilt:
% 1. $\gamma$ schickt schwache Äquivalenzen auf Isos
% 2. $\gamma$ ist initial unter dieser Voraussetzung

\begin{defn}
  Sei $\Cat$ ein Kategorie, $S \subset \Mor(\Cat)$ eine Klasse von Morphismen. Die \emph{Lokalisierung} $\Cat[S^{-1}]$ von $\Cat$ ist eine Kategorie, die folgende 2-universelle Eigenschaft erfüllt:
  \begin{itemize}
    \item $\gamma : \Cat \to \Cat[S^{-1}]$ schickt Morphismen aus $S$ aus Isos.
    \item Für jede Kategorie $\Dat$ ist $\gamma^* : [\Cat[S^{-1}], \Dat] \to [\Cat, \Dat]_{S \mapsto \text{Isos}}$ eine Kategorienäquivalenz.
  \end{itemize}
\end{defn}

\begin{bem}
  Die \spickerref{homoalg}{Homologische-Algebra-Zusammenfassung} behandelt Lokalisierung von Kategorien.
\end{bem}

\begin{defn}
  Die \emph{Homotopiekategorie} $\Ho \ModC$ einer Modellkategorie $\ModC$ ist die Lokalisierung von $\ModC$ an der Klasse der schwachen Äquivalenzen.
\end{defn}

\begin{konstr}
  Ganz explizit:
  \begin{align*}
    \Ob(\Ho \ModC) & \coloneqq \Ob(\ModC) \\
    \Hom_{\Ho \ModC}(X, Y) & \coloneqq \pi(RQX, RQY)
  \end{align*}
  Nach einem früheren Lemma ist die Komposition $([f], [g]) \mapsto [f \circ g]$ wohldefiniert.
  Der Funktor $\gamma : \ModC \to \Ho \ModC$ ist gegeben durch
  \[
    X \mapsto X, \quad
    f \mapsto [RQf].
  \]
\end{konstr}

\begin{lem}
  Sei $f : X \to Y$ in $\ModC$. Dann gilt $f \!\in\! \Weak \Leftrightarrow Qf \!\in\! \Weak \Leftrightarrow RQf \!\in\! \Weak$.
\end{lem}

\begin{lem}
  $\gamma$ wie definiert ist ein Funktor.
\end{lem}

% Probleme der Lokalisierung:
% 1. Das ist keine Kategorie im Marc'schen Sinne (da Hom-Mengen keine Mengen)
% 2. Man hat keine explizite Beschreibung für die Morphismen in der lokalisierten Kategorie

\begin{lem}
  $f \in \Weak \iff \gamma(f)$ ist ein Iso.
\end{lem}

\begin{lem}
  Sei $X$ kofasernd und $Y$ fasernd. Dann ist die Abbildung
  \[
    \pi(X, Y) \to \Hom_{\Ho \ModC}(X, Y), \quad
    [f] \mapsto [RQf]
  \]
  eine Bijektion.
\end{lem}

\begin{lem}
  Ist $F : \ModC \to \Cat$ ein Funktor, der schwache Äq. auf Isos schickt, dann identifiziert $F$ links- bzw. rechtshomotope Morphismen.
\end{lem}

\begin{lem}
  Jeder Morphismus in $\Ho \ModC$ ist Komposition von Morphismen der Form $\gamma(f)$, $f \in \Mor(\ModC)$ und der Form $\gamma(f)^{-1}$, $f \in \Weak$.
\end{lem}

\begin{lem}
  Obige Konstruktion erfüllt die geforderte univ. Eigenschaft.
\end{lem}

\begin{lem}
  Sei $\ModC_c \subset \ModC$ die volle Unterkategorie der kofasernden Objekte und $F : \ModC_c \to \Cat$ ein Funktor, der azyklische Kofaserungen auf Isos schickt. Dann identifiziert $F$ rechtshomotope Morphismen.
\end{lem}

\begin{thm}
  Ein Morphismus $p : Z \to Y$ zw. fasernden Objekten ist genau dann eine schwache Äquivalenz, wenn $p_* : \pi(X, Z) \to \pi(X, Y)$ bijektiv ist für alle kofasernden Objekte $X \in \ModC$.
\end{thm}

\begin{beob}
  Sei $X$ kofasernd und $Y$ fasernd. Dann ist $\Hom_{\Ho(\ModC)}(X, Y) = \Hom_\ModC(X, Y)/{\sim}$.
\end{beob}

\begin{defn}
  Eine Klasse $W \subseteq \Mor(\Cat)$ besitzt die \emph{2-aus-6-Eigenschaft}, wenn für alle Folgen von Morphismen
  \[ X \xrightarrow{u} Y \xrightarrow{v} Z \xrightarrow{w} K \qquad \in \Cat \]
  gilt: Wenn $v \circ u$ und $w \circ v$ aus $W$ sind, so auch $u$, $v$, $w$ und $w \circ v \circ u$.
\end{defn}

\begin{beob}
  Die Klasse der Isomor. besitzt die 2-aus-6-Eigenschaft.
\end{beob}

\begin{kor}
  Die Klasse der schwache Äquivalenzen in einer Modellkategorie besitzt die 2-aus-6-Eigenschaft.
\end{kor}

% §5. Kombinatorische und eigentliche Modellkategorien
\section{Klassen von Modellkategorien}

% §5.1
\subsection{Lokal präsentierbare Kategorien}

% soll andeuten: Die Objekte einer lokal präsentierbareen Kategorien sind präsentierbar, nicht die Kategorie als solche

\begin{motto}
  Eine lokal präsentierare Kategorie ist eine große Kategorie, welche erzeugt wird von kleinen Objekten unter kleinen Kolimiten.
\end{motto}

\begin{defn}
  Eine $\infty$-große Kardinalzahl $\kappa$ heißt \emph{regulär}, wenn die Vereinigung von weniger als $\kappa$ vielen Mengen, die alle weniger als $\kappa$-viele Elem. enthalten, selbst weniger als $\kappa$-viele Elemente enthält.
\end{defn}

\begin{bem}
  Zu jeder Kardinalzahl $\lambda$ existiert ein reguläres $\kappa$ mit $\lambda \leq \kappa$.
\end{bem}

\begin{defn}
  Sei $\kappa$ eine Kardinalzahl. Eine Kategorie heißt \emph{$\kappa$-klein}, falls sie nur $\kappa$-viele Morphismen besitzt.
\end{defn}

\begin{bem}
  Sei $\kappa$ regulär. Dann ist eine Kat. bereits dann $\kappa$-klein, falls sie nur $\kappa$-viele Objekte besitzt und alle Hom-Mengen $\kappa$-klein sind.
\end{bem}

\begin{defn}
  Eine Kategorie heißt \emph{$\kappa$-filtriert}, wobei $\kappa$ eine reguläre Kardinalzahl ist, wenn jedes $\alpha$-kleine Diagramm in der Kategorie einen Kokegel besitzt, wobei $\alpha < \kappa$.
\end{defn}

\begin{defn}
  Eine teilweise geordnete Menge $(I, \leq)$ heißt \emph{$\alpha$-gerichtet}, falls die zugehörige Kategorie $\alpha$-filtriert ist, \dh{} jeweils weniger als $\alpha$-viele Elemente haben eine obere Schranke.
\end{defn}

\begin{bem}
  Sei $\lambda \geq \kappa$. Dann ist jede $\lambda$-filtrierte Kategorie auch $\kappa$-filtriert.
\end{bem}

\begin{defn}
  Ein Objekt $X$ einer Kat. $\Cat$ heißt \emph{$\kappa$-kompakt} oder \emph{$\kappa$-klein}, wenn $\Hom(X, \blank) : \Cat \to \SetC$ mit $\kappa$-filtrierten Kolimiten vertauscht:
  \[
    \Colim_i \Hom_\Cat(X, T_i) \xrightarrow{\cong} \Hom_\Cat(X, \Colim_i T_i)
  \]
  für alle $\kappa$-filtrierte Diagramme $(T_i)_{i \in \Iat}$.
\end{defn}

\begin{defn}
  Ein Objekt heißt genau dann \emph{klein}, wenn es $\kappa$-kompakt ist für irgendeine reguläre Kardinalzahl $\kappa$.
\end{defn}

\iffalse
\begin{idee}
  Sei $X$ eine endliche Menge. Sei $X \subset \cup_{i \in \N} T_i$, $T_i \subseteq T_{i+1}$. Dann liegt $X$ schon vollständig in einem der $T_i$.
\end{idee}
\fi

% TODO: Was sind die interessantesten Beispiele?
\begin{bspe}
  \begin{itemize}
    \item Jede endliche Menge ist $\aleph_0$-kompakt in $\SetC$.
    \item Jeder endlich-dim. VR ist $\aleph_0$-kompakt in $\Vect(\R)$.
    \item Jeder endlich-präsentierte Modul ist $\aleph_0$-kompakt in $\Mod(R)$.
    \item Unendliche Mengen sind nicht $\aleph_0$-kompakt in $\SetC$.
    \item Jeder nicht diskrete topologische Raum ist nicht $\aleph_0$-kompakt.
    \item $\SetC$ ist lokal $\aleph_0$-präsentierbar mit $S = \{ \heartsuit \}$.
    \item $\Mod(R)$ ist lokal $\aleph_0$-präsentierbar mit $S = \Set{R^n/\im(A)}{n \geq 0, A \in R^{n \times m}, m \geq 0}$
  \end{itemize}
\end{bspe}

\begin{defn}
  Eine \emph{lokal $\kappa$-präsentierbare Kategorie} ist eine lokal kleine und kovollständige Kategorie, sodass eine {\em Menge} $S \subseteq \Ob(\Cat)$ von $\kappa$-kompakten Objekten existiert, sodass jedes Objekt aus $\Cat$ kleiner Kolimes von Objekten aus $S$ ist.
\end{defn}

% Ausgelassen: Intuition

\begin{defn}
  Eine Kategorie heißt genau dann \emph{lokal präsentierbar}, wenn sie lokal $\kappa$-präsentierbar für eine reguläre Kardinalzahl $\kappa$ ist.
\end{defn}

\begin{lem}
  Ist $\Cat$ lokal präsentierbar, so auch $\Cat/X$ mit $X \in \Ob(X)$.
\end{lem}

\begin{bspe}
  \begin{itemize}
    \item $\sSet$ ist lokal präsentierbar.
    \item Sei $\Cat$ klein. Dann ist $\PShSet(\Cat) \!=\! [\Cat^\op, \SetC]$ lokal präsentierbar.
    \item $\FinSetC$ ist nicht lokal präsentierbar (weil nicht kovollständig)
  \end{itemize}
\end{bspe}

\begin{fun}
  Sei $\Cat$ lokal präsentierbar. Wenn auch $\Cat^\op$ lokal präsen- tierbar ist, dann ist $\Cat$ die zu einer Quasiordnung gehörige Kategorie!
\end{fun}

\begin{lem}
  Sei $X : \Iat \times \Jat \to \SetC$ ein Funktor, wobei $\Iat$ $\alpha$-filtriert und $\Jat$ $\alpha$-klein.
  Dann ist der kanonische Isomorphismus $\colim_i \lim_j X(i, j) \to \lim_j \colim_i X(i, j)$ eine Bijektion.
\end{lem}

\begin{bsp}
  $\alpha$-kleine Kolimiten $\alpha$-kompakter Obj. sind wieder $\alpha$-kompakt.
\end{bsp}

\subsection{Kombinatorische Modellkategorien}

\begin{lem}[\emph{Kleines-Objekt-Argument}]\mbox{}\\
  Sei $\Cat$ lokal präsentierbar, $\Iat \subset \Mor(\Cat)$ eine Menge, $\Cell(\Iat)$ die Unterkat. der relativen $\Iat$-Zellenkomplexe und $\Coff(\Iat)$ die Unterkat. der Retrakte von $\Cell(\Iat)$.
  Dann ist $(\Coff(\Iat), \Iat^\lhhe)$ ein SFS.
\end{lem}

\begin{defn}
  \begin{itemize}
    \item Eine Modellkategorie \ModStr{} heißt \emph{kofasernd erzeugt}, wenn {\em Mengen} $\Iat, \Jat \subset \Mor(\ModC)$ mit $\Cof = \Coff(\Iat)$ und $\Cof \cap \Weak = \Coff(\Jat)$ existieren.
    \item Lokal präsentierbare und kofasernd erzeugte Modellkategorien heißen \emph{kombinatorisch}.
  \end{itemize}
\end{defn}

\begin{sprech}
  Die Kof. in $\Iat$ heißen \emph{erzeugende Kofaserungen}, die in $\Jat$ \emph{azyklische erzeugende Kofaserungen}.
\end{sprech}

\begin{satz}
  Sei $\ModC$ eine lokal präsentierbare Kategorie. % => $\ModC$ bivollständig
  Sei $\Weak \subseteq \Mor(\ModC)$ eine Unterkat. schw. Äquivalenzen.
  Seien $\Iat, \Jat \subseteq \Mor(\ModC)$ Mengen.
  Dann sind $\Iat$ und $\Jat$ genau dann erzeugende (azyklische) Kofaserungen einer Modellstruktur auf $\ModC$, falls \\[2pt]
  \inlineitem{$\Cell(\Jat) \subseteq \Weak$ \enspace (Azyklizität)} \quad
  \inlineitem{$\Iat^\lhhe = \Jat^\lhhe \cap \Weak$ \enspace (Kompatibilität)}
\end{satz}

\subsection{Eigentliche Modellkategorien}

\begin{defn}
  Eine Modellkategorie $\ModC$ heißt \emph{linkseigentlich}, falls für alle Pushouts der Form
  \begin{centertikzcd}
    A \arrow[r, "\sim"] \arrow[d, hook] \arrow[rd, phantom, "\ulcorner", very near end] & B \arrow[d, hook] \\
    X \arrow[r, "g"] & Y
  \end{centertikzcd}
  auch der Morphismus $g : X \to Y$ eine schwache Äquivalenz ist. \\
  $\ModC$ heißt rechtseigentlich, falls $\ModC^\op$ linkseigentlich ist, \dh{} Pullbacks schwacher Äquivalenzen längs Faserungen wieder schwache Äquivalenzen sind.
\end{defn}

\begin{bsp}
  Eine Modellkategorie, in der jedes Objekt kofasernd ist, ist linkseigentlich.
\end{bsp}

\begin{defn}
  $\ModC$ heißt \emph{eigentlich}, falls $\ModC$ links- und rechtseigentlich ist.
\end{defn}

\begin{prop}
  In jeder Modellkategorie ist der Pushout einer schwachen Äquivalenz zwischen kofasernden Objekten längs Kofaserungen wieder eine schache Äquivalenz.
\end{prop}

% Einschub: Homotopie-Erweiterungs-Eigenschaft
\begin{bem}
  Gute Homotopien kann man längs Kofaserungen erweitern:
  \begin{centertikzcd}
    A \arrow[r, "H"] \arrow[d, hook, "i"] & Y^I \arrow{d}{p_0}[swap, twoheadrightarrow]{{\sim}} \\
    X \arrow[r, "f"] \arrow[ru, dashed, "\overline{h}"] & Y
  \end{centertikzcd}
\end{bem}

% Vorlesung vom 27.5.2015

\begin{prop}
  Eine Modellkategorie $\ModC$ ist genau dann links-eigentlich, wenn für alle Diagramme der Form
  \begin{centertikzcd}
    A \arrow{d}{f}[swap]{{\sim}} \arrow[r, hookleftarrow, "i"] & C \arrow[r, "k"] \arrow{d}{g}[swap]{{\sim}} & B \arrow{d}{h}[swap]{{\sim}} \\
    A \arrow[r, hookleftarrow, "j"] & C \arrow[r, "l"] & B
  \end{centertikzcd}
  auch der ind. Mor. $A \cup_C B \to A' \cup_{C'} B'$ eine schwache Äq. ist.
\end{prop}

% §6. Quillen-Adjunktionen
\subsection{Quillen-Adjunktionen}

\begin{motto}
  Wir wollen Modellstrukturen und -kategorien vergleichen.
\end{motto}

% 6.1. Abgeleitete Funktoren

\begin{defn}
  Sei $\ModC$ eine Modellkategorie, $\HatC$ eine beliebige Kategorie. \\
  Ein Funktor $F : \ModC \to \HatC$ heißt \emph{homotopisch}, falls $F$ die schwachen Äquivalenzen in $\ModC$ auf Isomorphismen in $\HatC$ abbildet.
\end{defn}

\begin{bem}
  Homotopische Funktoren faktorisieren über $\ModC[\Weak^{-1}]$.
\end{bem}

\begin{bsp}
  Sei $F : \ModC \to \NodC$ ein Funktor zw. Modellkategorien, der schwache Äquivalenzen erhält.
  Dann ist $\delta \circ F : \ModC \to \Ho(\NodC)$ homotopisch, wobei $\delta : \NodC \to \Ho(\NodC)$ die Lokalisierung ist.
\end{bsp}

\begin{bem}
  Solch ein Funktor $F : \ModC \to \NodC$ induziert einen Funktor $\Ho(M) \to \Ho(\NodC)$.
\end{bem}

% XXX: Definition kürzen?
\begin{defn}
  Ein \emph{linksabgeleiteter Funktor} eine Funktors $F : \ModC \to \HatC$ ist ein Funktor $\LD{F} : \Ho(\ModC) \to \HatC$ zusammen mit einer natürlichen Transformation $\mu : \LD{F} \circ \gamma \Rightarrow F$, sodass für alle weiteren Funktoren $G : \Ho(\ModC) \to \HatC$ und nat. Transformationen $\xi : G \circ \gamma \Rightarrow F$ genau eine natürliche Transformation $\nu : G \Rightarrow \LD{F}$ existiert mit $\xi = \mu \circ \nu$, \dh{} $\Nat(G, \LD{F}) \cong \Nat(G \circ \gamma, F)$ ist für alle $G$ eine Bijektion, \dh{} eine Linksableitung von $F$ ist nichts anderes als eine Rechts-Kan-Erweiterung von $F$ längs $\gamma$.
  \begin{centertikzcd}
    \ModC \arrow[d, "\gamma"] \arrow[r, "F"] & \HatC \\
    \Ho(\ModC) \arrow[ru, "\LD{F}"]
  \end{centertikzcd}
  Analog ist eine \emph{Rechtsableitung} $\RD{F}$ von $F$ eine Linkskanerweiterung von $F$ längs $\gamma : \ModC \to \Ho(\ModC)$.
\end{defn}

\begin{satz}
  Sei $F : \ModC \to \HatC$ ein Funktor, der azyklische Kofaserungen zwischen kof. Obj. auf Isomorphismen abbildet.
  Dann existiert $\LD{F}$ und $\mu_X : \LD{F}(X) \to F(X)$ ist ein Iso für alle kofasernden $X$.
\end{satz}

\begin{konstr}
  Sei $h \ModC_c$ die volle Unterkategorie der kof. Objekte von $\ModC$ modulo Rechts-Homotopie.
  Betrachte die Komposition
  \[ \ModC \xrightarrow{Q} h \ModC_c \xrightarrow{F_*} \HatC. \]
  Dabei ist $Q$ der kofasernde Ersatz und $F_*$ wird induziert von $F$, da $F$ homotope Morphismen identifiziert.
  Nach Ken Brown bildet die Komposition schwache Äquivalenzen auf Isos ab und induziert daher den gesuchten Funktor $\LD{F} : \Ho(\ModC) \to \HatC$ mit $\LD{F} \circ \gamma = F_* \circ Q$.
  Definiere $\mu : \LD{F} \circ \gamma \!\to\! F$ durch $\mu_X \!\coloneqq\! F(q : QX \!\to\! X)$ für $X \!\in\! \Ob(\ModC)$.
  Falls $X$ selbst kofasernd ist, so ist $q$ eine schwache Äquivalenz zw. kofasernden Objekten und somit $\mu_X = F(q)$ ein Isomorphismus.
\end{konstr}

\begin{defn}
  Sei $F : \ModC \to \NodC$ ein Funktor zwischen Modellkategorien.
  Eine \emph{totale Linksableitung} $\LD{F}$ ist ein Funktor $\LD{F} : \Ho(\ModC) \to \Ho(\NodC)$, sodass $\LD{F}$ die Linksableitung von $\delta \circ F$ ist:
  \begin{centertikzcd}
    \ModC \arrow[r, "F"] \arrow[d, "\gamma"] & \NodC \arrow[d, "\delta"] \\
    \Ho(\ModC) \arrow[r, "\LD{F}"] \arrow[ru, Rightarrow] & \Ho(\NodC)
  \end{centertikzcd}
  Ein Funktor $F : \ModC \to \NodC$ bilde azyklische Kofaserungen zwischen kofasernden Objekten auf schache Äquivalenzen ab.
  Dann existiert seine totale Linksableitung $\LD{F} : \Ho(\ModC) \to \Ho(\NodC)$.
\end{defn}

% 6.2. Quillen-Adjunktionen und -Äquivalenzen

\begin{defn}
  Eine Adjunktion $F : \adj{\ModC}{\NodC} : U$ von Modellkategorien heißt \emph{Quillen-Adjunktion}, falls eine der folgenden äquivalenten Bedingungen erfüllt ist:
  \begin{itemize}
    \item $F$ erhält Kofaserungen und $U$ erhält Faserungen,
    \item $F$ erhält Kofaserungen und azyklische Kofaserungen,
    \item $U$ erhält Faserungen und azyklische Faserungen,
    \item $F$ erhält azyklische Kofaserungen und $U$ azyklische Faserungen.
  \end{itemize}
\end{defn}

\begin{bem}
  Die Äquivalenz folgt aus $Fi \lhhe p \iff i \lhhe Up$.
\end{bem}

\begin{defn}
  Eine Quillen-Adj. $(F, U)$ heißt \emph{Quillen-Äquivalenz}, falls
  \[ \fa{X \in \ModC_c, Y \in \NodC_f} (FX \to Y) \in \Weak \iff (X \to UY) \in \Weak. \]
\end{defn}

\begin{satz}
  Sei $(F, U)$ eine Quillenadjunktion. Dann existieren $\LD{F}$, $\RD{U}$ und bilden eine Adjunktion $\LD{F} : \Ho(\ModC) \rightleftarrows \Ho(\NodC) : \RD{U}$. \\
  Ist $(F, U)$ sogar eine Quillenäquivalenz, so ist $(\LD{F}, \RD{U})$ eine Adjunktion aus Äquivalenzen.
\end{satz}

\begin{kor}
  Quillenäq. Modellkat'n haben äquivalente Homotopiekat'n.
\end{kor}

\begin{prop}
  Für eine Quillenadjunktion $F : \ModC \rightleftarrows \NodC : U$ sind äquivalent:
  \begin{itemize}
    \item $(F, U)$ ist eine Quillenäquivalenz
    \item $(\LD{F}, \RD{U})$ ist eine Adjunktion von Äquivalenzen
    \item $F$ reflektiert schw. Äq'n zw. kof. Objekten und die Komposition $FQUY \xrightarrow{F(q_{UY})} FUY \xrightarrow{\epsilon} Y$ ist eine schw. Äq. für alle fas. $Y$.
    \item $U$ reflektiert schw. Äq'n zw. fas. Objekten und die Komposition $X \xrightarrow{\eta} UFX \xrightarrow{U(r_{FX})} URFX$ ist eine schw. Äq. für alle kof. $X$.
  \end{itemize}
  Falls $U$ schw. Äq'n in $\NodC$ erzeugt, dann ist auch äquivalent: \\
  \inlineitem{$\eta : X \to UFX$ ist eine schwache Äq. für alle kofasernden $X$.} \\[2pt]
  Falls $F$ schw. Äq'n in $\ModC$ erzeugt, dann ist auch äquivalent: \\
  \inlineitem{$\eta : X \to UFX$ ist eine schwache Äq. für alle kofasernden $X$.}
\end{prop}

\begin{defn}
  Sei $f : A \to B$ ein Mor. in der Modellkat. $\ModC$. Dieser induziert Funktoren $f^* : B/\ModC \to A/\ModC$ und $f_* : \ModC/A \to \ModC/B$.
  Der Funktor $f^*$ besitzt einen Linksadj. $f_{!} : A/\ModC \to B/\ModC$, der durch Pushout entlang $f$ geg. ist, und $f_*$ besitzt einen Rechtsadj. $f^{!} : \ModC/B \!\to\! \ModC/A$.
\end{defn}

\begin{prop}
  $\ModC$ ist genau dann linkseigentlich, wenn $(f_{!}, f^*)$ eine Quillenadjunktion ist und genau dann rechtseigentlich, wenn $(f_*, f^{!})$ eine Quillenadjunktion ist für alle schwachen Äquivalenzen $f$.
\end{prop}

% Vorlesung vom 3.6.2015

\begin{satz}
  Sei $F : \ModC \rightleftarrows \NodC : U$ eine Adj. von einer komb. Modellkat. $\ModC$ mit erz. Kofaserungen $I$ und erz. azyklischen Kofaserungen $J$ und einer lokal präsentierbaren Kategorie $\NodC$.
  Der Funktor $U$ erzeuge schwache Äquivalenzen in $\NodC$ (\dh{} wir nennen $f \in \Mor(\NodC)$ eine schwache Äquivalenz, falls $U(f)$ eine schwache Äquivalenz ist). \\
  Dann wird $\NodC$ eine Modellkategorie mit erzeugenden Kofaserungen $FI$ und erzeugenden azyklischen Kofaserungen $FJ$, falls gilt: \\
  Jeder relative $FJ$-Zellenkomplex ist eine schwache Äquivalenz (\dh{} $U(\Cell(FJ)) \subset \Weak_\ModC$).
  Bezüglich dieser Modellstruktur auf $\NodC$ wird $(F, U)$ zu einer Quillenadjunktion.
\end{satz}

\subsection{Scheibenkategorien als Modellkategorien}

\begin{lem}
  Sei $\ModC$ eine Modellkategorie, $X \in \Ob(\ModC)$ ein Objekt. \\
  Dann sind die Scheibenkategorien $X/\ModC$ und $\ModC/X$ Modellkat'n, wobei die Modellstruktur vom Vergissfunktor $U : X/\ModC \to \ModC$ bzw. $U : \ModC/X \to \ModC$ erzeugt wird, \dh{} ein Mor. $f$ ist genau dann eine Faserung/Kofaserung/schwache Äquivalenz, wenn $U(f)$ es ist.
\end{lem}

\begin{lem}
  \begin{itemize}
    \item Ist $\ModC$ links- oder rechtseigentlich, so auch $\ModC/X$ u. $X/\ModC$.
    \item Ist $\ModC$ eigentlich, so auch $\ModC/X$ und $X/\ModC$.
    \item Ist $\ModC$ kofasernd erzeugt, so auch $\ModC/X$.
    \item Ist $\ModC$ kombinatorisch, so auch $\ModC/X$.
  \end{itemize}
\end{lem}

% 6.3. Monoidale Modellkategorien
\subsection{Monoidale Modellkategorien}

\begin{defn}
  Eine \emph{monoidale Kategorie} ist eine Kategorie $\Cat$ zusammen mit einem Bifunktor $\otimes : \Cat \times \Cat \to \Cat$, einem Objekt $\UnitOb \in \Ob(\Cat)$, natürlichen Isomorphismen $\alpha : (\blank \otimes \blank) \otimes \blank \Rightarrow \blank \otimes (\blank \otimes \blank)$, $\lambda : \UnitOb \otimes \blank \Rightarrow \blank$ und $\rho : \blank \otimes \UnitOb \Rightarrow \blank$, sodass die Kohärenzdiagramme aus der Definition einer schwachen 2-Kategorie kommutieren.
\end{defn}

\begin{bem}
  Eine monoidale Kategorie ist das gleiche wie eine 2-Kategorie mit nur einem Objekt.
\end{bem}

\begin{bspe}
  Monoidale Kategorien sind: \quad
  \inlineitem{$(\SetC, \times, \LonelyHeart)$} \\
  \inlineitem{$(\BMod{R}{R}, \otimes_R, R)$ wobei $R$ ein Ring mit Eins ist}
\end{bspe}

\begin{defn}
  Eine \emph{symm. monoidale Kategorie} ist eine monoidale Kat. zusammen mit einem nat. Isomorphismus $\gamma : X \otimes Y \to Y \otimes X$, sodass die geeigneten Kohärenzdiagramme kommutieren. Es reicht aus, zu zeigen, dass folgende Diagramme kommutierten:
  \begin{centertikzcd}[column sep=1.5cm, row sep=0.5cm]
    (Y \otimes X) \otimes Z \arrow[d, "\alpha"] & (X \otimes Y) \otimes Z \arrow[l, swap, "\gamma \otimes \id_Z"] \arrow[r, "\alpha"] & X \otimes (Y \otimes Z) \arrow[d, "\gamma"] \\
    Y \otimes (X \otimes Z) \arrow[r, "\id_Y \otimes \gamma"] & Y \otimes (Z \otimes X) & (Y \otimes Z) \otimes X \arrow[l, swap, "\alpha"]
  \end{centertikzcd}
  \begin{centertikzcd}
    \UnitOb \otimes X \arrow[rr, "\gamma"] \arrow[dr, "\lambda"] && X \otimes \UnitOb \arrow[dl, "\rho"] \\
    & X
  \end{centertikzcd}
\end{defn}

% Ausgelassen: Beispiel: Supervektorräume

\begin{defn}
  Ein \emph{monoidaler Funktor} zwischen (symm.) monoidalen Kategorien $\Cat$ und $\Dat$ ist ein Funktor $F : \Cat \to \Dat$ zusammen mit natürlichen Isomorphismen $F \blank \otimes_\Dat F \blank \Rightarrow F (\blank \otimes_\Cat \blank)$ und $F \UnitOb_\Cat \Rightarrow \UnitOb_\Dat$, welche verträglich mit $\alpha$, $\lambda$, $\rho$ (und eventuell $\gamma$) sind.
\end{defn}

\begin{bsp}
  $\SetC \to \LMod{R}, \enspace X \mapsto \text{freier $R$-Modul mit Basis $X$}$
  %ist ein monoidaler Funktor.
\end{bsp}

\begin{defn}
  Seien $F, G : \Cat \to \Dat$ monoidale Funktoren. Eine natürliche Transformation $\eta : F \Rightarrow G$ heißt \emph{monoidal}, wenn folgende Diagramme kommutieren:
  \begin{centertikzcd}
    FX \otimes FY \arrow[r] \arrow[d, "\eta_X \otimes \eta_Y"] & F(X \otimes Y) \arrow[d, "\eta_{X \otimes Y}"] && \UnitOb \arrow[ld] \arrow[rd] \\
    GX \otimes GY \arrow[r] & G(X \otimes Y) & F(\UnitOb) \arrow[rr, "\eta_\UnitOb"] && G(\UnitOb)
  \end{centertikzcd}
\end{defn}

\begin{defn}
  Sei $\Cat$ eine monoidale Kategorie. Ein \emph{Rechts-$\Cat$-Modul} ist eine Kategorie $\Dat$ mit einem Funktor $\otimes : \Dat \times \Cat \to \Dat$ und \ldots
\end{defn}

\begin{bsp}
  Die Kat. $\Dat$ besitze kleine Koprodukte.
  Dann wird $\Dat$ zu einem $\SetC$-Modul durch $\times = \otimes : \Dat \to \SetC \to \Dat, \enspace (X, I) \mapsto \amalg_{i \in I} X$
\end{bsp}

\begin{defn}
  Sei $\Cat$ monoidale Kategorie. Ein Funktor $F : \Dat \to \Dat'$ zwischen $\Cat$-Rechts-Moduln $\Dat$ und $\Dat'$ heißt \emph{$\Cat$-Modulfunktor}, falls $F(X) \otimes I$ und $F(X \otimes I)$ natürlich isomorph sind.
\end{defn}

\begin{defn}
  Seien $\Cat$, $\Dat$ monoidale Kat'en und $i : \Cat \to \Dat$ ein monoidaler Funktor.
  Dann heißt $(\Dat, i)$ eine \emph{$\Cat$-Algebra}.
  Morphismen von $\Cat$-Algebren sind kommutative Quadrate von monoidalen Funktoren.
\end{defn}

% Ausgelassen: alle Definitionen mit symmetrisch

\begin{defn}
  Eine $\Cat$-Algebra $\Dat$ heißt \emph{zentral}, falls $i(A) \otimes_\Dat B \cong B \otimes_\Dat i(A)$ natürlich für alle $A \in \Ob(\Cat)$, $B \in \Ob(\Dat)$.
\end{defn}

\begin{bem}
  Ist die $\Cat$-Algebra $\Dat$ symmetrisch, so auch zentral.
\end{bem}

\begin{defn}
  Seien $\Cat$, $\Dat$, $\Eat$ Kategorien. Eine \emph{Adjunktion in 2 Variablen} oder Biadjunktion besteht aus Funktoren
  \[
    \otimes : \Cat \times \Dat \to \Eat, \quad
    \Hom_r : \Dat^\op \times \Eat \to \Cat,
    \Hom_l : \Cat^\op \times \Eat \to \Dat
  \]
  und natürlichen Isomorphismen
  \[
    \Hom_\Dat(D, \Hom_l(C, E)) \cong \Hom_\Eat(C \otimes D, E) \cong \Hom_\Cat(C, \Hom_r(D, E)).
  \]
\end{defn}

\begin{nota}
  $\prescript{C}{} E \coloneqq \Hom_l(C, E)$,
  $E^D \coloneqq \Hom_r(D, E)$
\end{nota}

\begin{bem}
  $k \otimes i \lhhe p \iff k \lhhe \Hom_r(i, p) \iff i \lhhe \Hom_l(k, p)$
\end{bem}

\begin{bsp}
  Seien $R$, $S$, $T$ drei Ringe, $\Cat \coloneqq \BMod{R}{S}$, $\Dat \coloneqq \BMod{S}{T}$, $\Eat \coloneqq \BMod{R}{T}$. Eine Biadjunktion ist dann gegeben durch
  \begin{alignat*}{3}
    & \otimes : \Cat \times \Dat \to \Eat, \quad & (M, N) & \mapsto M \otimes_S N, \\
    & \Hom_r : \Dat^\op \times \Eat \to \Cat, \quad & (N, P) & \mapsto \Hom_{\RMod{T}}(N, P), \\
    & \Hom_l : \Cat^\op \times \Eat \to \Dat, \quad & (M, P) & \mapsto \Hom_{\LMod{R}}(M, P).
  \end{alignat*}
\end{bsp}

\begin{defn}
  Eine monoidale Kategorie $(\Cat, \otimes, \UnitOb)$ heißt \emph{monoidal abgeschlossen}, wenn $\otimes$ Teil einer Biadjunktion ist.
\end{defn}

\begin{bspe}
  \inlineitem{$(\BMod{R}{R}, \otimes_R, R)$} \quad
  \inlineitem{$(\SetC, \times, \LonelyHeart)$}
\end{bspe}

% Ausgelassen: Gegenbeispiel: Topologische Räume
% Aber es funktioniert: kompakt erzeugte schwach-Hausdorff-Räume

\begin{defn}
  Sei $\otimes : \Cat \times \Dat \to \Eat$ Teil einer Biadjunktion, $\Cat$, $\Dat$ und $\Eat$ Modellkategorien.
  Dann heißt $\otimes$ \emph{Quillen-Biadjunktion}, falls für alle Kof'en $(f : U \hookrightarrow V) \in \Cat$, $(g : W \hookrightarrow X) \in \Dat$ der Morphismus
  \[ f \square g : P(f, g) \coloneqq V \otimes W \cup_{U \otimes W} U \otimes X \to V \otimes X \]
  eine Kofaserung in $\Eat$ ist, welche azyklisch ist, wenn $f$ oder $g$ azyklisch ist.
\end{defn}

\begin{lem}
  Die Bedingung ist äquivalent zu:
  Für alle Kofaserungen $(g : W \hookrightarrow X) \in \Dat$ und Faserungen $(p : Y \twoheadrightarrow Z) \in \Eat$ ist
  \[ \Hom_{r,\square} : \Hom_r(X, Y) \to \Hom_r(X, Z) \times_{\Hom_r(W, Z)} \Hom_r(W, Y) \]
  eine Faserung und azyklisch, wenn $g$ oder $p$ es ist.
  Analog für $\Hom_l$.
\end{lem}

\begin{prop}
  Sei $\Cat \!\times\! \Dat \to \Eat$ ein Quillenbifunktor.
  Ist $C \in \Ob(\Cat)$ kofasernd, so ist $C \otimes \blank : \Dat \to \Eat$ ein Quillenfunktor mit Rechtsadj. $\Hom_l(C, \blank)$.
\end{prop}

\begin{bem}
  Analog: Sei $E$ fasernd.
  Dann ist $\Hom_r(\blank, E) : \Dat \to \Cat^\op$ ein Quillen-Links-Adjungierter zu $\Hom_l(\blank, E) : \Cat^\op \to \Dat$.
\end{bem}

\begin{lem}
  Sei $\otimes : \Cat \!\times\! \Dat \to \Eat$ eine Biadj, $I \!\subseteq\! \Mor(\Cat)$, $J \!\subseteq\! \Mor(\Dat)$ Mengen. \\
  Dann gilt:
  $\Coff(I) \square \Coff(J) \subseteq \Coff(I \square J)$
  mit $\Coff(K) \!\coloneqq\! \prescript{\lhhe}{} (K^\lhhe)$.
\end{lem}

\begin{satz}
  Seien $(\Cat, I, J)$, $(\Dat, I', J')$ kombinatorische Modellkategorien. \\
  Dann ist $\otimes : \Cat \times \Dat \to \Eat$ genau dann ein Quillenbifunktor, wenn $I \square I'$ Kofaserungen in $\Eat$ und $I \square J'$, $J \square I'$ jeweils azyklische Kofaserungen in $\Eat$ sind.
\end{satz}

\begin{defn}
  Eine \emph{monoidale Modellkategorie} ist eine Modellkategorie $\ModC$ mit monoidal abgeschlossener Struktur $(\ModC, \otimes, \UnitOb)$, sodass
  \begin{itemize}
    \item $\otimes : \ModC \times \ModC \to \ModC$ ein Quillenbifunktor und
    \item $Q \UnitOb \otimes X \to \UnitOb \otimes X \cong X$ und $X \otimes Q \UnitOb \to X \otimes \UnitOb \cong X$ für alle kofasernden $X$ jeweils schwache Äquivalenzen sind.
  \end{itemize}
\end{defn}

\begin{bem}
  Die zweite Bedingung ist äquivalent zu:
  \[
    X \cong \Hom_r(\UnitOb, X) \to \Hom_r(Q \UnitOb, X), \quad
    X \cong \Hom_l(\UnitOb, X) \to \Hom_l(Q \UnitOb, X)
  \]
  sind schwache Äquivalenzen für alle fasernden $X$.
\end{bem}

\begin{beob}
  Sei $\ModC$ eine mon. Modellkat, $(A \xrightarrow{i} X), (E \xrightarrow{p} B) \in \ModC$.
  Es gilt
  \[ i \lhhe p \iff (\Hom_\ModC(X, E) \to P(i, p)) \text{ ist surjektiv.} \]
\end{beob}

% Vorlesung vom 10.6.2015

\begin{defn}
  % XXX: Warum wurde in der Vorlesung noch Bivollständigkeit gefordert?
  Eine Kategorie $\Cat$ heißt \emph{kartesisch abgeschlossen}, falls $(\Cat, \times, *)$ eine abgeschlossene monoidale Kategorie ist.
\end{defn}

\begin{bsp}
  Sei $\Cat$ eine bivollständige, kartesisch abgeschlossene Kategorie.
  Sei $\Cat_* \coloneqq */\Cat$.
  Das initiale und terminale Objekt dieser Kategorie ist $\id_*$, sie ist also punktiert.
  Für $X, Y \in \Ob(\Cat_*)$ definiere $X \wedge Y \in \Ob(\Cat)$ durch folgenden Pushout:
  \begin{centertikzcd}
    X \amalg Y \arrow[r] \arrow[d] & X \times Y \arrow[d] \\
    * \arrow[r] & X \wedge Y
  \end{centertikzcd}
  Für $X \in \Ob(\Cat)$ sei $X_+ \coloneqq X \amalg * \in \Ob(\Cat_*)$. \\
  Es besteht die Adj. $(\blank)_+ : \Cat \rightleftarrows \Cat_* : U$, wobei $U$ der Vergissfunktor ist.
  Mit $S^0 \coloneqq *_+ = * \amalg *$ wird $(\Cat_*, \wedge, S^0)$ zu einer symmetrischen monoidalen Kategorie und $(\blank)_+$ zu einem monoidalen Funktor. \\
  % \dh{} $(X \times Y)_+ \cong X_+ \wedge Y_+$ und $*_+ = S^0$.
  Für $V, W \in \Ob(\Cat_*)$ definiere $\IHom_{\Cat_*}(V, W)$ als Pullback
  \begin{centertikzcd}
    \IHom_{\Cat_*}(V, W) \arrow[r] \arrow[d] & \IHom_\Cat(V, W) \arrow[d] \\
    \IHom_\Cat(*, *) \arrow[r] & \IHom_\Cat(*, W)
  \end{centertikzcd}
  Dann ist $\IHom_{\Cat_*}(X, \blank)$ rechtsadjungiert zu $\blank \wedge X$ für alle $X \in \Ob(\Cat_*)$ und damit $\Cat_*$ sogar monoidal abgeschlossen. \\
  Trage $\Cat$ zusätzlich eine Modellstruktur, sodass $\times : \Cat \times \Cat \to \Cat$ ein Quillenfunktor und $*$ kofasernd ist (also $(\Cat, \times, *)$ eine monoidale Modellkategorie ist).
  Dann erzeugt $U : \Cat_* \to \Cat$ eine symmetrische monoidale Modellstruktur auf $(\Cat_*, \wedge, S^0)$ und $(\blank)_+ \ladj U$ ist eine Quillenadjunktion, sogar eine monoidale:
\end{bsp}

\begin{defn}
  Seien $\Cat$, $\Dat$ monoidale Modellkategorien. \\
  Eine Quillen-Adjunktion $F : \adj{\Cat}{\Dat} : U$ heißt \emph{monoidal}, falls
  \begin{itemize}
    \item $F$ monoidal ist und
    \item $FQ \UnitOb_\Cat \xrightarrow{Fq} F \UnitOb_\Cat$ eine schwache Äquivalenz ist.
  \end{itemize}
\end{defn}

\begin{defn}
  Sei $\Cat$ eine mon. Modellkat.
  Eine \emph{$\Cat$-Modellkategorie} ist eine Modellkat. $\Dat$ mit Struktur $\otimes : \Dat \times \Cat \to \Dat$ als $\Cat$-Rechtsmodul, sodass
  \begin{itemize}
    \item $\otimes : \Dat \times \Cat \to \Dat$ ist eine Quillen-Biadjunktion,
    \item $X \otimes Q \UnitOb \xrightarrow{\id_X \otimes q} X \otimes \UnitOb$ ist eine schw. Äq. für alle kof. $X \!\in\! \Ob(\Dat)$.
  \end{itemize}
\end{defn}

\begin{bem}
  Wenn $\Cat$ punktiert ist, so auch $\Dat$.
\end{bem}

\begin{prop}
  Sei $(\Cat, {\times}, *)$ eine monoidale Modellkategorie und $*$ kofasernd.
  Ist dann $\Dat$ eine $\Cat$-Modellkategorie, so ist $\Dat_*$ eine $\Cat_*$-Modellkategorie.
  Damit gibt es eine Äquivalenz
  \[ \{\, \text{punktierte $\Cat$-Modellkategorie} \,\} \longleftrightarrow \{\, \text{$\Cat_*$-Modellkategorien} \,\}. \]
\end{prop}

% Die Homotopiekategorie monoidaler Modellkatgorien

\begin{prop}
  Seien $\Cat$, $\Dat$, $\Eat$ Modellkategorien und $\otimes : \Cat \times \Dat \to \Eat$ eine Quillen-Biadjunktion.
  Dann ist $\otimes^\LD{\,} : \Ho(\Cat) \times \Ho(\Dat) \to \Ho(\Eat)$ eine Biadjunktion mit Adjungierten $\RD{\Hom_r}$ und $\RD{\Hom_l}$.
\end{prop}

\iffalse
\begin{bem}
  Sei $\otimes : \Cat \times \Dat \to \Eat$ ein Quillen-Bifunktor.
  Ist dann $C \times I$ ein Zylinderobjekt zu $C \in \Ob(\Cat)$, so ist $(C \times I) \otimes D$ ein Zylinderobjekt zu $C \otimes D$, $D \in \Ob(\Dat)$.
\end{bem}
\fi

\begin{satz}
  Ist $\Cat$ eine (symm.) monoidale Modellkategorie, so ist $\Ho(\Cat)$ eine monoidal abgeschlossene Kategorie.
\end{satz}

% §7. Simpliziale Mengen
\section{Simpliziale Mengen}

\begin{referenz}
  Die \spickerref{homoalg}{Homologische-Algebra-Zusammenfassung} enthält eine Einführung in simpliziale Mengen.
\end{referenz}

% XXXX: In HomoAlg-Zusammenfassung integrieren

\begin{bspe}
  \begin{itemize}
    \item $I \coloneqq \Delta[1]$ heißt \emph{Intervall},
    \item $\Delta^i[n] \coloneqq \Set{x \!\in\! \Delta[n]}{i \!\not\in\! \im(x)} \!\subset\! \Delta[n]$ heißt \emph{$i$-Seite},
    \item $S^n \coloneqq \cup_{i=0}^n \Delta^i[n]$ heißt \emph{$n$-Sphäre}.
    \item $\Lambda^i[n] \coloneqq \cup_{j \neq i} \Delta^j[n]$ heißt \emph{$i$-Horn}.
  \end{itemize}
\end{bspe}

\begin{defn}
  Ein Morphismus $p : E \to X$ simplizialer Mengen heißt \emph{Kan-Faserung}, falls $\Set{\Lambda^i[n] \hookrightarrow \Delta[n]}{0 \leq i \leq n} \lhhe p$
\end{defn}

\begin{defn}
  Eine simpl. Menge heißt \emph{Kan-Komplex}, falls $X \to * \coloneqq \Delta[0]$ eine Kan-Faserung ist.
\end{defn}

\begin{defn}
  \begin{itemize}
    \item Ein \emph{inneres Horn} ist ein $\Lambda^i[n] \subset \Delta[n]$ mit $0 \!<\! i \!<\! n$.
    \item Eine simpl. Menge $X$ heißt \emph{innerer Kan-Komplex}, falls
    \[ \Set{\Lambda^i[n] \hookrightarrow \Delta[n]}{0 < i < n} \lhhe (X \to *). \]
  \end{itemize}
\end{defn}

\begin{bem}
  Es ist $X$ also genau dann ein (innerer) Kan-Komplex, wenn man (innere) Hörner in $X$ füllen kann.
\end{bem}

\begin{defn}
  Seien $X \in \sSet$, $x, y \in X_0$, \dh{} $x, y : \Delta[0] \to X_0$.
  Setze
  \[ x \sim y \coloniff \ex{\alpha : I \to X} \alpha(0)  = x \wedge \alpha(1) = y. \]
  mit $\alpha(\epsilon) \coloneqq \alpha \circ (\Lambda^\epsilon[1] \hookrightarrow I)$ für $\epsilon = 0, 1$. Setze $\pi_0(X) \coloneqq X/{\sim}$.
\end{defn}

\begin{prop}
  Ist $X$ ein Kan-Komplex, so ist $\sim$ eine Äq'relation.
\end{prop}

% Vorlesung vom 17.6.2015

\begin{defn}
  Eine \emph{anodyne Erweiterung} ist ein Morphismus $i : A \to B$ von simpl. Mengen, welcher die LHHE bzgl. aller Kan-Faserungen hat, \dh{} die Unterkategorie der anodynen Erweiterungen ist die Saturierung von $\{\, \Lambda^i[n] \to \Delta[n] \,\}$, also $\Coff(\{\, \Lambda^i[n] \to \Delta[n] \,\})$.
\end{defn}

\begin{satz}
  Die Monomorphismen in $\sSet$ sind genau die Retrakte von Zellkomplexen über $\{\, \partial \Delta[n] \hookrightarrow \Delta[n] \,\}$.
\end{satz}

\begin{defn}
  Eine \emph{triviale Faserung} ist ein Mor. in $\sSet$, welcher die RHHE bzgl. $\{\, \partial \Delta[n] \hookrightarrow \Delta[n] \,\}$, \dh{} bzgl. allen Monomor. hat.
\end{defn}

\begin{satz}
  $(\text{anodyne Erweiterungen}, \text{Kan-Faserungen})$ und $(\text{Monomorphismen}, \text{triviale Faserungen})$ sind jeweils schwache Faktorisierungssysteme von $\sSet$.
\end{satz}

\begin{satz}[\emph{Gabriel-Zisman}]
  Sei $k : Y \to Z$ ein Monomorphismus. \\
  Ist dann $i : A \to B$ anodyn, so ist $i \square k : A \!\times\! Z \cup_{A \times Y} B \!\times\! Y \to B \!\times\! Z$ (mit ${\otimes} \coloneqq {\times}$) ebenfalls anodyn.
\end{satz}

\begin{bem}
  Damit wird folgen, dass $\sSet$ eine kartesisch abgeschlossene Modellkategorie wird (\dh{} $\times$ ist ein Quillen-Bifunktor).
\end{bem}

% Vorbereitungen:
% $\mathcal{A} \coloneqq \{\, \text{anodyne Erweiterungen}$ \,\}
% $\mathcal{B} \coloneqq \Coff(\{\, \Lambda^\epsilon[1] \times \Delta[n] \cup_{\Lambda^\epsilon[1] \times \partial \Delta[n]} \Delta[1] \partial \Delta[n] \xrightarrow{\square} \Delta[1] \times \Delta[n] \,\}) ($\epsilon \in \{0,1\}$; Erzeuger: Inklusion eines Hohlprismas ohne Deckel/Boden in das Vollprisma
% $\mathcal{B} \coloneqq \Coff(\{\, \Lambda^\epsilon[1] \times Y \cup_{\Lambda^\epsilon[1] \times \partial X} \Delta[1] \partial X \xrightarrow{\square} \Delta[1] \otimes Y | X \hookrightarrow Y \,\}) ($\epsilon \in \{0,1\}$; Inklusion eines verallgemeinerten Hohlprismas ohne Deckel/Boden in das Vollprisma

%\begin{lem}
%  Es gilt $\mathcal{B} \subseteq \mathcal{A}$ und $\mathcal{A} = \mathcal{C}$.
%\end{lem}

\begin{defn}
  Seien $X$, $Y$ simpliziale Mengen. Dann ist der \emph{Funktionenkomplex} $Y^X \in \Ob(\sSet)$ definiert durch
  \[ (Y^X)_n \coloneqq \Hom_\sSet(\Delta[n] \times X, Y) \]
\end{defn}

\begin{bem}
  %Es gilt $\Hom(Z, Y^X) = \Hom(\colim_{Z_n} \Delta[n], Y^X) = \lim_{Z_n} \Hom(\Delta[n], Y^X) = \lim_{Z_n} \Hom_\sSet(\Delta[n] \times X, Y) = \Hom(\colim_{Z_n} (\Delta[n] \times X), Y) = \Hom((\colim_{Z_n} \Delta[n]) \times X, Y) = \Hom(Z \times X, Y)$
  Es gilt $\Hom(Z, Y^X) \cong \Hom(Z \times X, Y)$.
\end{bem}

\begin{kor}
  Ist $Y$ ein Kan-Komplex, so ist $Y^X$ wieder ein Kan-Komplex.
\end{kor}

\begin{defn}
  Zwei Morphismen $f, g : X \to Y$ zwischen simpl. Mengen $X$, $Y$ heißen \emph{homotop}, falls $f \sim g$ in $Y^X$, \dh{} die Menge der Homotopieklassen von Morphismen ist $\pi_0(Y^X)$.
\end{defn}

\begin{kor}[\emph{Homotopieerweiterungseigenschaft, HEE}]\mbox{}\\
  Sei $p : E \to X$ eine Kan-Faserung und $i : Y \to Z$ ein Monomorphismus.
  Im kommutativen Diagramm
  \begin{centertikzcd}
    Y \times I \arrow[r, "h'"] \arrow[d, hook, swap, "i \times \id_I"] & E \arrow[d, "p"] \\
    Z \times I \arrow[r, "h"] \arrow[ru, dashed, "\overline{h}"] & X \\
    Z \times \Lambda^\epsilon[1] \arrow[ru] \arrow[u] \arrow[uur]
  \end{centertikzcd}
  existiert der gestrichelte Pfeil.
\end{kor}

\begin{defn}
  Ein Monomorphismus $i : A \to B$ in $\sSet$ heißt \emph{starker Deformationsretrakt} (SDR), falls ein $r : B \to A$ mit $ri = \id_A$ und $[ir] = [\id_B] \in \pi_0(B^B \text{ in } A/\sSet)$, \dh{} es existiert $h : B \times I \to B$ mit $h_0 = \id_B$, $h_1 = ir$, $h|_{A \times I} = \id_{A \times I}$ oder ein Zickzack solcher $h$'s.
\end{defn}

\begin{bspe}
  $\Lambda^0[1], \Lambda^1[1] \subset \Delta[1]$ sind starke Deformationsretrakte.
\end{bspe}

\begin{prop}
  Sei $i : A \to B$ anodyn, $A$, $B$ Kan-Komplexe. \\
  Dann ist $A$ ein SDR von $B$ vermöge $i$.
\end{prop}

% Vorlesung vom 24.6.2015

\begin{prop}
  Sei $i : A \to B$ ein Monomorphismus, sodass $A$ ein SDR von $B$ ist.
  Dann ist $i$ anodyn.
\end{prop}

\begin{prop}
  Für eine Kan-Faserung $p : E \to X$ sind äquivalent:
  \begin{itemize}
    \item $p$ ist trivial.
    \item Es existiert ein Schnitt $s : X \to E$ und ein $h : E \times \Delta[1] \to E$ mit $ps = \id_X$ und $h : \id_E \sim sp$ (mod $X$)
    % "Das ist sowas wie ein starker Deformationsretrakt in der dualen Kategorie"
    \item $p$ ist eine Homotopieäquivalenz, \dh{} es existiert ein $s : X \to E$ mit Homotopien $k : ps \sim \id_X$ und $h : sp \sim \id_E$.
  \end{itemize}
\end{prop}

\begin{prop}
  Sei $p : E \to X$ eine Faserung, $i : A \to X$ ein SDR.
  Dann ist $p^{-1}(A) \to E$ im Diagramm
  \begin{centertikzcd}
    p^{-1}(A) \arrow[r] \arrow[d] \arrow[dr, phantom, "\ulcorner", very near start] &
    E \arrow[d] \\
    A \arrow[r] &
    X
  \end{centertikzcd}
  ebenfalls ein SDR.
\end{prop}

\begin{kor}
  Sei $p : E \to X \times \Delta[1]$ eine Faserung.
  Sei $p_0 \coloneqq p|_{p^{-1}(X \times \Lambda^0[1])}$, $p_1 \coloneqq p|_{p^{-1}(X \times \Lambda^1[1])} : E_i \to X$
  Dann sind $p_0, p_1 : E_i \to X$ faserweise homotopieäquivalent, \dh{} es existieren $f$, $g$ in
  \begin{centertikzcd}
    E_0 \arrow[r, "f"] \arrow[rd, "p_0"] &
    E_1 \arrow[r, "g"] \arrow[d, "p_1"] &
    E_0 \arrow[ld, "p_0"] \\
    & X
  \end{centertikzcd}
  mit $\id_{E_0} \sim gf$ (mod $X$) und $fg \sim \id_{E_1}$ (mod $X$)
\end{kor}

\begin{kor}
  Sei $p : E \to Y$ eine Faserung und $f, g : X \to Y$ homotop.
  Dann sind die Pullbacks $f^* E$ und $g^* E$ faserweise homotop (also mod $Y$).
\end{kor}

\begin{kor}
  Sei $X$ zshgd (\dh{} $\pi_0(X) = *$), $p : E \to X$ eine Faserung.
  Dann sind je zwei Fasern von $p$ homotopieäquivalent.
\end{kor}

% Minimale Komplexe

\begin{defn}
  \begin{itemize}
    \item Seien $x, y : \Delta[n] \to X$ zwei $n$-Simplizes in einer simplizialen Menge $X$ mit $x|_{\partial \Delta[n]} = a = x|_{\partial \Delta[n]}$.
    Schreibe $x \sim y$ ($\partial \Delta[n]$), falls $\exists \, h : \Delta[n] \times \Delta[1] \to X$ mit $h|_{\Lambda^0[1]} = x$, $h_{\Lambda^1[1]} = y$, $h_{\partial \Delta[n]} = a$.
    \item Ein Kan-Komplex $X$ heißt \emph{minimal}, falls
    \[ x \sim y \text{ ($\partial \Delta[n]$)} \enspace\iff\enspace x = y. \]
    \TODO{Definition vervollständigen}
  \end{itemize}
\end{defn}

\begin{lem}
  Sei $X$ ein Kan-Komplex. Dann existiert ein SDR $X'$ von $X$, sodass $X'$ minimal ist.
\end{lem}

\begin{lem}
  Sei $X$ minimal und $f : X \to X$ mit $f \sim \id_X$. Dann ist $f$ ein Isomorphismus.
\end{lem}

\begin{kor}
  Sei $f : X \to Y$ eine Homotopieäquivalenz zwischen minimalen Komplexen.
  Dann ist $f$ schon ein Isomorphismus.
\end{kor}

\begin{defn}
  Eine Faserung $p : E \to X$ heißt \emph{minimal}, falls für alle Simplizes $x, y : \Hom(\Delta[n], E)$ mit $p \circ x = p \circ y$ mit $x \sim y$ (mod $\partial \Delta[n]$).
\end{defn}

\begin{defn}
  Ein \emph{Bündel} mit Faser $F$ ist eine Abb. $p : E \to B$, sodass für alle Simplizes $\sigma : \Delta[n] \to B$ der Pullback $\Delta[n] \times_B E$ homotopieäquivalent zu $\Delta[n] \times F$ ist.
\end{defn}

\begin{lem}
  Eine minimale Faserung $p : E \to X$ ist ein Bündel.
\end{lem}

\begin{defn}
  Ein Morphismus $f : X \to Y$ in $\sSet$ heißt \emph{schwache Äquivalenz}, falls für alle Kan-Komplexe $K$ die ind. Abb. $[f, K] : [Y, K] \to [X, K]$ (mit $[Y, K] \coloneqq \pi_0(K^Y)$) bij. ist.
\end{defn}

\begin{bspe}
  \begin{itemize}
    \item Homotopieäquivalenzen ist eine schwache Äquivalenzen.
    \item Triviale Faserungen sind schwache Äquivalenzen.
    \item Sei $i : A \to B$ eine anodyne Erweiterung.
    Für jeden Kan-Komplex $K$ ist dann $K^i : K^B \to K^A$ eine triviale Faserung, insbesondere also eine schwache Homotopieäquivalenz.
    \item Ist $f : X \to Y$ eine schwache Äquivalenz zwischen Kan-Komplexen $X$, $Y$, so ist $f$ eine Homotopieäquivalenz.
  \end{itemize}
\end{bspe}

\begin{bem}
  $f$ ist genau dann eine schwache Äquivalenz, wenn in allen Diagrammen der Form
  \begin{centertikzcd}
    X \arrow[r] \arrow[d, "f"] &
    \overline{X} \arrow[d, "\overline{f}"] \\
    Y \arrow[r] &
    \overline{Y}
  \end{centertikzcd}
  mit $\overline{X}$, $\overline{Y}$ Kankomplexe, $X \to \overline{X}$, $Y \to \overline{Y}$ anodyn der Morphismus $\overline{f}$ eine schwache Äquivalenz ist.
\end{bem}

\begin{lem}
  Sei $i : A \to B$ anodyn, $p : E \to A$ ein Bündel.
  Dann existiert ein Pullback-Diagramm
  \begin{centertikzcd}
    E \arrow[d, "p"] \arrow[r, hook] &
    E' \arrow[d, "p'"] \\
    A \arrow[r, hook] &
    B
  \end{centertikzcd}
  mit einem Bündel $p'$.
  Weiter ist $p'$ eindeutig bis auf Isomorphie.
  Außerdem ist $E \hookrightarrow E'$ anodyn.
\end{lem}

\begin{prop}
  \begin{itemize}
    \item Eine Faserung $p : E \to X$ ist genau dann trivial, wenn $p$ eine schwache Äquivalenz ist.
    \item Eine Kofaserung $i : A \to B$ ist genau dann anodyn, wenn $i$ eine schwache Äquivalenz ist.
  \end{itemize}
\end{prop}

\begin{lem}
  Pullbacks längs Bündeln erhalten schwache Äquivalenzen.
\end{lem}

\begin{satz}[Quillen]
  Die Kategorie der simplizialen Mengen mit den schwachen Äquivalenzen wird zu einer kartesisch abgeschlossenen, eigentlichen, kombinatorischen Modellkategorie, wenn als Kofaserungen die Monomorphismen und als Faserungen die Kan-Faserungen gewählt werden.
\end{satz}

% §6. Kettenkomplexe
\subsection{Kettenkomplexe}

Sei $R$ ein Ring und $\RMod{R}$ die Kategorie der $R$-Moduln.

\begin{defn}
  Ein $P \in \Ob(\RMod{R})$ heißt \emph{projektiv}, falls $\Hom_r(P, f) : \Hom(P, M) \to \Hom(P, N)$ für alle surjektiven $f : M \to N$ surjektiv ist.
\end{defn}

\begin{bem}
  Eine Abb. $f : M \to N$ ist genau dann surjektiv, wenn $\Hom(P, f)$ surjektiv für alle projektiven $P$.
\end{bem}

\begin{bspe}
  \begin{itemize}
    \item $R$ ist projektiv
    \item Direkte Summen projektiver Moduln sind projektiv.
    \item Retrakte projektiver Moduln sind wieder projektiv.
  \end{itemize}
\end{bspe}

\begin{bem}
  Damit hat $\RMod{R}$ genügend viele projektive, \dh{} jedes Modul $M$ erlaubt eine Surjektion $P \to M$ mit $P$ projektiv.
\end{bem}

\begin{defn}
  Sei $\Ch_*(R)$ die Kategorie der Kettenkomplexe, die in nichtnegativen Graden konzentriert sind, \dh{} Objekte sind Diagramme der Form
  \[
    \ldots \to C_3 \xrightarrow{\partial_3} C_2 \xrightarrow{\partial_2} C_1 \xrightarrow{\partial_1} C_0
    \quad \text{mit $\partial_i \circ \partial_{i+1} = 0$.}
  \]
  Elemente in $Z_n C \coloneqq \Set{x \in C_n}{\partial_n(x) = 0}$ heißen \emph{$n$-Zykel} und Elemente in $B_n C \coloneqq \Set{\partial_n(y)}{y \in C_{n+1}} \subseteq Z_n C$ heißen \emph{$n$-Ränder}.
  Die Gruppe $H_n(C) \coloneqq Z_n C / B_n C$ heißt \emph{$n$-te Homologie}.
  Ist $f : C \to C'$ ein Morphismus von Kettenkomplexen, so induziert dieser einen Morphismus $H_n(f) : H_n(C) \to H_n(C')$.
  Es heißt $f$ ein Homologie-Isomorphismus (oder Quasi-Isomorphismus), falls $H_n(f)$ für alle $n \geq 0$ ein Isomorphismus ist.
\end{defn}

\begin{satz}
  Zusammen mit den Homologie-Isomorphismen als schwache Äquivalenzen wird $\Ch_*(R)$ zu einer komb. Modellkategorie, wenn wir als Kofaserungen gradweise Monomorphismen mit gradweise projektivem Kokern und als Faserungen Morphismen, die im positiven Grad gradweise Surjektionen sind, wählen.
\end{satz}

\begin{defn}
  Diese Modellstr. heißt \emph{proj. Modellstruktur} auf $\Ch_*(R)$.
\end{defn}

\begin{bem}
  Alle Komplexe in die projektive Modellstruktur sind fasernd, insbesondere ist $\Ch_*(R)$ rechtseigentlich.
  Die kofasernden Objekte sind die Komplexe projektiver Moduln.
  Projektive Auflösung entspricht dem kofasernden Ersatz.
\end{bem}

% Aufgabe:
% 1) Ist $\Ch_*(R)$ links-eigentlich?
% 2) Ist $(\Ch_*(R), \otimes_R)$ symmetrisch monoidal?

\begin{defn}
  Es sei $D(n)$ der Komplex $\ldots \to R \xrightarrow{\id} R \to 0 \to \ldots$ mit $R$ im Grad $n$ und $n-1$ und
  $S(n)$ der Komplex $\ldots \to 0 \to R \to 0 \to \ldots$ mit $R$ im Grad $n$.
\end{defn}

\begin{defn}
  Der \emph{Einhängungsfunktor} ist
  \[
    \Sigma : \Ch_*(R) \to \Ch_*(R), \quad
    (\Sigma C)_{n+1} \coloneqq C_n, \enspace (\Sigma C)_0 \coloneqq 0.
  \]
\end{defn}

\begin{bem}
  Es gilt $D(n+1) = \Sigma(D(n))$ und $S(n+1) = \Sigma(S(n))$.
\end{bem}

\begin{bem}
  Es gilt $\Hom(D(n), C) \cong C_n$ und $\Hom(S(n), C) \cong Z_n C$.
\end{bem}

\begin{lem}
  Ein Morphismus $f : C \to C'$ ist genau dann eine Faserung (\dh{} im positiven Grad surjektiv), wenn $f$ die RHHE bzgl. aller $\Set{0 \to D(n)}{n \geq 1}$ hat.
\end{lem}

\begin{bem}
  Folglich ist $\Set{0 \to D(n)}{n \geq 1}$ die Menge der erz. azykl. Kof. der projektiven Modellstruktur.
\end{bem}

\begin{lem}
  Sei $f : C \to C'$ ein Mor. in $\Ch_*(R)$.
  Dann sind äquivalent:
  \begin{itemize}
    \item $f$ ist ein Homologie-Isomorphismus und in positiven Graden surjektiv
    \item Für alle $n \geq 0$ ist $C_n \to Z_{n-1} C \times_{Z_{n-1} C'} C'_n, x \mapsto (\partial x, f(x))$ surjektiv.
    \item $f$ hat die RHE bzgl. $\Set{S(n) \to D(n)}{n \geq 0}$
  \end{itemize}
\end{lem}

\begin{bem}
  Folglich ist $\Set{S(n) \to D(n)}{n \geq 0}$ die Menge der erzeugenden Kofaserungen der projektiven Modellstruktur.
\end{bem}

% Vorlesung vom 8.7.2015

\subsection{Die Homotopiekategorie als $\Ho(\sSet)$-Modul}

\begin{defn}
  Die \emph{Kategorie der Räume} ist $\Simpl \coloneqq \Ho(\sSet)$.
\end{defn}

\begin{ziel}
  Zeige, dass $\Ho(\ModC)$ für jede Modellkategorie $\ModC$ in natürlicher Weise ein Modul über $\Simpl$ ist, \dh{} wir wollen Funktoren
  \begin{align*}
    \Ho(\ModC) \times \Simpl \to \Ho(\ModC), \quad & (X, K) \mapsto X \otimes^\LL K, \\
    \Ho(\ModC)^\op \times \Ho(\ModC) \to \Simpl, \quad & (X, Y) \mapsto \Map(X, Y) =: Y^X, \\
    \Simpl^\op \times \Ho(\ModC) \to \Ho(\ModC), \quad & (K, X) \mapsto \RD{\Hom(K, X)} := X^K,
  \end{align*}
  sodass gilt:
  \[ \Hom_{\Ho(\ModC)}(X \otimes^\LL K, Y) \cong \Hom_\Simpl(K, Y^X) \cong \Hom_{\Ho(\ModC)}(X, Y^K). \]
  Weiter gilt: $X \otimes^\LL \Delta[0] \cong X$, \dh{} $\Hom_{\Ho(\ModC)}(X, Y) \cong \Hom_{\Ho(\ModC)}(X \times \Delta[0], Y) \cong \Hom_\Simpl(\Delta[0], Y^X) = \pi_0(Y^X)$.
\end{ziel}

% Kategorien von Diagrammen

\begin{defn}
  Sei $\Bat$ eine kleine Kategorie und $\lambda$ eine Ordinalzahl.
  Ein Funktor $f : \Bat \to \lambda$ heißt \emph{lineare Einbettung}, falls:
  \[ \fa{u : b \to c} F(u) = \id \implies u = \id. \]
  In diesem Fall heißt $F(i)$ für $i \in \Ob(\Bat)$ der \emph{Grad von $i$}. \\
  Eine Kategorie mit einer linearen Erweiterung heißt \emph{gerichtet}. \\
  Ist $\Bat^\op$ gerichtet, so heißt $\Bat$ inverse Kategorie.
\end{defn}

\begin{defn}
  \begin{itemize}
    \item Sei $\Cat$ kovollständig, $\Bat$ gerichtet, $i \in \Ob(\Bat)$. \\
    Das $i$-te \emph{Latching-Objekt} $L_i X$ eines Funktors $X \in \Ob(\Cat^\Bat)$ ist $L_i X \coloneqq \colim_{j \xrightarrow{{+}} i} X_j$.
    Die Indexkategorie ist dabei die Scheibenkategorie der Objekte vom Grad $< i$ über $X_i$.
    \item Sei $\Bat$ invers und $\Cat$ vollständig. Das $i$-te \emph{Matching-Objekt} von $X$ ist
    $M_i X \coloneqq \lim_{i \xrightarrow{{-}} j} X_j$.
  \end{itemize}
\end{defn}

\begin{bem}
  Es gibt nat. Transformationen $L_i X \to X_i$ und $X_i \to M_i X$.
\end{bem}

\begin{satz}
  Sei $\ModC$ eine Modellkategorie, $\Bat$ gerichtet.
  Dann gibt es folgende Modellstruktur auf $\ModC^\Bat$: Ein Funktor $\tau : X \to Y$ ist eine
  \begin{itemize}
    \item schw. Äq. $\iff$ $\fa{i \in \Ob(\Bat)} \tau_i : X_i \to Y_i$ ist schw. Äq,
    \item Faserung $\iff$ $\fa{i \in \Ob(\Bat)} \tau_i : X_i \to Y_i$ ist Faserung,
    \item Kof. $\iff$ $\fa{i \in \Ob(\Bat)} X_i \cup_{L_i X} L_i Y \to Y_i$ ist eine Kof,
    \item triv. Kof. $\iff$ $\fa{i \!\in\! \Ob(\Bat)\!}\! X_i \cup_{L_i X} L_i Y \to Y_i$ ist eine triv. Kof.
  \end{itemize}
  Bzgl. dieser Modellstr. ist $\colim : \Cat^\Bat \to \Cat$ ein Linksquillenfunktor.
\end{satz}

\begin{bem}
  \begin{itemize}
    \item Dual: $\ModC^{\Bat^\op} \simeq (\ModC^\op)^\Bat$
    \item Ist $\tau$ eine Kofaserung in $\ModC^\Bat$ und $\Bat$ gerichtet, so ist insbesondere $\fa{i} \tau_i : X_i \to Y_i$ eine Kofaserung.
  \end{itemize}
\end{bem}

\begin{beweis}
  Da $\colim$ ein Linksquillenfunktor ist, ist $L_i X \to L_i Y$ eine Kofaserung.
  Damit ist auch $X_i \to X_i \cup_{L_i X} L_i Y$ als Pushout von $L_i X \to L_i Y$ eine Kofaserung.
  Somit ist auch die Komposition mit der Kofaserung $X_i \cup_{L_i X} L_i Y \to Y_i$ wieder eine Kofaserung.
\end{beweis}

% Diagramme über Reedy-Kategorien

\begin{defn}
  Eine \emph{Reedy-Kategorie} ist eine Kategorie $\Bat$ zusammen mit Unterkategorien $\Bat_{+}$ und $\Bat_{-}$ und einem Funktor $d : \Bat \to \lambda$, der Gradfunktion, wobei $\lambda$ eine Ordinalzahl ist, sodass folgendes gilt:
  \begin{itemize}
    \miniitem{0.48 \linewidth}{$\Bat_{+}$ ist bzgl. $d$ gerichtet,}
    \miniitem{0.48 \linewidth}{$\Bat_{-}$ ist bzgl. $d$ invers,}
    \item Jeder Morphismus $f \in \Mor(\Bat)$ lässt sich eindeutig faktorisieren als $f = ip$ mit $i \in \Bat_{+}$ und $p \in \Bat_{-}$.
  \end{itemize}
\end{defn}

\begin{bsp}
  $\Delta$ ist Reedy mit $\Delta_{+} \coloneqq \{\text{ injektive }\}$ und $\Delta_{-} \coloneqq \{\text{ surjektive }\}$.
\end{bsp}

\begin{bspe}
  \begin{itemize}
    \item Sei $A^\bullet$ ein kosimpliziales Objekt in $\Cat$, \dh{} $A^\bullet \in \Ob(\Cat^\Delta)$.
    Wir schreiben $A^\bullet [n] \coloneqq A^\bullet_{[n]}$. Dann
    \[
      L_0 A^\bullet \!=\! \emptyset, \quad
      L_1 A^\bullet \!=\! A^\bullet[0] \amalg A^\bullet [0], \quad
      M_0 A^\bullet \!=\! *, \quad
      M_1 A^\bullet \!=\! A^\bullet [0].
    \]
    \item Sei $X_\bullet$ ein simpl. Objekt in $\Cat$, \dh{} $X \in \Ob(\Cat^{\Delta^\op})$. Dann
    \[
      L_0 X_\bullet = \emptyset, \quad
      L_1 A_\bullet = X_0, \quad
      M_0 X_\bullet = *, \quad
      M_1 X = X_0 \times X_0.
    \]
  \end{itemize}
\end{bspe}

\begin{bem}
  Sei $\Bat$ eine Reedy-Kategorie.
  Dann können wir Funktoren $X \in \Ob(\Cat^\Bat)$ wie folgt rekursiv konstruieren:
  Sei dazu $d : \Bat \to \lambda$ die Gradfunktion und $X|_{\Bat_{< \beta}}$ schon definiert.
  Sei dann $i \in \Bat$ mit $d(i) = \beta$.
  Wähle eine Faktorisierung $L_i X \to X_i \to M_i X$. \\
  Sei $i \to j$ mit $d(i), d(j) \leq \beta$.
  Gesucht ist ein Morphismus $X_i \to X_j$. \\
  Ohne Einschränkung sei $(i \to j) \neq \id$.
  Faktorisiere $(i \to j) = (k \xrightarrow{{+}} j) \circ (i \xrightarrow{{-}} k)$.
  Gesucht ist also $X_i \to X_k$ und $X_k \to X_j$ für $i \xrightarrow{{-}} k$, $k \xrightarrow{{+}} j$.
  Dazu betrachte
  \[
    X_i \xrightarrow{\text{gewählt}} M_i X \xrightarrow{\text{kanonisch}} X_k, \quad
    X_k \xrightarrow{\text{kanonisch}} L_j X \xrightarrow{\text{gewählt}} X_j.
  \]
\end{bem}

\iffalse
\begin{bsp}
  Sei $\Bat = \Delta$ und $A \in \Cat$. Ziel: Konstruktion $X^\bullet \in \Cat^{\Delta}$.
\end{bsp}
\fi

\begin{defn}
  Sei $\Bat = \Delta$ und $A \in \Cat$.
  Dann sind $\ell^\bullet A, r^\bullet A \in \Ob(\Cat^\Delta)$ def. durch
  \[
    \ell^\bullet A [i] = A \times [i] \coloneqq A \amalg \ldots \amalg A, \quad
    r^\bullet A [i] \coloneqq A.
  \]
\end{defn}

\begin{bem}
  $\ell^\bullet \ladj (\text{Ev}^\bullet : X^\bullet \to X^\bullet [0]) \ladj r^\bullet$
\end{bem}

\begin{satz}
  Sei $\ModC$ eine Modellkategorie und $\Bat$ eine Reedy-Kategorie.
  Dann existiert folgende Modellstruktur auf $\ModC^\Bat$: \\
  Eine natürliche Transformation $f : X \to Y$ ist eine
  \begin{itemize}
    \item schw. Äq. $\iff$ $\fa{i \in \Ob(\Bat)} f_i : X_i \to Y_i$ ist schw. Äq.
    \item (triv.) Kof. $\iff$ $\fa{i \in \Ob(\Bat)} X_i \cup_{L_i X} L_i Y \to Y_i$ ist (triv.) Kof.
    \item (triv.) Fas. $\iff$ $\fa{i \!\in\! \Ob(\Bat)\!}\! X_i \to Y_i \times_{M_i Y} M_i X$ ist (triv.) Fas.
  \end{itemize}
\end{satz}

\begin{bsp}
  $\ModC^\Delta$, $\ModC^{\Delta^\op}$ sind wieder Modellkategorien.
\end{bsp}

\begin{defn}
  Sei $\Cat$ eine Modellkategorie, $A \in \Ob(\Cat)$.
  \begin{itemize}
    \item Ein \emph{kosimplizialer Rahmen von $A$} ist eine Faktorisierung von $\ell^\bullet A \to r^\bullet A$ der Form $\ell^\bullet A \hookrightarrow A^* \xrightarrow{{\sim}} r^\bullet A$., sodass $A^* [0] = A$.
    \item Dual ist ein simpl. Rahmen eine Faktorisierung $l_\bullet A \xrightarrow{{\sim}} A_* \twoheadrightarrow r_\bullet A$.
  \end{itemize}
\end{defn}

\begin{bem}
  Ist $A^*$ ein kosimplizialer Rahmen von $A$, so ist $A^*[1]$ ein gutes Zylinderobjekt.
\end{bem}

\begin{bsp}
  Sei $\Cat$ eine simpliziale Modellkategorie.
  Ist dann $A$ kofasernd, so ist $A \otimes \Delta^\op$ mit $(A \otimes \Delta^\bullet)[n] = A \otimes (\Delta^\bullet [n])$ ein kosimplizialer Rahmen von $A$.
  Insbesondere ist $A \otimes \Delta^\bullet [1]$ ein gutes Zylinderobjekt.
\end{bsp}

\subsection{Die Wirkung von $\Simpl = \Ho(\sSet)$}

\begin{konstr}
  Sei $\ModC$ eine Modellkategorie, $K$ eine simpl. Menge, $X$ ein Objekt aus $\ModC$.
  Wir wollen $X \otimes^{\LL} K \in \Ob(\Ho(\ModC))$ definieren:
  \begin{enumerate}[label=\alph*),leftmargin=2em]
    \item Wähle kofasernden Ersatz $QX \xtwoheadrightarrow{{\sim}} X$
    \item Wähle kosimplizialen Rahmen $\ell^\bullet Q X \hookrightarrow X^* \xrightarrow{{\sim}} r^\bullet Q X$
    \item Setze $X \otimes^{\LL} K \coloneqq \CoEndC{n}{X^*[n] \times K_n}$.
    \item Fasse $X \otimes^{\LL} K$ als Objekt in $\Ho(\ModC)$ auf.
  \end{enumerate}

  Sei $Y \in \Ob(\ModC)$.
  Wir wollen $Y^K \in \Ob(\Ho(\ModC))$ wie folgt definieren:
  \begin{itemize}
    \item Wähle fasernden Ersatz $Y \xhookrightarrow{{\sim}} RY$.
    \item Wähle simplizialen Rahmen $\ell_\bullet RY \to Y_* \to r_\bullet RY$.
    \item Setze $Y^K \coloneqq \RD \Hom(K, Y) = \EndC{n} Y_*[n]^{K_n}$.
    \item Fasse $Y^K$ als Objekt in $\Ho(\ModC)$ auf.
  \end{itemize}

  Wir definieren $Y^X \in \Ob(\Simpl)$ wie folgt:
  \begin{itemize}
    \item Wähle $QX \to X$ und $Y \to RY$ wie eben.
    \item Wähle $X^*$, $Y_*$ wie eben.
    \item Setze $Y^X \coloneqq \text{Map}(X, Y) = \text{diag} \Hom(X^*, Y_*) = \Hom(X^*, Y) = \Hom(X, Y_*)$.
    \item Fasse $Y^X$ als Objekt von $\Ho(\sSet) = \Simpl$ auf.
  \end{itemize}
\end{konstr}

\begin{satz}
  Diese Konstruktion macht $\Ho(\ModC)$ zu einem wohldefinierten, abgeschl. $\Simpl$-Modul.
  Sie ist mit Quillenadjunktionen verträglich. % XXX: wie genau?
\end{satz}

\begin{satz}
  Analog wird die Homotopiekategorie einer jeden punktierten Modellkategorie $\ModC$ in kanonischer Weise zu einem $\Simpl_*$-Modul, wobei $S_* \coloneqq \Ho(\sSet_*)$.
  Wir erhalten
  \begin{align*}
    \Ho(\ModC) \times \Simpl_* \to \Ho(\ModC), \quad & (X, K) \mapsto X \wedge^{\LL} K, \\
    \Ho(\ModC)^\op \times \Ho(\ModC) \to \Simpl_*, \quad & (X, Y) \mapsto \Map_*(X, Y), \\
    \Simpl_*^\op \times \Ho(\ModC) \to \Ho(\ModC), \quad & (K, Y) \mapsto \RR \Hom_*(K, Y).
  \end{align*}
  Es gilt dann $A \wedge^{\LL} K_+ = A \otimes^{\LL} K$.
\end{satz}

% Mittwochs-Vorlesung der letzten Vorlesungswoche

\subsection{Punktierte Modellkategorien}

\begin{defn}
  Sei $\Cat_*$ eine punktierte Kategorie und $f : X \to Y$ ein Morphismus in $\Cat_*$.
  Sei $0 : X \to * \to Y$ der eindeutig bestimmte Morphismus.
  \begin{itemize}
    \item Die \emph{Kofaser} $\cofib f$ von $f$ ist der Differenzkokern von $f, 0 : X \to Y$.
    \item Die \emph{Faser} $\fib f$ von $f$ ist der Differenzkern von $f, 0$.
  \end{itemize}
\end{defn}

\begin{bem}
  Folgende Diagramme sind Pushout- bzw. Pullbackdiagramm:
  \begin{centertikzcd}
    X \arrow[d] \arrow[r, "f"] \arrow[rd, phantom, "\ulcorner", very near end] &
    Y \arrow[d] &
    \fib f \arrow[d] \arrow[r] \arrow[rd, phantom, "\lrcorner", very near start] &
    X \arrow[d, "f"] \\
    * \arrow[r] &
    \cofib f &
    * \arrow[r] &
    Y
  \end{centertikzcd}
\end{bem}

\begin{defn}
  $\Sph^0 \coloneqq \Delta[0]_{+} \in \Ob(\sSet_*) = \Ob(\Simpl_*)$, \quad
  $\Sph^1 \coloneqq \Delta[1]_{+} / \partial \Delta[1]_{+}$
\end{defn}

\begin{defn}
  Sei $\Cat_*$ eine punktierte Modellkategorie.
  Dann heißt
  \begin{alignat*}{4}
    & \Sigma : \Ho(\Cat_*) \to \Ho(\Cat_*), \enspace X \mapsto X \wedge^{\LL} \Sph^1 &&
    \text{\enspace\emph{Einhängung} und} \\
    & \Omega : \Ho(\Cat_*) \to \Ho(\Cat_*), \enspace Y \mapsto \RR \Hom_*(\Sph^1, Y) &&
    \text{\enspace\emph{Verschleifung}.}
  \end{alignat*}
\end{defn}

\begin{bem}
  $\Sigma \ladj \Omega$
\end{bem}

\begin{konstr}
  Sei $X \in \Ob(\Ho(\Cat_*))$.
  Wähle einen kofasernden Repräsentanten $QX \in \Ob(\Cat_*)$ von $X$.
  Wähle dann einen komsimplizialen Rahmen $(QX)^o$.
  Dann ist
  \begin{alignat*}{2}
    & (QX)^o \wedge^{\LL} \Sph^1 \quad
    = \left( (QX)^o \wedge \Delta[1]_+ \right) / \left( (QX)^o \wedge \delta \Delta[1]_+ \right) \\
    = & \left( (QX)^o \otimes \Delta[1] \right) / \left( (QX)^o \otimes \partial \Delta[1] \right) \quad
    = \left( QX \times I \right) / (QX \vee QX) \\
    = & \cofib(QX \vee QX \to \Cyl{QX})
  \end{alignat*}
  für einen gutes Zylinderobjekt $\Cyl{QX}$ zu $QX$.
  Ist $X$ kofasernd, so gilt $X \wedge^\LL \Sph^1 = \cofib(X \vee X \to \Cyl{X})$.
  Dual gilt für ein kofaserndes $Y$: $\Omega Y = \fib(\PO{Y} \to Y)$ für ein gutes Wegeobjekt $\PO{Y}$.
\end{konstr}

\begin{defn}
  $\Sph^l \coloneqq \Sigma^l \Sph^0$ heißt \emph{$l$-Sphäre}.
\end{defn}

% TODO: Ist [X, Y] := basispunkterhaltende Abbildungen irgendwo definiert?

\begin{nota}
  $[X, Y] \coloneqq \Hom_{\Ho(\ModC)}(X, Y)$ für $X, Y \in \Ob(\ModC)$
\end{nota}

\begin{bem}
  $\pi_0 \Map_*(X, Y) \cong [\Sph^0, \Map_*(X, Y)]$, \quad
  $\pi_l \Map_*(X, Y) = [\Sigma^l X, Y]$
\end{bem}

\begin{bem}
  $\Sph^1$ ist eine folgendermaßen ein Kogruppenobjekt in $\Simpl_*$: \\
  Für $X \in \Ob(\sSet)$ sei $\tilde{X}$ definiert als Pushout
  \begin{centertikzcd}
    \sk_0 X \arrow[d] \arrow[r, hook] \arrow[rd, phantom, "\ulcorner", very near end] &
    X \arrow[d] \\
    \Delta[0] \arrow[r] &
    \tilde{X}
  \end{centertikzcd}
  Zum Beispiel ist $\widetilde{\Delta[1]} = \Sph^1$.
  Die Inklusionen $i : \Delta^1 [2] \to \Delta [2]$ und $s : \widetilde{\Lambda^1 [2]} \to \widetilde{\Delta [2]}$ induzieren Abbildungen $\tilde{i} : \Sph^1 = \widetilde{\Delta^1 [2]} \to \widetilde{\Delta [2]}$ und $\tilde{s} : \Sph^1 \vee \Sph^1 = \Lambda^1 [2] \to \Delta [2]$.
  Da $s$ eine anodyne Erweiterung ist, ist auch $\tilde{s}$ als Pushout von $s$ anodyn, also eine schwache Äquivalenz.
  Dann ist die Kogruppenkomultiplikation der Morphismus
  \[
    \Sph^1 \xrightarrow{\tilde{i}} \widetilde{\Delta [2]} \xrightarrow{\tilde{s}^{-1}} \Sph^1 \vee \Sph^1 \quad
    \text{in $\Simpl_*$.}
  \]
\end{bem}

\begin{satz}
  Sei $X \in \Ho(\Cat_*)$. Dann ist
  \begin{itemize}
    \item $\Sigma^l X$ kanonisch eine Kogruppe für $l \geq 1$ und abelsch für $l \geq 2$,
    \item $\Omega^l Y$ kanonisch eine Gruppe für $l \geq 1$ und abelsch für $l \geq 2$.
  \end{itemize}
\end{satz}

\begin{kor}
  $\pi_l X \coloneqq \pi_0 \RR \Hom_*(\Sph^l, X) = [\Sph^l, X]$ ist eine Gruppe für $l \geq 1$ und abelsch für $l \geq 2$.
\end{kor}

% Unterkapitel: Homotopie-Kofaser

\begin{bem}
  Die Index-Kategorie $I \coloneqq \{ \bullet \leftarrow \bullet \to \bullet \}$ von Pushoutdiagrammen ist gerichtet, \dh{} $\Cat_*^I$ trägt eine Modellstruktur.
  Bzgl. dieser ist $\colim : \Cat_*^I \to \Cat_*$ ein Linksquillenfunktor.
  Wir erhalten einen Funktor $\colim^\LL : \Ho(\Cat_*^I) \to \Ho(\Cat_*)$.
\end{bem}

\begin{defn}
  Die \emph{Homotopiekofaser} eines Morphismus $(A \xrightarrow{f} B) \in \Cat_*$ ist
  \[ \cofib^\LL(f) \coloneqq \colim^\LL(* \leftarrow A \xrightarrow{f} B). \]
\end{defn}

\begin{konstr}
  Wir haben das Diagramm $(* \leftarrow A \xrightarrow{f} B) \in \Ob(\Cat_*^I)$. \\
  Durch Faktorisierung finden wir $\left( Q* \hookleftarrow{} QA \hookrightarrow QB \right) \in \Ob(\Cat_*^I)$, wobei $Q*$, $QA$ und $QB$ kofasernde Ersatzobjekte für $*$, $A$, bzw. $B$ und die Morphismen Kofaserungen sind.
  Dieses Diagramm ist dann ein kof. Ersatz für das originale Diagramm in $\Cat_*^I$ und somit gilt
  \[
    \cofib^\LL(f) = QB \cup_{QA} Q*.
  \]
\end{konstr}

\begin{defn}
  Dual ist die \emph{Homotopiefaser} eines Mor. $(A \xrightarrow{f} B) \in \Cat_*$
  \[ {\fib}^\RR(f) \coloneqq {\lim}^\RR(* \to B \xleftarrow{f} A). \]
\end{defn}

\begin{bspe}
  \begin{itemize}
    \item $\Omega Y = \fib^\RR(X \xrightarrow{\Delta} X \times X)$
    \item $X \wedge^\LL \Sph^1 = \cofib(QX \vee QX \to \gutCyl{QX}) = \cofib^\LL(X \vee X \xrightarrow{\nabla} X)$
  \end{itemize}
\end{bspe}

\begin{defn}
  Die Folge $A \xrightarrow{f} B \xrightarrow{g} C = \cofib^\LL(f)$ heißt \emph{Kofasersequenz}
  und $X = \fib^\RR(h) \to Y \xrightarrow{h} Z$ heißt \emph{Fasersequenz}.
\end{defn}

\begin{bem}
  $\Sigma A$ kooperiert folgendermaßen auf $C = \cofib^\LL(f)$: \\
  Die Kooperationsabbildung $C \to C \vee \Sigma A$ entspricht nach dem Yoneda-Lemma einer Operation $[\Sigma A, X] \times [C, X] \to [C, X]$, welche natürlich in $X \in \Ob(\Ho(\Cat_*))$ ist.
  Sei zunächst $RX$ ein faserner Ersatz von $X$. Sei $j \in [\Sigma A, X]$ und $k \in [C, X]$. Dann ist die Wirkung der Morphismus $C \xrightarrow{\beta} RX \times RX \xrightarrow{\mathrm{pr}_1} RX$ in folgendem Diagramm:
  \begin{centertikzcd}[column sep=0.5cm, row sep=0.5cm]
    QA \arrow[d, hook, "Qf"] \arrow[r, twoheadrightarrow, "\sim"] &
    A \arrow[r, "j^*"] &
    \Omega X \arrow[r] &
    \PO{RX} \arrow[d, twoheadrightarrow] \arrow[dd, "\sim", twoheadrightarrow, rounded corners, to path={
      -- ([xshift=14ex]\tikztostart.east)
      -- ([xshift=14ex]\tikztotarget.east) \tikztonodes
      -- (\tikztotarget.east)
      }] \\ % controls={+(3,-0.5) and +(4,0.5)}] \\
    QB \arrow[d, hook] \arrow[rrru, dashed, "\exists \, \alpha"] &&&
    RX \times RX \arrow[d, twoheadrightarrow, "\mathrm{pr}_0"] \arrow[r, twoheadrightarrow, "\mathrm{pr}_1"] &
    RX \\
    C = \cofib^\LL(f) \arrow[rr, "k"] \arrow[rrru, dashed, "\exists ! \, \beta"] &&
    X \arrow[r, hook, "\sim"] &
    RX
  \end{centertikzcd}
  Beachte:
  \begin{itemize}
    \item Dabei ist $\PO{RX}$ ist ein sehr gutes Pfadobjekt.
    \item Die ganz rechte Faserung ist azyklisch nach der 2-aus-3-Eigenschaft.
    \item $\mathrm{pr}_0$ und $\mathrm{pr}_1$ sind Faserungen sind als Pullback von $RX \twoheadrightarrow *$.
    \item Der Morphismus $QB \hookrightarrow C$ ist als Pushout einer Kofaserung selbst eine Kofaserung (siehe Konstruktion der Homotopiekofaser).
  \end{itemize}
  \TODO{Wie genau sind jetzt $\alpha$ und $\beta$ definiert? Und mit welcher Eigenschaft ist dann $\beta$ eindeutig und warum? Wer dies hier liest, und mir diese Fragen beantworten kann, bekommt ein Ü-Ei geschenkt!}
\end{bem}

\begin{dual}
  Sei $X = \fib^\RR(h) \to Y \xrightarrow{h} Z$ eine Fasersequenz.
  Dann operiert die Gruppe $\Omega Z$ auf $X$.
  In einem Setting von topologischen Räumen ist die Wirkung folgendermaßen gegeben:
  Ein Element von $\Omega Z$ wird repräsentiert durch einen geschlossenen Weg $\gamma : I \to Z$ mit $\gamma(0) = \gamma(1) = z_0$.
  Sei $x$ ein Element der Homotopie-Faser von $h$.
  Dann ist $[\gamma].x = \tilde{\gamma}(1)$, wobei $\tilde{\gamma} : I \to Y$ ein Lift von $\gamma$ ist.
\end{dual}

\begin{satz}
  Das sind in der Tat (Ko-) Operationen.
\end{satz}

\begin{defn}
  Sei $A \xrightarrow{f} B \xrightarrow{g} C$ eine Kofasersequenz.
  Dann heißt
  \[ \partial : C \xrightarrow{\text{Koop}} C \vee \Sigma A \xrightarrow{(C \to *) \vee \id} * \vee \Sigma A = \Sigma A \]
  der \emph{Korandoperator} in der Kofasersequenz $A \xrightarrow{f} B \xrightarrow{g} C \xrightarrow{\partial} \Sigma A$.
\end{defn}

\begin{dual}
  Sei $X \xrightarrow{f} Y \xrightarrow{g} Z$ eine Fasersequenz.
  Dann heißt
  \[
    \partial : \Omega Z = \Omega Z \times *
    \xrightarrow{\Omega Z \times (* \to X)}
    \Omega Z \times X
    \xrightarrow{\text{Op}}
    X
  \]
  der \emph{Randoperator} in der Fasersequenz $\Omega Z \xrightarrow{\partial} X \xrightarrow{f} Y \xrightarrow{g} Z$.
\end{dual}

\begin{bsp}
  $* \to X \xrightarrow{\id} X \xrightarrow{\partial} * = \Sigma *$ ist eine Kofasersequenz und als
  $* = \Omega * \xrightarrow{\partial} X \xrightarrow{\id} X \to *$ gleichzeitig eine Fasersequenz.
\end{bsp}

\begin{prop}
  Sei $X \xrightarrow{f} Y \xrightarrow{g} Z$ eine Kofasersequenz (in $\Ho(\Cat_*)$). \\
  Dann ist $Y \xrightarrow{g} Z \xrightarrow{\partial} \Sigma X$ wieder eine Kofasersequenz, wobei die Kooperation von $\Sigma Y$ auf $\Sigma X$ wie folgt ist:
  \[ \Sigma X \xrightarrow{\text{Komult.}} \Sigma X \vee \Sigma X \xrightarrow{\id \vee \Sigma f} \Sigma X \vee \Sigma Y \xrightarrow{\id \vee \text{Inv.}} \Sigma X \vee \Sigma Y. \]
\end{prop}

\begin{dual}
  Ist $X \xrightarrow{f} Y \xrightarrow{g} Z$ eine Fasersequenz, so ist $\Omega Z \xrightarrow{\partial} X \xrightarrow{f} Y$ mit geeigneter Operation von $\Omega Y$ auf $\Omega Z$ eine Fasersequenz.
\end{dual}

\begin{kor}
  Ausgehend von $X \xrightarrow{f} Y$ gibt eine lange Sequenz
  \[ X \xrightarrow{f} Y \xrightarrow{g} Z = \cofib^\LL(f) \xrightarrow{\partial} \Sigma X \xrightarrow{- \Sigma f} \Sigma Y \xrightarrow{- \Sigma g} \Sigma Z \xrightarrow{- \Sigma \partial} \to \ldots \]
  in der jedes Objekt die Homotopiekofaser der vorh. Mor. ist.
\end{kor}

\begin{dual}
  $\ldots \to \Omega X \to \Omega Y \to \Omega Z \to X \to Y \to Z$
\end{dual}

\begin{satz}
  \begin{itemize}
    \item Ist $X \to Y \to Z$ eine Kofasersequenz, so ist
    \begin{align*}
      \ldots & \to [\Sigma Z, W] \to [\Sigma Y, W] \to [\Sigma X, W] \to \\
      & \to [Z, W] \to [Y, W] \to [X, W]
    \end{align*}
    für alle $W \in \Ob(\Ho(\Cat_*))$ eine exakte Sequenz von Gruppen bzw. punktierten Mengen.
    \item Ist $X \to Y \to Z$ eine Fasersequenz, so ist
    \begin{align*}
      \ldots & \to [W, \Omega X] \to [W, \Omega Y] \to [W, \Omega Z] \to \\
      & \to [W, X] \to [W, Y] \to [W, Z]
    \end{align*}
    für alle $W$ eine exakte Sequenz.
  \end{itemize}
\end{satz}

\begin{bem}
  Für $W = \Sph^0$ erhält man eine lange exakte Sequenz von Homotopiegruppen:
  \begin{align*}
    \ldots \to \pi_2(Z) & \to \pi_1(X) \to \pi_1(Y) \to \pi_1(Z) \to \\
    & \to \pi_0(X) \to \pi_0(Y) \to \pi_0(Z)
  \end{align*}
\end{bem}

% Bonusvorlesung der letzten Woche

\begin{lem}
  Es sei der durchgezogene Teil des folgenden Diagramms gegeben.
  Dann existiert der gestrichelte Teil.
  \begin{centertikzcd}
    X \arrow[r, "f"] \arrow[d] & Y \arrow[d] \arrow[r, dashed] & \cofib^\LL(f) \arrow[d, dashed, "s"] \\
    X' \arrow[r, "f'"] & Y' \arrow[r, dashed] & \cofib^\LL(f)
  \end{centertikzcd}
  Außerdem ist $s$ verträglich mit den Kooperationen.
\end{lem}

\begin{lem}
  Sei der durchgezogene Teil des folgenden Diagramms gegeben:
  \begin{centertikzcd}[column sep=0.5cm, row sep=0.3cm]
    X \arrow[rr] \arrow[rd, "v"] && Z \arrow[rr] \arrow[rd] && W \\
    & Y \arrow[ru, "u"] \arrow[rd, "d"] && V \arrow[ru, dashed, "s"] \\
    && U \arrow[ru, dashed, "r"]
  \end{centertikzcd}
  Dabei sind $X \to Y \to U$, $Y \to Z \to W$ und $X \to Z \to V$ Kofasersequenzen.
  Dann existieren $r$, $s$ wie eingezeichnet und $r$ ist $\Sigma U$-äquivariant, $s$ ist $\Sigma v$-äquivariant.
  Des Weiteren ist $U \to V \to W$ eine Kofasersequenz mit Kowirkung
  \[
    W \xrightarrow{\text{Koop}} W \vee \Sigma Y \xrightarrow{\id \vee \Sigma d} W \vee \Sigma U.
  \]
\end{lem}

\begin{kor}
  $\Sigma \cofib^\LL(uv) = \cofib^\LL(\cofib^\LL u \to \Sigma \cofib^\LL v)$
\end{kor}

\begin{lem}
  Sei $X \xrightarrow{f} Y \xrightarrow{g} Z$ eine Kofasersequenz in $\Ho(\Cat_*)$, $X' \xrightarrow{i} Y' \xrightarrow{p} Z'$ eine Fasersequenz.
  Dann gilt
  \begin{centertikzcd}
    X \arrow[r, "f"] \arrow[d, "\alpha"] &
    Y \arrow[r, "g"] \arrow[d, "\beta"] &
    Z \arrow[r, "\partial"] \arrow[d, dashed, "\exists"] &
    \Sigma X \arrow[d, "- \alpha^*"] \\
    \Omega Z' \arrow[r, "\partial"] &
    X' \arrow[r, "i"] &
    Y' \arrow[r, "p"] &
    Z'
  \end{centertikzcd}
\end{lem}

\begin{lem}
  Ist $F : \Cat_* \rightleftarrows \Dat_* : U$ eine Quillenadjunktion, so erhält $\LD{F}$ Kofasersequenzen und $\RD{U}$ Fasersequenzen.
\end{lem}

\begin{satz}
  Der Funktor $\blank \wedge^\LL \blank : \Ho(\Cat_*) \times \Simpl_* \to \Ho(\Cat_*)$ ist mit Kofasersequenzen verträglich, \dh{}
  \begin{itemize}
    \item Ist $A \in \Ob(\Ho(\Cat_*))$, $X \to Y \to Z$ eine Kofaserseq. in $\Simpl_*$, so ist $A \wedge^\LL X \to A \wedge^\LL Y \to A \wedge^\LL Z$ eine Kofaserseq. mit Kowirkung
    \begin{align*}
      A \wedge^\LL Z \to A \wedge^\LL (Z \vee \Sigma X) & \cong (A \wedge^\LL Z) \vee (A \wedge^\LL \Sigma X) \\
      & \cong (A \wedge^\LL Z) \vee \Sigma (A \wedge^\LL X).
    \end{align*}
    \item Ist $K \in \Simpl_*$, $X \to Y \to Z$ eine Kofasersequenz in $\Ho(\Cat_*)$, so ist $X \wedge^\LL K \to Y \wedge^\LL K \to Z \wedge^\LL K$ eine Kofasersequenz mit Kowirkung
    \begin{align*}
      Z \wedge^\LL K \to (Z \vee \Sigma X) \wedge^\LL K & \cong (Z \wedge^\LL K) \vee (\Sigma X \wedge^\LL K) \\
      & \cong (Z \wedge^\LL K) \vee \Sigma (X \wedge^\LL K).
    \end{align*}
  \end{itemize}
\end{satz}

\begin{acht}
  Man beachte das Vorzeichen:
  \begin{centertikzcd}
    \Sph^n \wedge^\LL \Sph^m \arrow{r}{\text{kan}}[swap]{{\sim}} \arrow{d}{\gamma}[swap]{{\sim}} &
    \Sph^{m+n} \arrow[d, "(-1)^{mn} \id"] \\
    \Sph^m \wedge^\LL \Sph^n \arrow{r}{\text{kan}}[swap]{{\sim}} &
    \Sph^{n+m} \\
  \end{centertikzcd}
\end{acht}

\subsection{Stabile Modellkategorien}

\begin{defn}
  Eine punktierte Modellkategorie $\Cat_*$ heißt stabil, falls $\Sigma : \Ho(\Cat_*) \to \Ho(\Cat_*)$ eine Kategorienäquivalenz ist.
\end{defn}

\begin{bem}
  Es folgt, dass dann $\Omega = \Sigma^{-1}$.
\end{bem}

\begin{prop}
  Damit wird $\Ho(\Cat_*)$ zu einer additiven Kategorie.
\end{prop}

\begin{proof}
  Jedes Objekt $X \in \Ob(\Ho(\Cat_*))$ ist natürlicherweise eine abelsche Gruppe: $X \cong \Omega^2 \Sigma^2 X$.
\end{proof}

\begin{bem}
  Die Kowirkung von $\Sigma X$ auf $Z$ zu einer Kofasersequenz $X \xrightarrow{f} Y \xrightarrow{g} Z$ in einer stabilen Modellkategorie ist schon durch $\partial : Z \to \Sigma X$ gegeben:
  \begin{centertikzcd}[column sep=0.8cm]
    Z \arrow[r, "\text{Kowirk}"] \arrow[bend right]{rr}{(\id, \partial)}{} &
    Z \vee \Sigma X \arrow[r, equal] &
    Z \times \Sigma X
  \end{centertikzcd}
  Produkt und Koprodukt fallen zusammen, da $\Ho(\Cat_*)$ additiv ist.
\end{bem}

\begin{lem}
  Sei $\Cat_*$ stabil. Dann gilt
  \begin{alignat*}{4}
    & X \xrightarrow{f} Y \xrightarrow{g} Z \xrightarrow{\partial} \Sigma X \quad &&
    \text{ist Kofaserseq.} \\
    \iff \quad & \Sigma X \xrightarrow{- \Sigma f} \Sigma Y \xrightarrow{- \Sigma g} \Sigma Z \xrightarrow{- \Sigma \partial} \Sigma^2 X \quad &&
    \text{ist Kofaserseq.}
  \end{alignat*}
\end{lem}

\begin{prop}
  Es gilt
  \begin{alignat*}{4}
    & X \xrightarrow{f} Y \xrightarrow{g} Z \xrightarrow{\partial} \Sigma X \quad &&
    \text{ist Kofaserseq.} \\
    \iff \quad & \Omega Z \xrightarrow{- \partial^*} X \xrightarrow{f} Y \xrightarrow{g} \Sigma \Omega Z = Z \quad &&
    \text{ist Faserseq.}
  \end{alignat*}
  Mit anderen Worten fallen Faser- und Kofasersequenzen zusammen und $\Ho(\Cat_*)$ ist eine triangulierte Kategorie.
\end{prop}

\begin{kor}
  Ist $X \to Y \to Z \xrightarrow{\partial} \Sigma X$ ein exaktes Dreieck, so ist
  \begin{align*}
    \ldots \to [W, \Sigma^{-1} Z] & \to [W, X] \to [W, Y] \to [W, Z] \to \\
    & \to [W, \Sigma X] \to [W, \Sigma Y] \to [W, \Sigma Z] \to \ldots
  \end{align*}
  eine lange exakte Sequenz.
\end{kor}

% Kapitel: Ausblick

% Ausgelassen: Ein Satz

\begin{defn}[\emph{Bousfield-Lokalisierung}]
  Sei $S \subseteq \Mor(\ModC)$ eine Menge von Mor. einer Modellkat. $\ModC$.
  Dann heißt $X \in \Ob(\ModC)$ \emph{$S$-lokal}, falls
  \[
    \fa{(A \xrightarrow{f} B) \in S} \Map(B, X) \xrightarrow{f^*} \Map(Y, X) \quad
    \text{ist schwache Äq.}
  \]
  Weiter heißt $f : A \to B$ eine \emph{$S$-lokale Äquivalenz}, falls für alle $S$-lokalen Objekte $X$ der Morphismus $f^*$ wie eben eine schwache Äquivalenz ist.
  (Offensichtlich gilt dann $S \subseteq \{ \text{ $S$-lokale Äq. } \}$.) \\
  Ist dann $\ModC$ links-eigentlich und kombinatorisch, so existiert eine Modellstruktur $L_S \ModC$ auf $\ModC$, deren Kofaserungen diesselben sind wie von $\ModC$ und deren schwachen Äq. die $S$-lokalen Äq. sind.
\end{defn}

% TODO: Definieren, was eine simpliziale Modellkategorie ist!
\begin{defn}
  Sei $\Cat_*$ eine (punktierte), simpliziale, links-eigentliche, kombinatorische Modellkategorie.
  Die \emph{Kategorie der Spektren} $\Cat_*^\infty$ hat als Objekte Familien $(X_i)_{i \in \N}$ von Objekten aus $\Cat_*$ zusammen mit Morphismen $(\alpha_i : \Sigma X_i \to X_{i+1})_{i \in \N}$.
  Wir schreiben
  \[ X^\infty : X_0 \dashrightarrow X_1 \dashrightarrow X_2 \dashrightarrow \ldots \]
  Ein Morphismus besteht aus Morphismen $(X_i \to X_i')_{i /in \N}$, welche mit den $\alpha_i$ und $\alpha_i'$ verträglich sind.
\end{defn}

\begin{bsp}
  Sei $X \in \Ob(\Cat_*)$.
  Dann heißt
  \[
    \Sigma^\infty X : X \dashrightarrow \Sigma X \dashrightarrow \Sigma^2 X \dashrightarrow \ldots \enspace \text{\emph{Einhängungsspektrum} von $X$.}
  \]
  Das \emph{Sphärenspektrum} ist $\Sph^\infty \coloneqq \Sigma^\infty \Sph^0$.
\end{bsp}

\begin{defn}
  Die \emph{projektive Modellstruktur} auf $\Cat_*^\infty$ ist diejenige, für die gilt:
  Ein Mor. $f : X^\infty \to Y^\infty$ ist eine
  \begin{itemize}
    \item schwache Äq. $\iff$ alle $f_i : X_i^\infty \to Y_i^\infty$ sind schw. Äq.
    \item Faserung $\iff$ alle $f_i : X_i^\infty \to Y_i^\infty$ sind Faserungen.
  \end{itemize}
\end{defn}

\begin{defn}
  Ein Spektrum $X^\infty$ heißt $\Omega$-Spektrum, falls alle $\alpha_i^* : X_i \to \Omega X_{i+1}$ schwache Äquivalenzen sind.
\end{defn}

\begin{lem}
  Es gibt eine Klasse $S \subseteq \Mor(\Cat_*^\infty)$, sodass gilt:
  \[
    X^\infty \text{ ist $S$-lokal} \iff
    X^\infty \text{ ist ein $\Omega$-Spektrum.}
  \]
\end{lem}

\begin{bem}
  Damit existiert die Bousfield-Lokalisierung der projektiven Modellstruktur auf $\Cat_*^\infty$ nach den $\Omega$-Spektren.
  Diese Modellstruktur ist die übliche Modellstruktur auf $\Cat_*^\infty$.
\end{bem}

\begin{satz}
  Mit dieser Modellstr. wird $\Cat_*^\infty$ zu einer stabilen Modellkat.
\end{satz}

\TODO{Eilenberg-MacLance-simpliziale-Mengen (K(G, n)'s)}

\TODO{Dold-Kan vom Übungsblatt}

\pagebreak

\input{include/ordinalzahlen} % Ordinalzahlen-Anhang

\end{document}
